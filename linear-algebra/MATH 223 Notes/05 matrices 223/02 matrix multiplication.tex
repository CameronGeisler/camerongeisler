\documentclass[letterpaper,12pt]{article}
\newcommand{\myname}{Cameron Geisler}
\newcommand{\mynumber}{90856741}
\usepackage{amsmath, amsfonts, amssymb, amsthm}
\usepackage[paper=letterpaper,left=25mm,right=25mm,top=3cm,bottom=25mm]{geometry}
\usepackage{fancyhdr}
\usepackage{float}
\usepackage{siunitx}
\usepackage{caption}
\usepackage{graphicx}
\pagestyle{fancy}
\usepackage{tkz-euclide} \usetkzobj{all} %% figures
\usepackage{hyperref} %% for links
\usepackage{exsheets} %% for tasks
\usepackage{systeme} %% for linear systems
\graphicspath{{../images/}} %% graphics in images folder

\lhead{Math 223} \chead{} \rhead{\myname}
\lfoot{} \cfoot{Page \thepage} \rfoot{}
\renewcommand{\headrulewidth}{0.4pt}
\renewcommand{\footrulewidth}{0.4pt}

\setlength{\parindent}{0pt}
\usepackage{enumerate}
\theoremstyle{definition}
\newtheorem*{definition}{Definition}
\newtheorem*{theorem}{Theorem}
\newtheorem*{example}{Example}
\newtheorem*{corollary}{Corollary}
\newtheorem*{lemma}{Lemma}
\newtheorem*{result}{Result}

%% Math
\newcommand{\abs}[1]{\left\lvert #1 \right\rvert}
\newcommand{\set}[1]{\left\{ #1 \right\}}
\renewcommand{\neg}{\sim}
\newcommand{\brac}[1]{\left( #1 \right)}
\newcommand{\eval}[1]{\left. #1 \right|}
\renewcommand{\vec}[1]{\mathbf{#1}}
\newenvironment{amatrix}[1]{\left[\begin{array}{@{}*{#1}{c}|c@{}}}{\end{array}\right]} %% for augmented matrix

\newcommand{\vecii}[2]{\left< #1, #2 \right>}
\newcommand{\veciii}[3]{\left< #1, #2, #3 \right>}

%% Linear algebra
\DeclareMathOperator{\Ker}{Ker}
\DeclareMathOperator{\nullity}{nullity}
\DeclareMathOperator{\Image}{Im}
\newcommand{\Span}[1]{\text{Span}\left(#1 \right)}
\DeclareMathOperator{\rank}{rank}
\DeclareMathOperator{\colrk}{colrk}
\DeclareMathOperator{\rowrk}{rowrk}
\DeclareMathOperator{\Row}{Row}
\DeclareMathOperator{\Col}{Col}
\DeclareMathOperator{\Null}{N}
\newcommand{\tr}[1]{tr\left( #1 \right)}
\DeclareMathOperator{\matref}{ref}
\DeclareMathOperator{\matrref}{rref}
\DeclareMathOperator{\sol}{Sol}
\newcommand{\inp}[2]{\left< #1, #2 \right>}
\newcommand{\norm}[1]{\left\lVert #1 \right\rVert}

%% Statistics
\newcommand{\prob}[1]{P\left( #1 \right)}
\newcommand{\overbar}[1]{\mkern 1.5mu \overline {\mkern-1.5mu#1 \mkern-1.5mu} \mkern 1.5mu}


\renewcommand{\frame}[1]{\tilde{\underline{\vec{#1}}}}

\chead{Matrix Multiplication}

\begin{document}

\section*{Product of Matrix with a Vector}
\begin{definition}
Let $A$ be an $m \times n$ matrix, $x = \begin{bmatrix} x_1 \\ \vdots \\ x_n \end{bmatrix} \in \mathbb{F}^n$ be a column vector ($n \times 1$ matrix). Then, the \textbf{matrix vector product}, $Ax$, is the $m \times 1$ matrix given by
\begin{equation*}
    Ax = \begin{bmatrix} \sum_{k=1}^n A_{1k} x_k \\ \vdots \\ \sum_{k=1}^n A_{mk} x_k \end{bmatrix}
\end{equation*}
\begin{equation*}
    (Ax)_i = \sum_{k=1}^n A_{ik} x_{k}
\end{equation*}
Equivalently, $Ax$ can be thought of as the linear combination of the columns of $A$, with coefficients $x_1, \dots, x_n$. In other words,
\begin{equation*}
    Ax = x_1 \begin{bmatrix} A_{11} \\ \vdots \\ A_{m1} \end{bmatrix} + \dots + x_n \begin{bmatrix} A_{1n} \\  \vdots \\ A_{mn} \end{bmatrix}
\end{equation*}
\end{definition}

\begin{theorem}
Let $A$ be a $m \times n$ matrix, $x$, $y \in \mathbb{F}^n$, $a \in \mathbb{F}$. Then,
\begin{enumerate}
    \item $A(x + y) = Ax + Ay$
    \item $A(ax) = a(Ax)$
\end{enumerate}
\end{theorem}

\section*{Matrix Multiplication (Products of Matrices)}
\begin{definition}
Let $A$ be a $m \times n$ matrix, $B$ be a $n \times p$. The \textbf{product} of $A$ and $B$, $AB$, is the $m \times p$ matrix such that
\begin{equation*}
    (AB)_{ij} = \sum_{k=1}^n A_{ik} B_{kj}
\end{equation*}
\begin{itemize}
    \item In other words, the entry in the $i$th row and $j$th column is the sum of products of the $i$th row of $A$ and $j$th column of $B$.
    \item In order for the product $AB$ to be defined, the number of columns of $A$ must be equal to the number of rows of $B$.
    \item ``$(m \times n) \cdot (n \times p) = (m \times p)$" In other words, the two ``inner" dimensions must be equal, and the two ``outer" dimensions are the dimensions of the product.
\end{itemize}
\end{definition}

\begin{example}
If $A$ has size $3 \times 1$, $B$ has size $1 \times 4$, then the product $AB$ will have size $3 \times 4$, but the product $BA$ does not exist.
\end{example}

\begin{example}
Determine the product if possible
\begin{equation*}
    \begin{bmatrix} 2 & -2 & 2 \\ 4 & 0 & 2 \end{bmatrix} \begin{bmatrix} -2 \\ 2 \\ 2 \end{bmatrix}
\end{equation*}
The first matrix has dimensions $2 \times 3$, the second has dimensions $3 \times 1$, so the product exists and has dimensions $2 \times 1$.
\begin{align*}
    \begin{bmatrix} 2 & -2 & 2 \\ 4 & 0 & 2 \end{bmatrix} \begin{bmatrix} -2 \\ 2 \\ 2 \end{bmatrix} & = \begin{bmatrix} 2(-2) + (-2)(2) + 2(2) \\ 4(-2) + 0(2) + 2(2) \end{bmatrix} \\
    & = \begin{bmatrix} -4 \\ -4 \end{bmatrix}
\end{align*}
\end{example}

\begin{example}
Determine the product if possible
\begin{equation*}
    \begin{bmatrix} -1 & -3 & -3 \\ 2 & -1 & 5 \end{bmatrix} \begin{bmatrix} 2 & 0 \\ -5 & 0 \\ -4 & 2 \end{bmatrix}
\end{equation*}
The first matrix has dimension $2 \times 3$, the second has dimension $3 \times 2$, so the product exists and has dimensions $2 \times 2$.
\begin{align*}
    \begin{bmatrix} -1 & -3 & -3 \\ 2 & -1 & 5 \end{bmatrix} \begin{bmatrix} 2 & 0 \\ -5 & 0 \\ -4 & 2 \end{bmatrix} & = \begin{bmatrix} -1(2) - 3(-5) - 3(-4) & -1(0) - 3(0) - 3(2) \\ 2(2) - 1(-5) + 5(-4) & 2(0) - 1(0) + 5(2) \end{bmatrix} \\
    & = \begin{bmatrix} 25 & -6 \\ -11 & 10 \end{bmatrix}
\end{align*}
\end{example}

\section*{Properties of Matrix Multiplication}
Matrix multiplication shares many properties of real number multiplication, assuming the matrices have the appropriate dimensions.

\begin{theorem}
Let $A$ be an $m \times n$ matrix, $B$ and $C$ be $n \times p$ matrices, $D$ and $E$ be $q \times m$ matrices, $a \in \mathbb{R}$. Then,
\begin{enumerate}[(a)]
    \item \textbf{Distributive over addition}.
    \begin{equation*}
        A(B + C) = AB + AC \qquad \text{and} \qquad (D + E)A = DA + EA
    \end{equation*}
    \item \textbf{Associative with a scalar}.
    \begin{equation*}
        a(AB) = (aA)B = A(aB)
    \end{equation*}
    \item \textbf{Associative with matrices}.
    \begin{equation*}
        A(BC) = (AB)C
    \end{equation*}
\end{enumerate}
\end{theorem}
\begin{proof}
\begin{align*}
    (A(B + C))_{ij} & = \sum_{k=1}^n A_{ik}(B + C)_{kj} && \text{definition of matrix multiplication} \\
    & = \sum_{k=1}^n A_{ik}(B_{kj} + C_{kj}) && \text{definition of matrix addition} \\
    & = \sum_{k=1}^n (A_{ik} B_{kj} + A_{ik} C_{kj}) \\
    & = \sum_{k=1}^n A_{ik}B_{kj} + \sum_{k=1}^n A_{ik}C_{kj} \\
    & = (AB)_{ij} + (AC)_{ij} && \text{definition of matrix multiplication} \\
    & = (AB + AC)_{ij} && \text{definition of matrix addition}
\end{align*}
\end{proof}

\section*{Matrix Multiplication is not Commutative}
Matrix multiplication, unlike multiplication of real numbers, is not commutative, in that in general, $AB \neq BA$.
\\ \\ In fact, for a $m \times n$ matrix $A$ and $n \times p$ matrix $B$, while $AB$ is defined like above, $BA$ is not defined, unless $m = p$. If $m = p$, then while $AB$ has dimensions $m \times m$ like above, $BA$ will have dimensions $n \times n$, and so clearly $AB$ cannot be equal to $BA$, unless $m = n$, or both matrices are square.
\\ \\ In this special case, matrix multiplication is still not commutative in general.

\section*{Identity Matrix}
Recall that for real numbers, multiplying a number by 1 results in the number being unchanged. In other words, 1 is the \textit{multiplicative identity} for multiplication of real numbers, $a \cdot 1 = a$ for all $a \in \mathbb{R}$. Similarly, we can define a matrix which, when multiplied by a matrix, leaves it unchanged, called an \textit{identity matrix}.

\begin{definition}
The $n \times n$ \textbf{identity matrix}, $I_n$, is the matrix with $1$'s along the main diagonal (from top left to bottom right), and $0$'s everywhere else. In other words,
\begin{equation*}
    I_n = \begin{bmatrix} 1 & 0 & \dots & 0 \\ 0 & 1 & \dots & 0 \\ \vdots & \vdots & \ddots & \vdots \\ 0 & 0 & \dots & 1 \end{bmatrix}
\end{equation*}
\end{definition}

For example, the $2 \times 2$ identity matrix $I_2$, and the $3 \times 3$ identity matrix $I_3$ are given by
\begin{equation*}
    I_2 = \begin{bmatrix} 1 & 0 \\ 0 & 1 \end{bmatrix} \qquad I_3 = \begin{bmatrix} 1 & 0 & 0 \\ 0 & 1 & 0 \\ 0 & 0 & 1 \end{bmatrix}
\end{equation*}

The definition of an identity matrix can be simplified by defining a symbol called the Kronecker delta. 

\begin{definition}
The \textbf{Kronecker delta}, $\delta_{ij}$, is defined to be
\begin{align*}
    \delta_{ij} = \begin{cases} 1 & \text{if $i = j$} \\ 0 & \text{if $i \neq j$} \end{cases}
\end{align*}
\end{definition}

For example, $\delta_{12} = 0$, and $\delta_{55} = 1$. Then, the $(i,j)$-th entry of the identity matrix is given by
\begin{equation*}
    (I_n)_{ij} = \delta_{ij}
\end{equation*}

Then, the identity matrix has the property that if $A$ is an $n \times n$ matrix, then
\begin{equation*}
    AI_n = I_n A = A
\end{equation*}

\section*{Product of Matrix and Identity Matrix}
The identity matrix is the multiplicative identity for matrices, in that the product of any matrix and the appropriate identity matrix is simply the matrix itself.
\begin{theorem}
Let $A$ be an $m \times n$ matrix. Then,
\begin{equation*}
    I_m A = A = AI_n
\end{equation*}
\begin{itemize}
    \item The identity matrix functions as the multiplicative identity, like 1 in arithmetic.
\end{itemize}
\end{theorem}
\begin{proof}
\begin{align*}
    (I_m A)_{ij} & = \sum_{k=1}^n (I_m)_{ik} A_{kj} && \text{definition of matrix multiplication} \\
    & = \sum_{k=1}^n \delta_{ik} A_{kj} && \text{definition of identity matrix} \\
    & = \delta_{ii} A_{ij} \\
    & = A_{ij}
\end{align*}
\begin{align*}
    (A I_n)_{ij} & = \sum_{k=1}^n A_{ik} (I_n)_{kj} && \text{definition of matrix multiplication} \\
    & = \sum_{k=1}^n A_{ik} \delta_{kj} && \text{definition of identity matrix} \\
    & = A_{ij} \delta_{jj} \\
    & = A_{ij}
\end{align*}
\end{proof}

\begin{theorem}
Let $V$ be an $n$-dimensional vector space, $B \subseteq V$ be an ordered basis of $V$. Then,
\begin{equation*}
    [\operatorname{Id}_{V}]_{B} = I_n
\end{equation*}
\end{theorem}

\end{document}