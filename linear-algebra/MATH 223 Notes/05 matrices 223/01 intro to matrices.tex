\documentclass[letterpaper,12pt]{article}
\newcommand{\myname}{Cameron Geisler}
\newcommand{\mynumber}{90856741}
\usepackage{amsmath, amsfonts, amssymb, amsthm}
\usepackage[paper=letterpaper,left=25mm,right=25mm,top=3cm,bottom=25mm]{geometry}
\usepackage{fancyhdr}
\usepackage{float}
\usepackage{siunitx}
\usepackage{caption}
\usepackage{graphicx}
\pagestyle{fancy}
\usepackage{tkz-euclide} \usetkzobj{all} %% figures
\usepackage{hyperref} %% for links
\usepackage{exsheets} %% for tasks
\usepackage{systeme} %% for linear systems
\graphicspath{{../images/}} %% graphics in images folder

\lhead{Math 223} \chead{} \rhead{\myname}
\lfoot{} \cfoot{Page \thepage} \rfoot{}
\renewcommand{\headrulewidth}{0.4pt}
\renewcommand{\footrulewidth}{0.4pt}

\setlength{\parindent}{0pt}
\usepackage{enumerate}
\theoremstyle{definition}
\newtheorem*{definition}{Definition}
\newtheorem*{theorem}{Theorem}
\newtheorem*{example}{Example}
\newtheorem*{corollary}{Corollary}
\newtheorem*{lemma}{Lemma}
\newtheorem*{result}{Result}

%% Math
\newcommand{\abs}[1]{\left\lvert #1 \right\rvert}
\newcommand{\set}[1]{\left\{ #1 \right\}}
\renewcommand{\neg}{\sim}
\newcommand{\brac}[1]{\left( #1 \right)}
\newcommand{\eval}[1]{\left. #1 \right|}
\renewcommand{\vec}[1]{\mathbf{#1}}
\newenvironment{amatrix}[1]{\left[\begin{array}{@{}*{#1}{c}|c@{}}}{\end{array}\right]} %% for augmented matrix

\newcommand{\vecii}[2]{\left< #1, #2 \right>}
\newcommand{\veciii}[3]{\left< #1, #2, #3 \right>}

%% Linear algebra
\DeclareMathOperator{\Ker}{Ker}
\DeclareMathOperator{\nullity}{nullity}
\DeclareMathOperator{\Image}{Im}
\newcommand{\Span}[1]{\text{Span}\left(#1 \right)}
\DeclareMathOperator{\rank}{rank}
\DeclareMathOperator{\colrk}{colrk}
\DeclareMathOperator{\rowrk}{rowrk}
\DeclareMathOperator{\Row}{Row}
\DeclareMathOperator{\Col}{Col}
\DeclareMathOperator{\Null}{N}
\newcommand{\tr}[1]{tr\left( #1 \right)}
\DeclareMathOperator{\matref}{ref}
\DeclareMathOperator{\matrref}{rref}
\DeclareMathOperator{\sol}{Sol}
\newcommand{\inp}[2]{\left< #1, #2 \right>}
\newcommand{\norm}[1]{\left\lVert #1 \right\rVert}

%% Statistics
\newcommand{\prob}[1]{P\left( #1 \right)}
\newcommand{\overbar}[1]{\mkern 1.5mu \overline {\mkern-1.5mu#1 \mkern-1.5mu} \mkern 1.5mu}


\renewcommand{\frame}[1]{\tilde{\underline{\vec{#1}}}}

\chead{Introduction to Matrices}

\begin{document}

\section*{Misc}


Matrices provide an efficient way to represent and solve systems of linear equations, and also to record data.

\section*{Matrices}
A matrix is an array of numbers arranged in (horizontal) rows and (vertical) columns, and within brackets.

\begin{definition}
Let $m, n \in \mathbb{N}$. An $m \times n$ \textbf{matrix} (read as ``$m$ by $n$ matrix"), is a rectangular array of $m$ rows and $n$ columns of real numbers,
\begin{equation*}
    \begin{bmatrix}
    a_{11} & a_{12} & \dots & a_{1n} \\
    a_{21} & a_{22} & & \vdots \\
    \vdots & & \ddots & \\
    a_{m1} & \dots & & a_{mn}
    \end{bmatrix}
\end{equation*}
Each \textbf{entry} (or \textbf{coefficient}, or \textbf{element}) in row $i$ and column $j$ is $a_{ij}$.
\begin{itemize}
    \item The plural of matrix is \textbf{matrices}.
    \item The \textbf{ith row} of a matrix $A$, is the row vector (or row matrix) $\begin{bmatrix} a_{i1} & \dots & a_{in} \end{bmatrix}$
    \item The \textbf{jth column} of a matrix $A$, is the column vector (or column matrix) $\begin{bmatrix} a_{1j} \\ \vdots \\ a_{mj} \end{bmatrix}$
    \item The set of all $m \times n$ matrices with coefficients in $\mathbb{F}$ is $M(m \times n, \mathbb{F})$
\end{itemize}
\end{definition}

Vectors can be represented as matrices. A row vector is a $1 \times n$ matrix (with 1 row and $n$ ``columns"), and a column vector is a $m \times 1$ matrix (with 1 column and $m$ ``rows").


\section*{Classification of Matrices}


\begin{example}
For example,
\begin{equation*}
    \begin{bmatrix} -4 & 4 \\ -5 & 9 \end{bmatrix}
\end{equation*}
is a $2 \times 2$ square matrix.
\end{example}


\section*{Upper/Lower Triangular Matrices}
\begin{definition}
$A$ is \textbf{upper triangular} if all the entries below the main diagonal are zero. In other words, for all $i > j$, $A_{ij} = 0$.
\end{definition}

\begin{definition}
$A$ is \textbf{lower triangular} if all the entries above the main diagonal are zero. In other words, for all $i < j$, $A_{ij} = 0$.
\end{definition}

\subsection*{Diagonal Matrices}
\begin{definition}
$A \in M(n \times n, \mathbb{F})$ is \textbf{diagonal} if and only if for all $i \neq j$, $A_{ij} = 0$. In other words, $A$ only has non-zero entries on the diagonal from top-left to bottom-right.
\end{definition}

\subsection*{Trace of a Matrix}
\begin{definition}
The \textbf{trace} of a matrix $A \in M(n \times n, \mathbb{F})$, $\operatorname{tr}(A)$, is the sum of the diagonal elements of $A$. In other words,
\begin{equation*}
    \operatorname{tr}(A) = A_{11} + A_{22} + \dots + A_{nn}
\end{equation*}
\end{definition}

\subsection*{Skew-Symmetric Matrices}
\begin{definition}
$A \in M(n \times n, \mathbb{F})$ is \textbf{skew-symmetric} if $A^T = -A$
\end{definition}

\section*{Matrix Operations Using a Graphing Calculator}




\end{document}