\documentclass[letterpaper,12pt]{article}
\newcommand{\myname}{Cameron Geisler}
\newcommand{\mynumber}{90856741}
\usepackage{amsmath, amsfonts, amssymb, amsthm}
\usepackage[paper=letterpaper,left=25mm,right=25mm,top=3cm,bottom=25mm]{geometry}
\usepackage{fancyhdr}
\usepackage{float}
\usepackage{siunitx}
\usepackage{caption}
\usepackage{graphicx}
\pagestyle{fancy}
\usepackage{tkz-euclide} \usetkzobj{all} %% figures
\usepackage{hyperref} %% for links
\usepackage{exsheets} %% for tasks
\usepackage{systeme} %% for linear systems
\graphicspath{{../images/}} %% graphics in images folder

\lhead{Math 223} \chead{} \rhead{\myname}
\lfoot{} \cfoot{Page \thepage} \rfoot{}
\renewcommand{\headrulewidth}{0.4pt}
\renewcommand{\footrulewidth}{0.4pt}

\setlength{\parindent}{0pt}
\usepackage{enumerate}
\theoremstyle{definition}
\newtheorem*{definition}{Definition}
\newtheorem*{theorem}{Theorem}
\newtheorem*{example}{Example}
\newtheorem*{corollary}{Corollary}
\newtheorem*{lemma}{Lemma}
\newtheorem*{result}{Result}

%% Math
\newcommand{\abs}[1]{\left\lvert #1 \right\rvert}
\newcommand{\set}[1]{\left\{ #1 \right\}}
\renewcommand{\neg}{\sim}
\newcommand{\brac}[1]{\left( #1 \right)}
\newcommand{\eval}[1]{\left. #1 \right|}
\renewcommand{\vec}[1]{\mathbf{#1}}
\newenvironment{amatrix}[1]{\left[\begin{array}{@{}*{#1}{c}|c@{}}}{\end{array}\right]} %% for augmented matrix

\newcommand{\vecii}[2]{\left< #1, #2 \right>}
\newcommand{\veciii}[3]{\left< #1, #2, #3 \right>}

%% Linear algebra
\DeclareMathOperator{\Ker}{Ker}
\DeclareMathOperator{\nullity}{nullity}
\DeclareMathOperator{\Image}{Im}
\newcommand{\Span}[1]{\text{Span}\left(#1 \right)}
\DeclareMathOperator{\rank}{rank}
\DeclareMathOperator{\colrk}{colrk}
\DeclareMathOperator{\rowrk}{rowrk}
\DeclareMathOperator{\Row}{Row}
\DeclareMathOperator{\Col}{Col}
\DeclareMathOperator{\Null}{N}
\newcommand{\tr}[1]{tr\left( #1 \right)}
\DeclareMathOperator{\matref}{ref}
\DeclareMathOperator{\matrref}{rref}
\DeclareMathOperator{\sol}{Sol}
\newcommand{\inp}[2]{\left< #1, #2 \right>}
\newcommand{\norm}[1]{\left\lVert #1 \right\rVert}

%% Statistics
\newcommand{\prob}[1]{P\left( #1 \right)}
\newcommand{\overbar}[1]{\mkern 1.5mu \overline {\mkern-1.5mu#1 \mkern-1.5mu} \mkern 1.5mu}


\renewcommand{\frame}[1]{\tilde{\underline{\vec{#1}}}}

\chead{Transpose of a Matrix}

\begin{document}

\section*{Transpose of a Matrix}
\begin{definition}
Let $A$ be a $m \times n$ matrix. The \textbf{transpose} of $A$, $A^T$, is the matrix that results from switching the rows and columns of $A$. The $j$th column of $A^T$ is the $j$th row of $A$, and vise versa.
\end{definition}

\begin{example}
If $A = \begin{bmatrix} a & b \\ c & d \end{bmatrix}$, then $A^T = \begin{bmatrix} a & c \\ b & d \end{bmatrix}$.
\end{example}

\section*{Transpose of a Matrix}
\begin{definition}
Let $A$ be a $m \times n$ matrix. The \textbf{transpose} of $A$, $A^T$, is the matrix that results from switching the rows and columns of $A$. The $k$th row becomes the $k$th column, and vise versa. In other words, if $A = \begin{bmatrix} a_{ij} \end{bmatrix}$, then
\begin{align*}
    (A^T)_{ij} = a_{ji}
\end{align*}
In particular, the $(i,j)$-entry of $A^T$ is the $(j,i)$-entry of $A$.
\end{definition}

Note that transposition has higher precedence of matrix multiplication and addition, so $AB^T$ represents $A(B^T)$, and $A + B^T = A + (B^T)$, as opposed to $(AB)^T$ and $(A + B)^T$ which is the transpose of the product matrix $AB$ and sum $A + B$, respectively.
\\ \\ Intuitively, transposition is a kind of ``reflection".

\section*{Basic Properties of the Transpose}
\begin{theorem}
Let $A, B$ be $m \times n$ matrices, $c \in \mathbb{R}$. Then,
\begin{enumerate}[(a)]
    \item \textbf{Transpose of a transpose}.
    \begin{equation*}
        (A^T)^T = A
    \end{equation*}
    \item \textbf{Transpose of a sum}.
    \begin{equation*}
        (A + B)^T = A^T + B^T
    \end{equation*}
    \item \textbf{Transpose of a scalar multiple}.
    \begin{equation*}
        (cA)^T = cA^T
    \end{equation*}
\end{enumerate}
\end{theorem}

In words, (a) says that the transpose of a transpose is equal to the original matrix. Properties (b) and (c) together mean that taking the transpose is a ``linear" operation. These properties are intuitively true, however the proofs are quite dense with subtle notation and subscripts.

\begin{proof}
Let $A = \begin{bmatrix} a_{ij} \end{bmatrix}, B = \begin{bmatrix} b_{ij} \end{bmatrix}$.
\begin{enumerate}[(a)]
    \item The $(i,j)$-entry of $A^T$ is the $(j,i)$-entry of $A$. Then, the $(i,j)$-entry of $(A^T)^T$ is the $(j,i)$-entry of $A^T$, which in turn is the $(i,j)$-entry of $A$. Thus, $(A^T)^T = A$. In notation, $(A^T)_{ij} = a_{ji}$, so $(A^T)^T_{ij} = a_{ij} = A_{ij}$, and so $(A^T)^T = A$.
    \item Then, $(A^T)_{ij} = a_{ji}$ and $(B^T)_{ij} = b_{ji}$. Also, $(A + B)_{ij} = a_{ij} + b_{ij}$. Then,
    \begin{align*}
        \underbrace{(A + B)^T_{ij}}_{\text{$(i,j)$-entry of $(A+B)^T$}} & = (A + B)_{ji} && \text{by definition of transpose} \\
        & = a_{ji} + b_{ji} && \text{as the $(j,i)$-entry of $A + B$ is $a_{ji} + b_{ji}$} \\
        & = (A^T)_{ij} + (B^T)_{ij}
    \end{align*}
    \item The $(i,j)$-entry of $cA$ is $ca_{ij}$ or $(cA)_{ij} = ca_{ij}$. Then, the $(i,j)$-entry of $(cA)^T$ is the $(j,i)$-entry of $cA$ which is $ca_{ji}$, and so $((cA)^T)_{ij} = ca_{ji}$. Thus, $(cA)^T = cA^T$.
\end{enumerate}
\end{proof}

\section*{Transpose of a Product}
\begin{theorem}
Let $A$ be an $m \times n$ matrix, $B$ be an $n \times p$ matrix. Then, the transpose of the product $BA$ is the product of the transposes in reverse order, or
\begin{equation*}
    \boxed{(AB)^T = B^T A^T}
\end{equation*}
\end{theorem}

In particular, $(AB)^T \neq A^T B^T$, even if the dimensions of $A^T, B^T$ are such that the product $A^T B^T$ is defined.

\begin{proof}
\begin{align*}
    (AB)^t_{ij} & = (AB)_{ji} && \text{definition of transpose} \\
    & = \sum_{k=1}^n A_{jk} B_{ki} && \text{definition of matrix multiplication} \\
    & = \sum_{k=1}^n A^t_{kj} B^t_{ik} && \text{definition of transpose} \\
    & = \sum_{k=1}^n B^t_{ik} A^t_{kj} \\
    (AB)^t_{ij} & = (B^t A^t)_{ij} && \text{definition of matrix multiplication}
\end{align*}
\end{proof}

More generally, the transpose of a product of matrices is equal to the product of their transposes in \textit{reverse order},
\begin{equation*}
    \boxed{(A_1 A_2 \cdots A_n)^T = A_n^T \cdots A_2^T A_1^T}
\end{equation*}

\end{document}