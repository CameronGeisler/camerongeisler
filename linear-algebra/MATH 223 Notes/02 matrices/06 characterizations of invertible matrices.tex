\documentclass[letterpaper,12pt]{article}
\newcommand{\myname}{Cameron Geisler}
\newcommand{\mynumber}{90856741}
\usepackage{amsmath, amsfonts, amssymb, amsthm}
\usepackage[paper=letterpaper,left=25mm,right=25mm,top=3cm,bottom=25mm]{geometry}
\usepackage{fancyhdr}
\usepackage{float}
\usepackage{siunitx}
\usepackage{caption}
\usepackage{graphicx}
\pagestyle{fancy}
\usepackage{tkz-euclide} \usetkzobj{all} %% figures
\usepackage{hyperref} %% for links
\usepackage{exsheets} %% for tasks
\usepackage{systeme} %% for linear systems
\graphicspath{{../images/}} %% graphics in images folder

\lhead{Math 223} \chead{} \rhead{\myname}
\lfoot{} \cfoot{Page \thepage} \rfoot{}
\renewcommand{\headrulewidth}{0.4pt}
\renewcommand{\footrulewidth}{0.4pt}

\setlength{\parindent}{0pt}
\usepackage{enumerate}
\theoremstyle{definition}
\newtheorem*{definition}{Definition}
\newtheorem*{theorem}{Theorem}
\newtheorem*{example}{Example}
\newtheorem*{corollary}{Corollary}
\newtheorem*{lemma}{Lemma}
\newtheorem*{result}{Result}

%% Math
\newcommand{\abs}[1]{\left\lvert #1 \right\rvert}
\newcommand{\set}[1]{\left\{ #1 \right\}}
\renewcommand{\neg}{\sim}
\newcommand{\brac}[1]{\left( #1 \right)}
\newcommand{\eval}[1]{\left. #1 \right|}
\renewcommand{\vec}[1]{\mathbf{#1}}
\newenvironment{amatrix}[1]{\left[\begin{array}{@{}*{#1}{c}|c@{}}}{\end{array}\right]} %% for augmented matrix

\newcommand{\vecii}[2]{\left< #1, #2 \right>}
\newcommand{\veciii}[3]{\left< #1, #2, #3 \right>}

%% Linear algebra
\DeclareMathOperator{\Ker}{Ker}
\DeclareMathOperator{\nullity}{nullity}
\DeclareMathOperator{\Image}{Im}
\newcommand{\Span}[1]{\text{Span}\left(#1 \right)}
\DeclareMathOperator{\rank}{rank}
\DeclareMathOperator{\colrk}{colrk}
\DeclareMathOperator{\rowrk}{rowrk}
\DeclareMathOperator{\Row}{Row}
\DeclareMathOperator{\Col}{Col}
\DeclareMathOperator{\Null}{N}
\newcommand{\tr}[1]{tr\left( #1 \right)}
\DeclareMathOperator{\matref}{ref}
\DeclareMathOperator{\matrref}{rref}
\DeclareMathOperator{\sol}{Sol}
\newcommand{\inp}[2]{\left< #1, #2 \right>}
\newcommand{\norm}[1]{\left\lVert #1 \right\rVert}

%% Statistics
\newcommand{\prob}[1]{P\left( #1 \right)}
\newcommand{\overbar}[1]{\mkern 1.5mu \overline {\mkern-1.5mu#1 \mkern-1.5mu} \mkern 1.5mu}


\renewcommand{\frame}[1]{\tilde{\underline{\vec{#1}}}}

\chead{Characterizations of Invertible Matrices}

\begin{document}


\section*{Invertible Matrix Theorem}
The previous results, along with the multiple characterizations of linear systems, form the \textit{invertible matrix theorem}, which provides a summary of many equivalent conditions for a square matrix being invertible.

\begin{theorem}
\textbf{Invertible matrix theorem}. Let $A$ be a square $n \times n$ matrix. Then, the following are equivalent:
\begin{enumerate}[(a)]
    \item $A$ is invertible.
    \item $A$ is row equivalent to the $n \times n$ identity matrix $I_n$.
    \item $A$ has $n$ pivot positions.
    \item The equation $A\vec{x} = \vec{0}$ has only the trivial solution.
    \item The columns of $A$ form a linearly independent set.
    \item The linear transformation of $A$ ($\vec{x} \mapsto A\vec{x}$) is one-to-one.
    \item The equation $A\vec{x} = \vec{b}$ has at least one solution for every $\vec{b} \in \mathbb{R}^n$.
    \item The columns of $A$ span $\mathbb{R}^n$.
    \item The linear transformation $\vec{x} \mapsto A\vec{x}$ maps $\mathbb{R}^n$ onto $\mathbb{R}^n$.
    \item There exists an $n \times n$ matrix $C$ such that $CA = I$.
    \item There exists an $n \times n$ matrix $D$ such that $AD = I$.
    \item $A^T$ is invertible.
\end{enumerate}
\end{theorem}

\begin{theorem}
Let $A, B$ be square matrices. If $AB = I$, then $A$ and $B$ are both invertible, and they are inverses of each other, $A = B^{-1}, B = A^{-1}$.
\end{theorem}

Recall that matrices $A, B$ are inverses if $AB = I$. The previous theorem means that the converse is also true: if $AB = I$, then $A, B$ are inverses. This implies that the set of all $n \times n$ matrices are divided into two disjoint classes: the invertible (non-singular) matrices, and the non-invertible (singular) matrices.


\end{document}