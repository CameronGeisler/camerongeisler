\documentclass[letterpaper,12pt]{article}
\newcommand{\myname}{Cameron Geisler}
\newcommand{\mynumber}{90856741}
\usepackage{amsmath, amsfonts, amssymb, amsthm}
\usepackage[paper=letterpaper,left=25mm,right=25mm,top=3cm,bottom=25mm]{geometry}
\usepackage{fancyhdr}
\usepackage{float}
\usepackage{siunitx}
\usepackage{caption}
\usepackage{graphicx}
\pagestyle{fancy}
\usepackage{tkz-euclide} \usetkzobj{all} %% figures
\usepackage{hyperref} %% for links
\usepackage{exsheets} %% for tasks
\usepackage{systeme} %% for linear systems
\graphicspath{{../images/}} %% graphics in images folder

\lhead{Math 223} \chead{} \rhead{\myname}
\lfoot{} \cfoot{Page \thepage} \rfoot{}
\renewcommand{\headrulewidth}{0.4pt}
\renewcommand{\footrulewidth}{0.4pt}

\setlength{\parindent}{0pt}
\usepackage{enumerate}
\theoremstyle{definition}
\newtheorem*{definition}{Definition}
\newtheorem*{theorem}{Theorem}
\newtheorem*{example}{Example}
\newtheorem*{corollary}{Corollary}
\newtheorem*{lemma}{Lemma}
\newtheorem*{result}{Result}

%% Math
\newcommand{\abs}[1]{\left\lvert #1 \right\rvert}
\newcommand{\set}[1]{\left\{ #1 \right\}}
\renewcommand{\neg}{\sim}
\newcommand{\brac}[1]{\left( #1 \right)}
\newcommand{\eval}[1]{\left. #1 \right|}
\renewcommand{\vec}[1]{\mathbf{#1}}
\newenvironment{amatrix}[1]{\left[\begin{array}{@{}*{#1}{c}|c@{}}}{\end{array}\right]} %% for augmented matrix

\newcommand{\vecii}[2]{\left< #1, #2 \right>}
\newcommand{\veciii}[3]{\left< #1, #2, #3 \right>}

%% Linear algebra
\DeclareMathOperator{\Ker}{Ker}
\DeclareMathOperator{\nullity}{nullity}
\DeclareMathOperator{\Image}{Im}
\newcommand{\Span}[1]{\text{Span}\left(#1 \right)}
\DeclareMathOperator{\rank}{rank}
\DeclareMathOperator{\colrk}{colrk}
\DeclareMathOperator{\rowrk}{rowrk}
\DeclareMathOperator{\Row}{Row}
\DeclareMathOperator{\Col}{Col}
\DeclareMathOperator{\Null}{N}
\newcommand{\tr}[1]{tr\left( #1 \right)}
\DeclareMathOperator{\matref}{ref}
\DeclareMathOperator{\matrref}{rref}
\DeclareMathOperator{\sol}{Sol}
\newcommand{\inp}[2]{\left< #1, #2 \right>}
\newcommand{\norm}[1]{\left\lVert #1 \right\rVert}

%% Statistics
\newcommand{\prob}[1]{P\left( #1 \right)}
\newcommand{\overbar}[1]{\mkern 1.5mu \overline {\mkern-1.5mu#1 \mkern-1.5mu} \mkern 1.5mu}


\renewcommand{\frame}[1]{\tilde{\underline{\vec{#1}}}}

\chead{Introduction to Vector Spaces}

\begin{document}

Recall that in $\mathbb{R}^n$, vector addition and scalar multiplication satisfy certain properties. Other mathematical objects, such as matrices, polynomials, and functions also satisfy these properties, given suitable definitions of addition and scalar multiplication. This leads to the abstract definition of a \textbf{vector space}.

\section*{Vector Spaces}
\begin{definition}
A \textbf{real vector space} is a triple, $(V, +, \cdot)$, where $V$ is a set, $+$ and $\cdot$ are binary operations called \textbf{vector addition} (or simply \textbf{addition}) and \textbf{scalar multiplication},
\begin{align*}
    +: V \times V & \longrightarrow V \\
    (x,y) & \longmapsto x + y
\end{align*}
\begin{align*}
    \cdot: \mathbb{R} \times V & \longrightarrow V \\
    (\lambda, x) & \longmapsto \lambda \cdot x
\end{align*}
such that for all $x, y, z \in V$ and $\lambda, \mu \in \mathbb{R}$,
\begin{enumerate}[(a)]
    \item \textbf{Closed under addition}. For all 
    \item \textbf{Addition associative}. For all $x, y, z \in V$,
    \begin{equation*}
        (x + y) + z = x + (y + z)
    \end{equation*}
    \item \textbf{Addition commutative}. For all $x, y \in V$,
    \begin{equation*}
        x + y = y + x
    \end{equation*}
    \item \textbf{Existence of additive identity}. There exists an element $0 \in V$ such that for all $x \in V$,
    \begin{equation*}
        x + 0 = 0 + x = x
    \end{equation*}
    \item \textbf{Existence of additive inverse}. For all $x \in V$, there exists $-x \in V$ such that
    \begin{equation*}
        x + (-x) = 0
    \end{equation*}
    \item \textbf{Scalar multiplication associative}. For all $x \in V$, $\lambda, \mu \in \mathbb{R}$,
    \begin{equation*}
        \lambda(\mu x) = (\lambda \mu) x
    \end{equation*}
    \item \textbf{Multiplicative identity}. For all $x \in V$,
    \begin{equation*}
        1x = x
    \end{equation*}
    \item \textbf{Scalar distributive over addition}. For all $x, y \in V$, $\lambda \in \mathbb{R}$,
    \begin{equation*}
        \lambda(x + y) = \lambda x + \lambda y
    \end{equation*}
    \item \textbf{Vector distributive over addition}. For all $x \in V$, $\lambda, \mu \in \mathbb{R}$,
    \begin{equation*}
        (\lambda + \mu)x = \lambda x + \mu x
    \end{equation*}
\end{enumerate}
\end{definition}

\section*{Properties of the Axioms}

\begin{theorem}
The additive identity $0 \in V$ is unique.
\end{theorem}
\begin{proof}
By contradiction, suppose $\exists 0$, $0' \in V$, $0 \neq 0'$, such that $\forall x \in V$, $x + 0 = x$ and $x + 0' = x$. Then,
\begin{align*}
    0 & = 0 + 0' && \text{as $x = x + 0'$} \\
    & = 0' + 0 && \text{by commutivity} \\
    & = 0' && \text{as $x = x + 0$}
\end{align*}
Thus, $0 = 0'$, which contradicts our assumption that $0 \neq 0'$.
\end{proof}

\begin{corollary}
For each $x \in V$, the additive inverse $-x \in V$ is unique.
\end{corollary}
\begin{proof}
Let $x \in V$. By contradiction, suppose $\exists a$, $b \in V$, $a \neq b$, such that $x + a = 0$ and $x + b = 0$. Then,
\begin{align*}
    a & = a + 0 && \text{additive identity} \\
    & = a + (x + b) && \text{as $0 = x + b$} \\
    & = (a + x) + b && \text{by associativity} \\
    & = (x + a) + b && \text{by commutivity} \\
    & = 0 + b && \text{as $x + a = 0$} \\
    & = b + 0 && \text{by commutivity} \\
    & = b
\end{align*}
Thus, $a = b$, which contradicts our assumption that $a \neq b$.
\end{proof}

\begin{theorem}
Let $V$ be a vector space over $\mathbb{F}$, $x \in V$, $\lambda \in \mathbb{F}$.
\begin{itemize}
    \item $\lambda 0 = 0$
    \item $0x = 0$
    \item $\lambda x = 0$ if and only if $\lambda = 0$ or $x = 0$
\end{itemize}
\end{theorem}

\section*{$\mathbb{R}^n$ as a Vector Space}
\begin{example}
$(\mathbb{R}^n, +, \cdot)$ is a real vector space, where $\mathbb{R}^n = \set{(x_1, \dots, x_n): x_1, \dots, x_n \in \mathbb{R}}$. Addition and scalar multiplication is defined as
\begin{align*}
    + : \mathbb{R}^n \times \mathbb{R}^n & \longrightarrow \mathbb{R}^n \\
    ((x_1, \dots, x_n), (y_1, \dots, y_n)) & \longmapsto (x_1 + y_1, \dots, x_n + y_n)
\end{align*}
\begin{align*}
    \cdot : \mathbb{R} \times \mathbb{R}^n & \longrightarrow \mathbb{R}^n \\
    (\lambda, (x_1, \dots, x_n)) & \longmapsto (\lambda x_1, \dots, \lambda x_n)
\end{align*}
\begin{enumerate}
    \item Let $x$, $y$, $z \in \mathbb{R}^n$. Then,
    \begin{align*}
        (x + y) + z & = (x_1 + y_1, \dots, x_n + y_n) + (z_1, \dots, z_n) \\
        & = ((x_1 + y_1) + z_1, \dots, (x_n + y_n) + z_n) \\
        & = (x_1 + (y_1 + z_1), \dots, x_n + (y_n + z_n)) && \text{by associativity of $+$ in $\mathbb{R}$} \\
        & = x + (y + z)
    \end{align*}
\end{enumerate}
\end{example}


\begin{example}
The set of integers, with the usual addition and scalar multiplication, does not form a vector space, as it is not closed under addition.
\end{example}

\section*{Function Spaces}
\begin{definition}
A \textbf{vector space of functions} (or a \textbf{function space}) is a set of functions, all with a common domain, such that
\begin{enumerate}[(a)]
    \item The zero function, $f(x) = 0$, is in $V$
    \item If $f$, $g \in V$, then $f + g \in V$
    \item If $f \in V$, $t \in \mathbb{R}$, then $tf \in \mathbb{R}$
\end{enumerate}
\end{definition}

\begin{example}
\textbf{Vector space of polynomials}. The set of polynomials of degree at most $n$ (where $n \in \mathbb{N}$) is a vector space. For example, consider the set $P_2$ of all polynomials of degree at most 2,
\begin{equation*}
    P_2 = \set{a_0 + a_1 x + a_2 x^2: a_0, a_1, a_2 \in \mathbb{R}}
\end{equation*}
\end{example}

\begin{example}
\textbf{Vector space of continuous functions}. The set of all continuous functions on $\mathbb{R}$, or on an interval $[a,b]$, is a vector space.
\end{example}

\begin{example}
The set of all degree 2 polynomials is not a vector space, under the usual addition and scalar multiplication.
\end{example}


\section*{Complex Vector Spaces, Vector Spaces over General Fields}
A complex vector space is defined analogously to that for a real vector space, except with $\mathbb{C}$ replacing $\mathbb{R}$ as the field of scalars. In fact, more generally, we can consider a vector space $V$ over a general field $\mathbb{F}$.

\end{document}