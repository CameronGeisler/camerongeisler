\documentclass[letterpaper,12pt]{article}
\newcommand{\myname}{Cameron Geisler}
\newcommand{\mynumber}{90856741}
\usepackage{amsmath, amsfonts, amssymb, amsthm}
\usepackage[paper=letterpaper,left=25mm,right=25mm,top=3cm,bottom=25mm]{geometry}
\usepackage{fancyhdr}
\usepackage{float}
\usepackage{siunitx}
\usepackage{caption}
\usepackage{graphicx}
\pagestyle{fancy}
\usepackage{tkz-euclide} \usetkzobj{all} %% figures
\usepackage{hyperref} %% for links
\usepackage{exsheets} %% for tasks
\usepackage{systeme} %% for linear systems
\graphicspath{{../images/}} %% graphics in images folder

\lhead{Math 223} \chead{} \rhead{\myname}
\lfoot{} \cfoot{Page \thepage} \rfoot{}
\renewcommand{\headrulewidth}{0.4pt}
\renewcommand{\footrulewidth}{0.4pt}

\setlength{\parindent}{0pt}
\usepackage{enumerate}
\theoremstyle{definition}
\newtheorem*{definition}{Definition}
\newtheorem*{theorem}{Theorem}
\newtheorem*{example}{Example}
\newtheorem*{corollary}{Corollary}
\newtheorem*{lemma}{Lemma}
\newtheorem*{result}{Result}

%% Math
\newcommand{\abs}[1]{\left\lvert #1 \right\rvert}
\newcommand{\set}[1]{\left\{ #1 \right\}}
\renewcommand{\neg}{\sim}
\newcommand{\brac}[1]{\left( #1 \right)}
\newcommand{\eval}[1]{\left. #1 \right|}
\renewcommand{\vec}[1]{\mathbf{#1}}
\newenvironment{amatrix}[1]{\left[\begin{array}{@{}*{#1}{c}|c@{}}}{\end{array}\right]} %% for augmented matrix

\newcommand{\vecii}[2]{\left< #1, #2 \right>}
\newcommand{\veciii}[3]{\left< #1, #2, #3 \right>}

%% Linear algebra
\DeclareMathOperator{\Ker}{Ker}
\DeclareMathOperator{\nullity}{nullity}
\DeclareMathOperator{\Image}{Im}
\newcommand{\Span}[1]{\text{Span}\left(#1 \right)}
\DeclareMathOperator{\rank}{rank}
\DeclareMathOperator{\colrk}{colrk}
\DeclareMathOperator{\rowrk}{rowrk}
\DeclareMathOperator{\Row}{Row}
\DeclareMathOperator{\Col}{Col}
\DeclareMathOperator{\Null}{N}
\newcommand{\tr}[1]{tr\left( #1 \right)}
\DeclareMathOperator{\matref}{ref}
\DeclareMathOperator{\matrref}{rref}
\DeclareMathOperator{\sol}{Sol}
\newcommand{\inp}[2]{\left< #1, #2 \right>}
\newcommand{\norm}[1]{\left\lVert #1 \right\rVert}

%% Statistics
\newcommand{\prob}[1]{P\left( #1 \right)}
\newcommand{\overbar}[1]{\mkern 1.5mu \overline {\mkern-1.5mu#1 \mkern-1.5mu} \mkern 1.5mu}


\renewcommand{\frame}[1]{\tilde{\underline{\vec{#1}}}}

\chead{Coordinates and Change of Basis}

\begin{document}

Recall that if $B$ is a basis for a vector space $V$, then every vector $\vec{x} \in V$ can be written uniquely as a linear combination of the vectors of $B$. The scalars which make up this linear combination are called the \textbf{coordinates of $\vec{x}$ relative to $B$}.

\begin{definition}
Let $V$ be a vector space, $B = \set{\vec{v}_1, \dots, \vec{v}_n}$ be an ordered basis, $\vec{x} \in V$ with
\begin{equation*}
    \vec{x} = a_1 \vec{v}_1 + \dots + a_n \vec{v}_n
\end{equation*}
Then, the scalars $a_1, \dots, a_n$ are the \textbf{coordinates of $\vec{x}$ relative to the basis $B$}. The \textbf{coordinate vector} (or \textbf{coordinate matrix}) of $\vec{x}$ relative to $B$, is the column vector in $\mathbb{R}^n$ whose components are the coordinate of $\vec{x}$,
\begin{equation*}
    [\vec{x}]_B = \begin{bmatrix} a_1 \\ \vdots \\ a_n \end{bmatrix}
\end{equation*}
\end{definition}

Note that the basis is ordered, so the coordinates are ordered.

\section*{Change of Basis}

\begin{definition}
Let $B = \set{\vec{v}_1, \dots, \vec{v}_n}, B' = \set{\vec{u}_1, \dots, \vec{u}_n}$ are basis of $V$, and $\vec{x} \in V$ with coordinate vectors,
\begin{equation*}
    [\vec{x}]_B = \begin{bmatrix} a_1 \\ \vdots \\ a_n \end{bmatrix} \qquad [\vec{x}]_{B'} = \begin{bmatrix} b_1 \\ \vdots \\ b_n \end{bmatrix}
\end{equation*}
Then, the \textbf{transition matrix} from $B'$ to $B$ is the matrix $P$ such that,
\begin{equation*}
    [\vec{x}]_B = P [\vec{x}]_{B'}
\end{equation*}
\end{definition}


\begin{theorem}
If $P$ is the transition matrix from a basis $B'$ to a basis $B$ in $\mathbb{R}^n$, then $P$ is invertible and the transition matrix from $B$ to $B'$ is $P^{-1}$.
\end{theorem}



\section*{Misc Change of Coordinates}


\begin{example}
Let $R: \mathbb{R}^2 \rightarrow \mathbb{R}^2$ be a map that is reflection across the line $L$, $2x + 3y = 0$, or $L = \Span{\begin{pmatrix} 3 \\ -2 \end{pmatrix}}$.
\\ \\ We want to determine the matrix $[R]$ with respect to the standard basis $\phi = (e_1, e_2)$.
\\ \\ Formula is $[R]_{\phi}^{\phi} = [id]_{\phi}^B [R]_{B}^B [id]_B^{\phi}$
\\ \\ The change of basis matrix is $\begin{pmatrix} 3 & 2 \\ -2 & 3 \end{pmatrix}$
\\ \\ $B$ is the basis $\begin{pmatrix} 3 \\ -2 \end{pmatrix}$, $\begin{pmatrix} 2 \\ 3 \end{pmatrix}$
\\ \\ Thus, $[R] = \begin{pmatrix} 3 & 2 \\ -2 & 3 \end{pmatrix} [R]_B^B \begin{pmatrix} 3 & 2 \\ -2 & 3 \end{pmatrix}^{-1}$
\\ \\ We get
\begin{align*}
    R\begin{pmatrix} x' \\ y' \end{pmatrix} & = \begin{pmatrix} x' \\ -y' \end{pmatrix} \\
    & = \begin{pmatrix} 1 & 0 \\ 0 & -1 \end{pmatrix} \begin{pmatrix} x' \\ y' \end{pmatrix} \\
    [R]_B^B = \begin{pmatrix} 1 & 0 \\ 0 & -1 \end{pmatrix}
\end{align*}
Alternatively, $R\begin{pmatrix} 3 \\ -2 \end{pmatrix} = \begin{pmatrix} 3 \\ -2 \end{pmatrix} = 1 \begin{pmatrix} 3 \\ -2 \end{pmatrix} + 0 \begin{pmatrix} 2 \\ 3 \end{pmatrix}$
\\ \\ And, $R\begin{pmatrix} 2 \\ 3 \end{pmatrix} = \begin{pmatrix} 3 \\ -2 \end{pmatrix} = 0 \begin{pmatrix} 3 \\ -2 \end{pmatrix} -1 \begin{pmatrix} 2 \\ 3 \end{pmatrix}$
Thus,
$[R]_{\phi}^{\phi} = \begin{pmatrix} 3 & 2 \\ -2 & 3 \end{pmatrix} \begin{pmatrix} 1 & 0 \\ 0 & -1 \end{pmatrix} \begin{pmatrix} 3 & 2 \\ -2 & 3 \end{pmatrix}^{-1}$
\end{example}

\begin{example}
Let $R_{\theta}$ be rotation by $\theta$ counter-clockwise in $\mathbb{R}^2$. Rotate the coordinate system $B = (R_{\theta}e_1, R_{\theta}e_2)$
\\ \\ Change of basis matrix $[id]_{\phi}^B = (R_{\theta}e_1, R_{\theta}e_2) = R_{\theta}I_2 = R_{\theta}$
\\ \\ The change of basis matrix is equal to the rotation matrix.
\\ \\ $[v]_B = [id]_B^{\phi} [v]_{\phi}$
\\ $[v]_B = R_{\theta}^{-1} [v]_{\phi}$
\\ $\begin{pmatrix} x' \\ y' \end{pmatrix} = R_{\theta}^{-1} \begin{pmatrix} x \\ y \end{pmatrix}$
\\ The inverse of the change of basis matrix, $R_{\theta}^{-1}$, converts into new coordinates.
\end{example}

\end{document}