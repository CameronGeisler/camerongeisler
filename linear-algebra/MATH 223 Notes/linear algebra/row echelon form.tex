\documentclass[letterpaper,12pt]{article}
\newcommand{\myname}{Cameron Geisler}
\newcommand{\mynumber}{90856741}
\usepackage{amsmath, amsfonts, amssymb, amsthm}
\usepackage[paper=letterpaper,left=25mm,right=25mm,top=3cm,bottom=25mm]{geometry}
\usepackage{fancyhdr}
\usepackage{float}
\usepackage{siunitx}
\usepackage{caption}
\usepackage{graphicx}
\pagestyle{fancy}
\usepackage{tkz-euclide} \usetkzobj{all} %% figures
\usepackage{hyperref} %% for links
\usepackage{exsheets} %% for tasks
\usepackage{systeme} %% for linear systems
\graphicspath{{../images/}} %% graphics in images folder

\lhead{Math 223} \chead{} \rhead{\myname}
\lfoot{} \cfoot{Page \thepage} \rfoot{}
\renewcommand{\headrulewidth}{0.4pt}
\renewcommand{\footrulewidth}{0.4pt}

\setlength{\parindent}{0pt}
\usepackage{enumerate}
\theoremstyle{definition}
\newtheorem*{definition}{Definition}
\newtheorem*{theorem}{Theorem}
\newtheorem*{example}{Example}
\newtheorem*{corollary}{Corollary}
\newtheorem*{lemma}{Lemma}
\newtheorem*{result}{Result}

%% Math
\newcommand{\abs}[1]{\left\lvert #1 \right\rvert}
\newcommand{\set}[1]{\left\{ #1 \right\}}
\renewcommand{\neg}{\sim}
\newcommand{\brac}[1]{\left( #1 \right)}
\newcommand{\eval}[1]{\left. #1 \right|}
\renewcommand{\vec}[1]{\mathbf{#1}}
\newenvironment{amatrix}[1]{\left[\begin{array}{@{}*{#1}{c}|c@{}}}{\end{array}\right]} %% for augmented matrix

\newcommand{\vecii}[2]{\left< #1, #2 \right>}
\newcommand{\veciii}[3]{\left< #1, #2, #3 \right>}

%% Linear algebra
\DeclareMathOperator{\Ker}{Ker}
\DeclareMathOperator{\nullity}{nullity}
\DeclareMathOperator{\Image}{Im}
\newcommand{\Span}[1]{\text{Span}\left(#1 \right)}
\DeclareMathOperator{\rank}{rank}
\DeclareMathOperator{\colrk}{colrk}
\DeclareMathOperator{\rowrk}{rowrk}
\DeclareMathOperator{\Row}{Row}
\DeclareMathOperator{\Col}{Col}
\DeclareMathOperator{\Null}{N}
\newcommand{\tr}[1]{tr\left( #1 \right)}
\DeclareMathOperator{\matref}{ref}
\DeclareMathOperator{\matrref}{rref}
\DeclareMathOperator{\sol}{Sol}
\newcommand{\inp}[2]{\left< #1, #2 \right>}
\newcommand{\norm}[1]{\left\lVert #1 \right\rVert}

%% Statistics
\newcommand{\prob}[1]{P\left( #1 \right)}
\newcommand{\overbar}[1]{\mkern 1.5mu \overline {\mkern-1.5mu#1 \mkern-1.5mu} \mkern 1.5mu}


\renewcommand{\frame}[1]{\tilde{\underline{\vec{#1}}}}

\chead{}

\begin{document}

Let $A \in M(m \times n, \mathbb{F})$. Then,

\begin{definition}
$A$ is in \textbf{row echelon form} (REF) if and only if
\begin{enumerate}
    \item Every \textbf{zero row} (rows with all zeros) is below every \textbf{non-zero row} (rows with at least one non-zero entry).
    \item The first non-zero entry of every non-zero row is strictly to the right of the first non-zero entry of every row above it.
\end{enumerate}

\begin{equation*}
\begin{pmatrix}
    0 & \dots & 0 & p_1 & & & & & & \\
    0 & \dots & \dots & 0 & p_2 \\
    0 & \dots & \dots & \dots & 0 & p_3 \\
    \dots & \dots & & & & \\
    0 & \dots & & & & & \dots & 0 & p_k \\
    0 & \dots & & & & & & & \dots & 0 \\
    \dots & & & & & & & & & \dots \\
    0 & \dots & & & & & & & \dots & 0 \\
\end{pmatrix}  
\end{equation*}

\begin{itemize}
    \item $k$ is the number of non-zero rows
    \item The \textbf{pivot entries} or \textbf{pivots}, $p_1, \dots, p_k$, are the leading entries of the non-zero rows.
    \item The \textbf{pivot columns}, $j_1, \dots, j_n$, are the columns with a pivot entry.
\end{itemize}
\end{definition}

\begin{theorem}
Every matrix $A$ can be converted to a matrix in REF, $\ref{A}$, by applying a sequence of elementary row operations.
\begin{itemize}
    \item This provides an algorithm to compute the rank of a matrix $A$, to find $A^{-1}$, and to find a basis for $\Col{A}$
\end{itemize}
\end{theorem}

\begin{theorem}
If $A$ is in REF, then the rows of $A$ are linearly independent.
\end{theorem}
\begin{proof}
We want to show that if $\lambda_1 A^1 + \lambda_2 A^2 + \dots + \lambda_n A^n = 0$, then $\lambda_1 = \dots = \lambda_n = 0$.
\\ \\ Let $\lambda_1 A^1 + \lambda_2 A^2 + \dots + \lambda_n A^n = 0$.
\\ \\ CONTINUE PROOF
\end{proof}
\begin{corollary}
If $A$ is in REF, then the rows of $A$ are a basis for the rowspace of $A$.
\end{corollary}

\begin{theorem}
If $A$ is in REF, then the pivot columns of $A$ are linearly independent.
\end{theorem}
\begin{proof}
PROOF HERE
\end{proof}
\begin{corollary}
If $A$ is in REF, then the pivot columns of $A$ are a basis for the column space of $A$.
\begin{itemize}
    \item To find a basis for a set of vectors, put them into a matrix, put the matrix in REF. The pivot columns form the basis of those vectors.
\end{itemize}
\end{corollary}


\begin{definition}
$A$ is in reduced row echelon form if and only if
\begin{enumerate}
    \item $A$ is in REF.
    \item The pivot entry in each non-zero row is $1$.
    \item In every column with a pivot, the pivot is the only non-zero entry in the column. (Recall: in REF, only the entries above the pivot must be zero)
\end{enumerate}
Notation: The matrix $A$ converted to row echelon form obtained by elementary row operations is $\matrref{A}$.
\end{definition}



\end{document}