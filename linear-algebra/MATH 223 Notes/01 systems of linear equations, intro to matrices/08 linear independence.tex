\documentclass[letterpaper,12pt]{article}
\newcommand{\myname}{Cameron Geisler}
\newcommand{\mynumber}{90856741}
\usepackage{amsmath, amsfonts, amssymb, amsthm}
\usepackage[paper=letterpaper,left=25mm,right=25mm,top=3cm,bottom=25mm]{geometry}
\usepackage{fancyhdr}
\usepackage{float}
\usepackage{siunitx}
\usepackage{caption}
\usepackage{graphicx}
\pagestyle{fancy}
\usepackage{tkz-euclide} \usetkzobj{all} %% figures
\usepackage{hyperref} %% for links
\usepackage{exsheets} %% for tasks
\usepackage{systeme} %% for linear systems
\graphicspath{{../images/}} %% graphics in images folder

\lhead{Math 223} \chead{} \rhead{\myname}
\lfoot{} \cfoot{Page \thepage} \rfoot{}
\renewcommand{\headrulewidth}{0.4pt}
\renewcommand{\footrulewidth}{0.4pt}

\setlength{\parindent}{0pt}
\usepackage{enumerate}
\theoremstyle{definition}
\newtheorem*{definition}{Definition}
\newtheorem*{theorem}{Theorem}
\newtheorem*{example}{Example}
\newtheorem*{corollary}{Corollary}
\newtheorem*{lemma}{Lemma}
\newtheorem*{result}{Result}

%% Math
\newcommand{\abs}[1]{\left\lvert #1 \right\rvert}
\newcommand{\set}[1]{\left\{ #1 \right\}}
\renewcommand{\neg}{\sim}
\newcommand{\brac}[1]{\left( #1 \right)}
\newcommand{\eval}[1]{\left. #1 \right|}
\renewcommand{\vec}[1]{\mathbf{#1}}
\newenvironment{amatrix}[1]{\left[\begin{array}{@{}*{#1}{c}|c@{}}}{\end{array}\right]} %% for augmented matrix

\newcommand{\vecii}[2]{\left< #1, #2 \right>}
\newcommand{\veciii}[3]{\left< #1, #2, #3 \right>}

%% Linear algebra
\DeclareMathOperator{\Ker}{Ker}
\DeclareMathOperator{\nullity}{nullity}
\DeclareMathOperator{\Image}{Im}
\newcommand{\Span}[1]{\text{Span}\left(#1 \right)}
\DeclareMathOperator{\rank}{rank}
\DeclareMathOperator{\colrk}{colrk}
\DeclareMathOperator{\rowrk}{rowrk}
\DeclareMathOperator{\Row}{Row}
\DeclareMathOperator{\Col}{Col}
\DeclareMathOperator{\Null}{N}
\newcommand{\tr}[1]{tr\left( #1 \right)}
\DeclareMathOperator{\matref}{ref}
\DeclareMathOperator{\matrref}{rref}
\DeclareMathOperator{\sol}{Sol}
\newcommand{\inp}[2]{\left< #1, #2 \right>}
\newcommand{\norm}[1]{\left\lVert #1 \right\rVert}

%% Statistics
\newcommand{\prob}[1]{P\left( #1 \right)}
\newcommand{\overbar}[1]{\mkern 1.5mu \overline {\mkern-1.5mu#1 \mkern-1.5mu} \mkern 1.5mu}


\renewcommand{\frame}[1]{\tilde{\underline{\vec{#1}}}}

\chead{Linear Independence}

\begin{document}

The concept of linear independence is crucial to linear algebra.

\section*{Linear Independence}
\begin{definition}
An indexed set of vectors $\set{\vec{v}_1, \dots, \vec{v}_n}$ in $\mathbb{R}^n$ is \textbf{linearly independent} if the vector equation,
\begin{equation*}
    a_1 \vec{v}_1 + \dots + a_n \vec{v}_n = \vec{0}
\end{equation*}
has only the trivial solution.
\end{definition}

\begin{definition}
Otherwise, $\set{\vec{v}_1, \dots, \vec{v}_n}$ is \textbf{linearly dependent}. In other words, $\set{\vec{v}_1, \dots, \vec{v}_n}$ is linearly dependent if there exists $a_1, \dots, a_n$, not all zero, such that
\begin{equation*}
    a_1 \vec{v}_1 + \dots + a_n \vec{v}_n = \vec{0}
\end{equation*}
\end{definition}

The ``zero" linear combination $a_1 \vec{v}_1 + \dots + a_n \vec{v}_n$ where $a_1 = \cdots = a_n = 0$ is called the \textbf{trivial} linear combination. Any other linear combination is called \textbf{non-trivial}.

\begin{example}
A set containing a single vector $\set{\vec{v}}$ is linearly independent if and only if $\vec{v}$ is non-zero. This is because $a\vec{v} = \vec{0}$ has only the trivial solution for $a$ if $\vec{v} \neq \vec{0}$. If $\vec{v} = 0$, then $a\vec{0} = \vec{0}$ has non-trivial solutions (any non-zero value for $a$).
\end{example}

\begin{example}
\textbf{Set with two vectors}. Consider a set with two vectors $\set{\vec{v}_1, \vec{v}_2}$, and consider
\begin{equation*}
    a \vec{v}_1 + b\vec{v}_2 = \vec{0}
\end{equation*}
If $\vec{v}_1, \vec{v}_2$ are linearly dependent, then at least one of $a, b$ is non-zero. If say $a \neq 0$, then by solving for $\vec{v}_1$, $\vec{v}_1 = -\frac{b}{a} \vec{v}_2$, i.e. $\vec{v}_1$ is a scalar multiple of $\vec{v}_2$. Conversely, if one vector is a scalar multiple of the other, say $\vec{v}_1 = k \vec{v}_2$, then $\vec{v}_1 - k \vec{v}_2 = \vec{0}$, which is a non-trivial linear dependence relation, so $\vec{v}_1, \vec{v}_2$ are linearly dependent.
\end{example}

In summary,

\begin{theorem}
A set of two vectors $\set{\vec{v}_1, \vec{v}_2}$ is linearly dependent if at least one vector is a scalar multiple of the other. Otherwise, if neither vector is a scalar multiple of the other, then the set is linearly independent.
\end{theorem}

\section*{Linear Dependence and Linear Independence}
\begin{definition}
Let $V$ be a vector space over $\mathbb{F}$, $S = \set{v_1, \dots, v_k} \subseteq V$ be a set of vectors. $S$ is \textbf{linearly dependent} (or the vectors in $S$ are linearly dependent) if there exists $a_1, \dots, a_k \in \mathbb{F}$, not all zero, such that
\begin{equation*}
    a_1 v_1 + \dots + a_k v_k = 0
\end{equation*}
\begin{itemize}
    \item For $a_1 = \dots = a_k = 0$, the ``zero" linear combination $a_1 v_1 + \dots + a_k v_k = 0$ is called the \textbf{trivial} linear combination. Any other linear combination is called \textbf{non-trivial}.
\end{itemize}
\end{definition}

\begin{theorem}
$S$ is linearly dependent if and only if there exists a vector $v_i \in S$ that can be written as a linear combination of the other vectors in $S$. In other words,
\begin{equation*}
    v_i \in \Span{v_1, \dots, v_{i-1}, v_{i+1}, \dots, v_k}
\end{equation*}
\end{theorem}

\begin{definition}
$S$ is \textbf{linearly independent} (or the vectors in $S$ are linearly independent) if $S$ is not linearly dependent. In other words, the equation
\begin{equation*}
    a_1 v_1 + \dots + a_k v_k = 0
\end{equation*}
only has solution $a_1 = \dots = a_k = 0$. In other words, there does not exist non-trivial linear combinations of vectors in $S$.
\end{definition}

In other words, the only representations of $0$ as linear combinations of the vectors of $S$ are trivial.

\begin{example}
By convention, $\emptyset$ is linearly independent. Also, a set consisting of a single non-zero vector is linearly independent.
\end{example}

\section*{Linearly Dependence Condition}
\begin{theorem}
A set $S = \set{v_1, \dots, v_n}$, where $n \geq 2$, is linearly dependent if and only if at least one vector $v_i$ can be written as a linear combination of the other vectors in $S$. In other words,
\begin{equation*}
    v_i \in \Span{v_1, \dots, v_{i-1}, v_{i+1}, \dots, v_k}
\end{equation*}
for some $i$. Equivalently, $S$ is linearly independent if no vector $v_i$ can be written as a linear combination of the others.
\end{theorem}

\begin{proof}
Let $\set{v_1, \dots, v_k}$ be linearly dependent. Then, there exists $a_1, \dots, a_k \in \mathbb{F}$, not all zero, such that,
\begin{equation*}
    a_1 v_1 + \dots + a_k v_k = 0
\end{equation*}
Then, for some $i$ with $a_i \neq 0$, solve for the corresponding vector $v_i$,
\begin{equation*}
    v_i = \frac{1}{a_i}\brac{-a_1 v_1 - \dots - a_{i-1} v_{i-1} - a_{i+1} v_{i+1} - \dots - a_k v_k}
\end{equation*}
Thus, $v_i$ can be writen as a linear combination of $\set{v_1, \dots, v_{i-1}, v_{i+1}, \dots, v_k}$, or
\begin{equation*}
    v_i \in \Span{v_1, \dots, v_{i-1}, v_{i+1}, \dots, v_k}
\end{equation*}
Conversely, let $v_i \in \Span{v_1, \dots, v_{i-1}, v_{i+1}, \dots, v_k}$ be written as a linear combination,
\begin{equation*}
    v_i = a_1 v_1 + \dots + a_{i-1} v_{i-1} + a_{i+1} v_{i+1} + \dots + a_k v_k
\end{equation*}
where $a_1, \dots, a_k$ are not all zero. Then, rearranging,
\begin{equation*}
    a_1 v_1 + \dots + a_{i-1} v_{i-1} - v_i a_{i+1} v_{i+1} + \dots + a_k v_k = 0
\end{equation*}
which is a non-trivial linear combination of $\set{v_1, \dots, v_k}$ which sums to 0. Thus, $S$ is linearly dependent.
\end{proof}

The preceding proof can be notationally simplified somewhat by assuming, without loss of generality, that $v_1$ is the vector which has non-zero coefficient $a_1$ and that can be written as a linear combination of the others.

\begin{corollary}
Two vectors $v, u \in V$ are linearly dependent if and only if one vector is a scalar multiple of the other.
\end{corollary}

\begin{proof}

\end{proof}

\section*{Properties}
\begin{example}
If $0 \in \set{v_1, \dots, v_k}$, then $\set{v_1, \dots, v_k}$ is linearly dependent.
\begin{proof}
If say $v_i = 0$, then,
\begin{equation*}
    0 v_1 + \dots + 0v_{i-1} + 1 v_i + 0 v_{i+1} + \dots + 0 v_k = 1v_i = 0
\end{equation*}
which is a non-trivial linear combination which sums to 0.
\end{proof}
\end{example}

\section*{Determining Linear Independence/Dependence}

Let $S = \set{\vec{v}_1, \dots, \vec{v}_k}$ be a set of vectors in $V$. To determine whether $S$ is linearly independent,
\begin{enumerate}[(a)]
    \item Form the vector equation $a_1 \vec{v}_1 + \dots + a_k \vec{v}_k = \vec{0}$, form a system of equations in $a_1, \dots, a_k$,
    \begin{equation*}
        \begin{bmatrix} \mid & \dots & \mid \\
        \vec{v}_1 & \dots & \vec{v}_k \\
        \mid & \dots & \mid \end{bmatrix} \begin{bmatrix} a_1 \\ \vdots \\ a_k \end{bmatrix} = \begin{bmatrix} 0 \\ \vdots \\ 0 \end{bmatrix}
    \end{equation*}
    This is a homogeneous system of equations $A\vec{x} = \vec{0}$, where the columns of $A$ are the vectors $\vec{v}_1, \dots, \vec{v}_k$.
    \item If the system has only the trivial solution $a_1 = \dots = a_k = 0$, then the $S$ is linearly independent. If the system has non-trivial solutions, then $S$ is linearly independent.
    \\ \\ In other words, if $\det{A} \neq 0$, then the system is linearly independent, and if $\det{A} = 0$, then the system is linearly dependent.
\end{enumerate}

In summary, a set of vectors $S = \set{\vec{v}_1, \dots, \vec{v}_k}$ is linearly independent if and only if
\begin{equation*}
    \det{\begin{bmatrix} \mid & \dots & \mid \\
    \vec{v}_1 & \dots & \vec{v}_k \\
    \mid & \dots & \mid \end{bmatrix}} \neq 0
\end{equation*}


















\section*{Linear Dependence/Independence of Subsets}
\begin{theorem}
Let $V$ be a vector space, $S_2 \subseteq S_1 \subseteq V$. If $S_2$ is linearly dependent, then $S_1$ is linearly dependent. Equivalently, if $S_1$ is linearly independent, then $S_2$ is linearly independent.
\begin{itemize}
    \item Intuitively, removing vectors from a linearly independent set preserves linear independence, and adding vectors to an already linearly dependent set preserves linear dependence.
\end{itemize}
\end{theorem}

\begin{theorem}
Let $V$ be a vector space, $S \subseteq V$ be linearly independent, $v \in V \setminus S$. Then, $S \cup \set{v}$ if linearly dependent if and only if $v \in \Span{v}$
\begin{itemize}
    \item In other words, if $S$ is linearly dependent, we can remove vectors from $S$ until we get a linearly independent set, forming the subset $S' \subseteq S$, such that $\Span{S'} = \Span{S}$.
\end{itemize}
\end{theorem}

\end{document}