\documentclass[letterpaper,12pt]{article}
\newcommand{\myname}{Cameron Geisler}
\newcommand{\mynumber}{90856741}
\usepackage{amsmath, amsfonts, amssymb, amsthm}
\usepackage[paper=letterpaper,left=25mm,right=25mm,top=3cm,bottom=25mm]{geometry}
\usepackage{fancyhdr}
\usepackage{float}
\usepackage{siunitx}
\usepackage{caption}
\usepackage{graphicx}
\pagestyle{fancy}
\usepackage{tkz-euclide} \usetkzobj{all} %% figures
\usepackage{hyperref} %% for links
\usepackage{exsheets} %% for tasks
\usepackage{systeme} %% for linear systems
\graphicspath{{../images/}} %% graphics in images folder

\lhead{Math 223} \chead{} \rhead{\myname}
\lfoot{} \cfoot{Page \thepage} \rfoot{}
\renewcommand{\headrulewidth}{0.4pt}
\renewcommand{\footrulewidth}{0.4pt}

\setlength{\parindent}{0pt}
\usepackage{enumerate}
\theoremstyle{definition}
\newtheorem*{definition}{Definition}
\newtheorem*{theorem}{Theorem}
\newtheorem*{example}{Example}
\newtheorem*{corollary}{Corollary}
\newtheorem*{lemma}{Lemma}
\newtheorem*{result}{Result}

%% Math
\newcommand{\abs}[1]{\left\lvert #1 \right\rvert}
\newcommand{\set}[1]{\left\{ #1 \right\}}
\renewcommand{\neg}{\sim}
\newcommand{\brac}[1]{\left( #1 \right)}
\newcommand{\eval}[1]{\left. #1 \right|}
\renewcommand{\vec}[1]{\mathbf{#1}}
\newenvironment{amatrix}[1]{\left[\begin{array}{@{}*{#1}{c}|c@{}}}{\end{array}\right]} %% for augmented matrix

\newcommand{\vecii}[2]{\left< #1, #2 \right>}
\newcommand{\veciii}[3]{\left< #1, #2, #3 \right>}

%% Linear algebra
\DeclareMathOperator{\Ker}{Ker}
\DeclareMathOperator{\nullity}{nullity}
\DeclareMathOperator{\Image}{Im}
\newcommand{\Span}[1]{\text{Span}\left(#1 \right)}
\DeclareMathOperator{\rank}{rank}
\DeclareMathOperator{\colrk}{colrk}
\DeclareMathOperator{\rowrk}{rowrk}
\DeclareMathOperator{\Row}{Row}
\DeclareMathOperator{\Col}{Col}
\DeclareMathOperator{\Null}{N}
\newcommand{\tr}[1]{tr\left( #1 \right)}
\DeclareMathOperator{\matref}{ref}
\DeclareMathOperator{\matrref}{rref}
\DeclareMathOperator{\sol}{Sol}
\newcommand{\inp}[2]{\left< #1, #2 \right>}
\newcommand{\norm}[1]{\left\lVert #1 \right\rVert}

%% Statistics
\newcommand{\prob}[1]{P\left( #1 \right)}
\newcommand{\overbar}[1]{\mkern 1.5mu \overline {\mkern-1.5mu#1 \mkern-1.5mu} \mkern 1.5mu}


\renewcommand{\frame}[1]{\tilde{\underline{\vec{#1}}}}

\chead{Matrices and Linear Transformations}

\begin{document}

Recall that every matrix transformation $\vec{x} \mapsto A\vec{x}$
Every linear transformation $T$ from $\mathbb{R}^n$ to $\mathbb{R}^m$ can be written as a matrix transformation. In other words, every linear transformation $T: \mathbb{R}^n \rightarrow \mathbb{R}^m$ can be written in the form $T(\vec{x}) = A\vec{x}$ for some $m \times n$ matrix $A$. Intuitively, the matrix $A$ is the ``formula" which allows for concrete computation of the images $T(\vec{x})$.
\\ \\ First, consider a linear transformation in the plane, $T: \mathbb{R}^2 \rightarrow \mathbb{R}^2$. Notice that any vector $\vec{x} \in \mathbb{R}^2$, say
\begin{equation*}
    \vec{x} = \begin{bmatrix} x_1 \\ x_2 \end{bmatrix}
\end{equation*}
can be written in terms of the columns of the identity matrix $I_2$, which are $\vec{e}_1 = \begin{bmatrix} 1 \\ 0 \end{bmatrix}, \vec{e}_2 = \begin{bmatrix} 0 \\ 1 \end{bmatrix}$,
\begin{equation*}
    \vec{x} = \begin{bmatrix} x_1 \\ x_2 \end{bmatrix} = x_1 \begin{bmatrix} 1 \\ 0 \end{bmatrix} + x_2 \begin{bmatrix} 0 \\ 1 \end{bmatrix} = x_1 \vec{e}_1 + x_2 \vec{e}_2
\end{equation*}
Then, consider the image of $\vec{x}$,
\begin{align*}
    T(\vec{x}) = T(x_1 \vec{e}_1 + x_2 \vec{e}_2) & = x_1 T(\vec{e}_1) + x_2 T(\vec{e}_2)
\end{align*}
In this way, $T(\vec{x})$ is completely determined by $T(\vec{e}_1)$ and $T(\vec{e}_2)$. Further,
\begin{equation*}
    T(\vec{x}) = \begin{bmatrix} T(\vec{e}_1) & T(\vec{e}_2) \end{bmatrix} \begin{bmatrix} x_1 \\ x_2 \end{bmatrix}
\end{equation*}

This can be generalized for a linear transformation $T: \mathbb{R}^n \rightarrow \mathbb{R}^m$.

\begin{theorem}
\textbf{Matrix representation of a linear transformation}. Let $T: \mathbb{R}^n \rightarrow \mathbb{R}^m$ be a linear transformation. Then, there exists a unique matrix $A$ such that
\begin{equation*}
    T(\vec{x}) = A\vec{x} \qquad \text{for all $\vec{x} \in \mathbb{R}^n$}
\end{equation*}
In particular, $A$ is precisely the $m \times n$ matrix whose $j$th column is given by $T(\vec{e}_j)$, where $\vec{e}_j$ is the $j$th column of the $n \times n$ identity matrix, i.e.
\begin{equation*}
    A = \begin{bmatrix} T(\vec{e}_1) & \cdots & T(\vec{e}_n) \end{bmatrix}
\end{equation*}
The matrix $A$ is called the \textbf{standard matrix} for the linear transformation $T$.
\end{theorem}

\begin{proof}
For $\vec{x} \in \mathbb{R}^n$,
\begin{equation*}
    \vec{x} = x_1 \vec{e}_1 + \dots + x_n \vec{e}_n
\end{equation*}
Then,
\begin{align*}
    T(\vec{x}) = T(x_1 \vec{e}_1 + \dots + x_n \vec{e}_n) & = x_1 T(\vec{e}_1) + \dots + x_n T(\vec{e}_n) \\
    & = \begin{bmatrix} T(\vec{e}_1) & \cdots & T(\vec{e}_n) \end{bmatrix} \begin{bmatrix} x_1 \\ \vdots \\ x_n \end{bmatrix} \\
    & = A\vec{x}
\end{align*}
\end{proof}

Intuitively, the term \textit{linear transformation} focusses on the properties of the mapping, whereas \textit{matrix transformation} describes how the mapping is implemented.

\end{document}