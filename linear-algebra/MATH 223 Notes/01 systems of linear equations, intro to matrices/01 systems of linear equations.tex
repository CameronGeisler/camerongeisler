\documentclass[letterpaper,12pt]{article}
\newcommand{\myname}{Cameron Geisler}
\newcommand{\mynumber}{90856741}
\usepackage{amsmath, amsfonts, amssymb, amsthm}
\usepackage[paper=letterpaper,left=25mm,right=25mm,top=3cm,bottom=25mm]{geometry}
\usepackage{fancyhdr}
\usepackage{float}
\usepackage{siunitx}
\usepackage{caption}
\usepackage{graphicx}
\pagestyle{fancy}
\usepackage{tkz-euclide} \usetkzobj{all} %% figures
\usepackage{hyperref} %% for links
\usepackage{exsheets} %% for tasks
\usepackage{systeme} %% for linear systems
\graphicspath{{../images/}} %% graphics in images folder

\lhead{Math 223} \chead{} \rhead{\myname}
\lfoot{} \cfoot{Page \thepage} \rfoot{}
\renewcommand{\headrulewidth}{0.4pt}
\renewcommand{\footrulewidth}{0.4pt}

\setlength{\parindent}{0pt}
\usepackage{enumerate}
\theoremstyle{definition}
\newtheorem*{definition}{Definition}
\newtheorem*{theorem}{Theorem}
\newtheorem*{example}{Example}
\newtheorem*{corollary}{Corollary}
\newtheorem*{lemma}{Lemma}
\newtheorem*{result}{Result}

%% Math
\newcommand{\abs}[1]{\left\lvert #1 \right\rvert}
\newcommand{\set}[1]{\left\{ #1 \right\}}
\renewcommand{\neg}{\sim}
\newcommand{\brac}[1]{\left( #1 \right)}
\newcommand{\eval}[1]{\left. #1 \right|}
\renewcommand{\vec}[1]{\mathbf{#1}}
\newenvironment{amatrix}[1]{\left[\begin{array}{@{}*{#1}{c}|c@{}}}{\end{array}\right]} %% for augmented matrix

\newcommand{\vecii}[2]{\left< #1, #2 \right>}
\newcommand{\veciii}[3]{\left< #1, #2, #3 \right>}

%% Linear algebra
\DeclareMathOperator{\Ker}{Ker}
\DeclareMathOperator{\nullity}{nullity}
\DeclareMathOperator{\Image}{Im}
\newcommand{\Span}[1]{\text{Span}\left(#1 \right)}
\DeclareMathOperator{\rank}{rank}
\DeclareMathOperator{\colrk}{colrk}
\DeclareMathOperator{\rowrk}{rowrk}
\DeclareMathOperator{\Row}{Row}
\DeclareMathOperator{\Col}{Col}
\DeclareMathOperator{\Null}{N}
\newcommand{\tr}[1]{tr\left( #1 \right)}
\DeclareMathOperator{\matref}{ref}
\DeclareMathOperator{\matrref}{rref}
\DeclareMathOperator{\sol}{Sol}
\newcommand{\inp}[2]{\left< #1, #2 \right>}
\newcommand{\norm}[1]{\left\lVert #1 \right\rVert}

%% Statistics
\newcommand{\prob}[1]{P\left( #1 \right)}
\newcommand{\overbar}[1]{\mkern 1.5mu \overline {\mkern-1.5mu#1 \mkern-1.5mu} \mkern 1.5mu}


\renewcommand{\frame}[1]{\tilde{\underline{\vec{#1}}}}

\chead{Systems of Linear Equations}

\begin{document}

In high-school algebra, we consider systems of \textit{two} linear equations in \textit{two} unknowns. For example,
\begin{equation*}
    \systeme{x + 2y = 5, 2x + 3y = 8}
\end{equation*}
These systems can be solved graphically, using substitution, or using elimination.
\begin{equation*}
    \boxed{\text{Solving systems of linear equations is the foundational problem of linear algebra.}}
\end{equation*}
In linear algebra, we develop more advanced and general techniques to solve larger systems of equations.

\section*{Systems of Linear Equations}

Recall that in algebra, the degree of a term is the sum of the powers of the variables it contains. For example, $5x^2$ has degree 2, $4xyz$ has degree 3, and $x$ has degree 1. A linear equation, broadly, is an equation where every term has degree at most 1. Here, we define a general linear equation in $n$ variables.

\begin{definition}
A \textbf{linear equation} in the variables $x_1, \dots, x_n$ is an equation which can be written in the form,
\begin{equation*}
    a_1 x_1 + a_2 x_2 + \dots + a_n x_n = b
\end{equation*}
where $a_1, \dots, a_n, b \in \mathbb{R}$ are called the \textbf{coefficients} of the equation.
\end{definition}

Most considered examples will contain from 2-5 variables (i.e. $n = 2, \dots, 5$). Most applications of linear algebra have $n$ much larger, containing hundreds or thousands of variables. Notice that the variables are denoted by $x_1, \dots, x_n$ rather than the more familiar $x, y, z$, etc. because we want to conveniently represent a large number of equations. If there are only two or three equations, then $x, y$ and $z$ are sometimes used.

\begin{definition}
A \textbf{system of linear equations}, broadly, is a collection of one or more linear equations involving the same variables. More precisely, a system of $m$ linear equations in $n$ unknowns is of the form
\begin{align*}
    a_{11}x_1 + a_{12}x_2 + \dots + a_{1n}x_n & = b_1 \\
    a_{21}x_1 + a_{22}x_2 + \dots + a_{2n}x_n & = b_2 \\
    \vdots \qquad \qquad \qquad \qquad \qquad & \vdots \\
    a_{m1}x_1 + a_{m2}x_2 + \dots + a_{mn}x_n & = b_n
\end{align*}
where $a_{ij}, b_i \in \mathbb{R}$ are the \textbf{coefficients} of the system. Notice that the coefficients $a_{ij}$ use \textit{double subscript} notation. Here, $a_{ij}$ represents the coefficients of $x_j$ in the $i$th equation.
\begin{itemize}
    \item A system of linear equations is also sometimes called \textbf{simultaneous linear equations}, or simply a \textbf{linear system}, or sometimes simply a \textbf{system}.
    \item The variables $x_1, \dots, x_n$ are sometimes called \textbf{unknowns}.
    \item A \textbf{solution} of a linear system is an ordered tuple $(s_1, \dots, s_n)$ which, when substituted for $(x_1, \dots, x_n)$, respectively, makes each equation a true statement.
    \item The set of all possible solutions of a linear system is called its \textbf{solution set}.
    \item Two linear systems are \textbf{equivalent} if they share the same solution set. In other words, every solution of the first system is also a solution of the second, and vise versa.
\end{itemize}
\end{definition}



\section*{Systems of Linear Equations in Two Variables}
A system of two linear equations in two variables is of the form,
\begin{equation*}
    \begin{cases} a_{11} x_1 + a_{12} x_2 = b_1 \\ a_{21} x_1 + a_{22} x_2 = b_2 \end{cases}
\end{equation*}
These two linear equations are lines, say $L_1, L_2$, in the plane. An ordered pair $(s_1, s_2)$ is a solution to the system if and only if it satisfies both equations in the system, which occurs if and only if $(s_1, s_2)$ lies on both lines $L_1, L_2$. In other words,
\begin{equation*}
    \boxed{\text{The solutions to the linear system are precisely the intersection points of the lines}}
\end{equation*}
Two lines can
\begin{itemize}
    \item Intersect at a single point, and have a single solution.
    \item Be parallel, and never intersect, and have no solution
    \item Coincide, and ``intersect" at every point on the line, and have infinitely many solution.
\end{itemize}
In general, systems of linear equations can have
\begin{itemize}
    \item No solution.
    \item Exactly one solution.
    \item Infinitely many solutions.
\end{itemize}

\begin{definition}
A system of linear equations is \textbf{consistent} if it has at least one solution (i.e. exactly one or infinitely many). Otherwise, it is \textbf{inconsistent}, i.e. if it has no solution.
\end{definition}

\section*{Misc}

\begin{definition}
Let $A$ be a $m \times n$ matrix, $\vec{b} = \begin{bmatrix} b_1 \\ \vdots \\ b_m \end{bmatrix} \in \mathbb{R}^m$ be a column vector. Then, a \textbf{system of linear equations} for $(x_1, \dots, x_n)$ is of the form
\begin{align*}
    a_{11}x_1 + \dots + a_{1n}x_n & = b_1 \\
    \dots \qquad \qquad \dots \quad & \quad \dots \\
    a_{m1}x_1 + \dots + a_{mn}x_1 & = b_m
\end{align*}
where $x_1, \dots, x_n$ are the \textbf{unknowns} of the system.
\\ \\ Alternatively, the system can be represented as $Ax = b$.
\end{definition}

\section*{Solving Using Inverses}
Let $A\vec{x} = \vec{b}$ be a system of linear equations. If the coefficient matrix $A$ is invertible, then by multiplying both sides by $A^{-1}$, we can solve for $\vec{x}$,
\begin{align*}
    A^{-1}(A \vec{x}) & = A^{-1} \vec{b} \\
    \vec{x} & = A^{-1} \vec{b}
\end{align*}
This provides an explicit expression for the solution $\vec{x}$. However, computing inverses of a matrix requires a lot of computation (WHY?), so this is not a useful method in practice.

\end{document}