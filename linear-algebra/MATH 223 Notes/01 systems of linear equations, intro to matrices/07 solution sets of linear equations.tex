\documentclass[letterpaper,12pt]{article}
\newcommand{\myname}{Cameron Geisler}
\newcommand{\mynumber}{90856741}
\usepackage{amsmath, amsfonts, amssymb, amsthm}
\usepackage[paper=letterpaper,left=25mm,right=25mm,top=3cm,bottom=25mm]{geometry}
\usepackage{fancyhdr}
\usepackage{float}
\usepackage{siunitx}
\usepackage{caption}
\usepackage{graphicx}
\pagestyle{fancy}
\usepackage{tkz-euclide} \usetkzobj{all} %% figures
\usepackage{hyperref} %% for links
\usepackage{exsheets} %% for tasks
\usepackage{systeme} %% for linear systems
\graphicspath{{../images/}} %% graphics in images folder

\lhead{Math 223} \chead{} \rhead{\myname}
\lfoot{} \cfoot{Page \thepage} \rfoot{}
\renewcommand{\headrulewidth}{0.4pt}
\renewcommand{\footrulewidth}{0.4pt}

\setlength{\parindent}{0pt}
\usepackage{enumerate}
\theoremstyle{definition}
\newtheorem*{definition}{Definition}
\newtheorem*{theorem}{Theorem}
\newtheorem*{example}{Example}
\newtheorem*{corollary}{Corollary}
\newtheorem*{lemma}{Lemma}
\newtheorem*{result}{Result}

%% Math
\newcommand{\abs}[1]{\left\lvert #1 \right\rvert}
\newcommand{\set}[1]{\left\{ #1 \right\}}
\renewcommand{\neg}{\sim}
\newcommand{\brac}[1]{\left( #1 \right)}
\newcommand{\eval}[1]{\left. #1 \right|}
\renewcommand{\vec}[1]{\mathbf{#1}}
\newenvironment{amatrix}[1]{\left[\begin{array}{@{}*{#1}{c}|c@{}}}{\end{array}\right]} %% for augmented matrix

\newcommand{\vecii}[2]{\left< #1, #2 \right>}
\newcommand{\veciii}[3]{\left< #1, #2, #3 \right>}

%% Linear algebra
\DeclareMathOperator{\Ker}{Ker}
\DeclareMathOperator{\nullity}{nullity}
\DeclareMathOperator{\Image}{Im}
\newcommand{\Span}[1]{\text{Span}\left(#1 \right)}
\DeclareMathOperator{\rank}{rank}
\DeclareMathOperator{\colrk}{colrk}
\DeclareMathOperator{\rowrk}{rowrk}
\DeclareMathOperator{\Row}{Row}
\DeclareMathOperator{\Col}{Col}
\DeclareMathOperator{\Null}{N}
\newcommand{\tr}[1]{tr\left( #1 \right)}
\DeclareMathOperator{\matref}{ref}
\DeclareMathOperator{\matrref}{rref}
\DeclareMathOperator{\sol}{Sol}
\newcommand{\inp}[2]{\left< #1, #2 \right>}
\newcommand{\norm}[1]{\left\lVert #1 \right\rVert}

%% Statistics
\newcommand{\prob}[1]{P\left( #1 \right)}
\newcommand{\overbar}[1]{\mkern 1.5mu \overline {\mkern-1.5mu#1 \mkern-1.5mu} \mkern 1.5mu}


\renewcommand{\frame}[1]{\tilde{\underline{\vec{#1}}}}

\chead{Solution Sets of Linear Equations}

\begin{document}

\section*{Homogeneous Systems}
\begin{definition}
A system of linear equations is \textbf{homogeneous} if it can be written in the form $A\vec{x} = \vec{0}$, where $A$ is an $m \times n$ matrix and $\vec{0}$ is the zero vector in $\mathbb{R}^m$.
\begin{itemize}
    \item Otherwise, a system $A\vec{x} = \vec{b}$ is said to be \textbf{non-homogeneous}.
\end{itemize}
\end{definition}

Homogeneous systems always have at least one solution, in particular the solution $\vec{x} = \vec{0}$, often called the \textbf{trivial} solution. Then, the important question is whether the equation $A\vec{x} = \vec{0}$ has any \textbf{non-trivial} solutions, that is, solutions $\vec{x}$ which are not the zero vector.
\\ \\ Recall that if a system is consistent, then it has a unique solution if the REF of its augmented matrix has no free variables, and infinitely many solutions if there is at least one free variable. Then, for homogeneous systems,

\begin{theorem}
The homogeneous equation $A\vec{x} = \vec{0}$ has a non-trivial solution if and only if the equation has at least one free variable.
\end{theorem}

The solution set of a homogeneous equation can always be written as the span of some collection of vectors.
\begin{itemize}
    \item If the system has no non-trivial solutions, then the solution set is $\Span{\vec{0}} = \vec{0}$.
\end{itemize}

\section*{Nonhomogeneous Systems}
\begin{theorem}
Let $A\vec{x} = \vec{b}$ be a linear system, which is consistent with solution $\vec{v}_p$. Then, the solution set of $A\vec{x} = \vec{b}$ is the set of all vectors of the form $\vec{v} = \vec{v}_p + \vec{v}_h$, where $\vec{v}_h$ is any solution of the associated homogeneous system $A\vec{x} = \vec{0}$.
\end{theorem}



\section*{Solving a Consistent System in Parametric Vector Form}
\begin{enumerate}
    \item Use row reduction to convert the augmented matrix to RREF.
    \item Solve for each basic variable in terms of any free variables.
    \item Write a typical solution $\vec{x}$ as a vector which depends on any free variables.
    \item Decompose $\vec{x}$ into a linear combination of vectors with only numerical entries, and use free variables as parameters.
\end{enumerate}










\end{document}