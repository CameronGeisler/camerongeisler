\documentclass[letterpaper,12pt]{article}
\newcommand{\myname}{Cameron Geisler}
\newcommand{\mynumber}{90856741}
\usepackage{amsmath, amsfonts, amssymb, amsthm}
\usepackage[paper=letterpaper,left=25mm,right=25mm,top=3cm,bottom=25mm]{geometry}
\usepackage{fancyhdr}
\usepackage{float}
\usepackage{siunitx}
\usepackage{caption}
\usepackage{graphicx}
\pagestyle{fancy}
\usepackage{tkz-euclide} \usetkzobj{all} %% figures
\usepackage{exsheets} %% for tasks

\lhead{Math 223} \chead{} \rhead{\myname \\ \mynumber}
\lfoot{\myname} \cfoot{Page \thepage} \rfoot{\mynumber}
\renewcommand{\headrulewidth}{0.4pt}
\renewcommand{\footrulewidth}{0.4pt}
\renewcommand\labelitemii{\textbullet} %changes 2nd level bullet to bullet

\setlength{\parindent}{0pt}
\usepackage{enumerate}
\theoremstyle{definition}
\newtheorem*{definition}{Definition}
\newtheorem*{theorem}{Theorem}
\newtheorem*{example}{Example}
\newtheorem*{corollary}{Corollary}
\newtheorem*{lemma}{Lemma}
\newtheorem*{result}{Result}

%% Math
\newcommand{\abs}[1]{\left\lvert #1 \right\rvert}
\newcommand{\set}[1]{\left\{ #1 \right\}}
\renewcommand{\neg}{\sim}
\newcommand{\brac}[1]{\left( #1 \right)}
\newcommand{\eval}[1]{\left. #1 \right|}

%% Vectors
\renewcommand{\vec}[1]{\mathbf{#1}}

%% Linear algebra
\DeclareMathOperator{\Ker}{Ker}
\DeclareMathOperator{\nullity}{nullity}
\DeclareMathOperator{\Image}{Im}
\newcommand{\Span}[1]{\text{Span}\left(#1 \right)}
\DeclareMathOperator{\rank}{rank}
\DeclareMathOperator{\colrk}{colrk}
\DeclareMathOperator{\rowrk}{rowrk}
\DeclareMathOperator{\Row}{Row}
\DeclareMathOperator{\Col}{Col}
\DeclareMathOperator{\Null}{N}
\newcommand{\tr}[1]{tr\left( #1 \right)}
\DeclareMathOperator{\matref}{ref}
\DeclareMathOperator{\matrref}{rref}
\DeclareMathOperator{\sol}{Sol}
\newcommand{\inp}[2]{\left< #1, #2 \right>}
\newcommand{\norm}[1]{\left\lVert #1 \right\rVert}

%% Statistics
\newcommand{\prob}[1]{P\left( #1 \right)}
\newcommand{\overbar}[1]{\mkern 1.5mu \overline {\mkern-1.5mu#1 \mkern-1.5mu} \mkern 1.5mu}

\begin{document}

\begin{itemize}
    \item pg. 1 Leontief economic analysis, ch.1.6 and 2.6, applications of linear algebra (in particular, linear systems of equations), 
    \item Oil exploration, ships searching for offshore oil deposits solve thousands of systems of linear equations every day. The seismic data for the equations are obtained from underwater shock waves created by explosions from air guns. The waves bounce off subsurface rocks and are measured by geophones attached to mile-long cables behind the ship. 
    \item Linear programming, airlines scheduling flight crews, monitoring the location of aircraft.
    \item Network flow (e.g. electrical networks, traffic flow, etc.), relies on linear algebra and linear systems of equations.
    \item Applications to physics, engineering, probability and statistics, economics, and biology.
    \item Geometry of a methane molecule.
    \item Factor analysis in psychology.
    \item Chemistry, balancing chemical equations as a system of equations.
\end{itemize}


Linear algebra is just as useful and applicable as calculus, and arguably more applicable.

\end{document}