\documentclass[letterpaper,12pt]{article}
\newcommand{\myname}{Cameron Geisler}
\newcommand{\mynumber}{90856741}
\usepackage{amsmath, amsfonts, amssymb}
\usepackage[paper=letterpaper,left=25mm,right=25mm,top=3cm,bottom=25mm]{geometry}
\usepackage{fancyhdr}
\usepackage{amsthm}
\usepackage{siunitx}
\pagestyle{fancy}

\lhead{Math 223}
\chead{Groups}
\rhead{\myname \\ \mynumber}
\lfoot{\myname}
\cfoot{Page \thepage}
\rfoot{\mynumber}
\renewcommand{\headrulewidth}{0.4pt}
\renewcommand{\footrulewidth}{0.4pt}
\renewcommand\labelitemii{\textbullet} %changes 2nd level bullet to bullet

\setlength{\parindent}{0pt}
\theoremstyle{definition}
\newtheorem*{result}{Result}
\newtheorem*{definition}{Definition}
\newtheorem*{theorem}{Theorem}
\newtheorem*{example}{Example}
\newtheorem*{corollary}{Corollary}
\newtheorem*{lemma}{Lemma}
\usepackage{enumerate}
\newcommand{\ihat}{\hat{\imath}}
\newcommand{\jhat}{\hat{\jmath}}
\newcommand{\set}[1]{\left\{ #1 \right\}}
\renewcommand{\vec}[1]{\overrightarrow{#1}} %vector
\newcommand{\abs}[1]{\left\lvert #1 \right\rvert} %absolute value / magnitude of vector
\renewcommand{\neg}{\sim}

%% Linear algebra
\DeclareMathOperator{\Ker}{Ker}
\DeclareMathOperator{\Image}{Im}
\DeclareMathOperator{\Span}{Span}
\DeclareMathOperator{\rk}{rk}
\DeclareMathOperator{\matref}{ref}
\DeclareMathOperator{\matrref}{rref}
\DeclareMathOperator{\sol}{Sol}
\newcommand{\inp}[2]{\left< #1, #2 \right>}
\newcommand{\norm}[1]{\| #1 \|}

\begin{document}

\section*{Groups}
\begin{definition}
A \textbf{group} is a pair $(G, \cdot)$, where $G$ is a set, and $\cdot$ is the group operation
\begin{align*}
    \cdot : G \times G & \longrightarrow G \\
    (a,b) & \longmapsto a \cdot b = ab
\end{align*}
with the following axioms:
\begin{enumerate}
    \item \textbf{Closure under the group operation}, $\forall a$, $b \in G$, $a \cdot b \in G$. This is superfluous, but important if $\cdot$ is taken from a larger set having $G$ as a subset.
    \item \textbf{Associative property}, $\forall a$, $b$, $c \in G$, $(ab)c = a(bc)$
    \item \textbf{Existence of a "neutral element"}, $\exists e \in G$ such that $\forall a \in G$ such that $ge = eg = g$
    \begin{itemize}
        \item The neutral element is typically denoted by $1$
    \end{itemize}
    \begin{corollary}
    The neutral element $e$ is unique.
    \end{corollary}
    \begin{proof}
    Let $e$, $e' \in G$ such that $\forall a \in G$, $ae = ea = a$ and $ae' = e'a = a$. Then, for $a = e$, we have $ee' = e$, and for $a = e'$, we have $e'e = ee' = e'$. Thus, $e = e'$.
    \end{proof}
    \item \textbf{Existence of inverses}, $\forall a \in G$, $\exists a^{-1} \in G$ such that $aa^{-1} = a^{-1}a = e$
    \begin{corollary}
    The inverse element $a_{-1}$ is unique.
    \end{corollary}
    \begin{proof}
    Let $a \in G$ with inverses $b = c = a^{-1}$, so that $ab = ba = e$ and $ac = ca = e$. Then, $c = ce = c(ab) = (ca)b = eb = b$, and so $c = b$.
    \end{proof}
\end{enumerate}
\end{definition}

\section*{Abelian Groups}
\begin{definition}
An \textbf{abelian group} is a group that satisfies the commutative property. In other words, $\forall a$, $b \in G$, $ab = ba$
\begin{itemize}
    \item For abelian groups, typically the map $\cdot$ is notated as $+$, the neutral element by $0$, and the inverse element of $a$ by $-a$.
\end{itemize}
\end{definition}

\begin{example}
Addition over the real numbers or integers, $(\mathbb{Z}, +)$ or $(\mathbb{R}, +)$, is an abelian group.
\end{example}

\begin{example}
Multiplication over the real numbers excluding zero, $(\mathbb{R} \setminus \set{0}, \cdot)$, is an abelian group.
\end{example}

\begin{example}
Let $(\mathbb{F}, +, \cdot)$ be a field. Then, $(\mathbb{F}, +)$ and $(\mathbb{F} \setminus \set{0}, \cdot)$ are abelian groups.
\end{example}

\begin{example}
Let $(V, +, \cdot)$ be a vector space over $\mathbb{F}$. Then, $(V, +)$ is an abelian group.
\end{example}

\begin{definition}
Let $(G, \cdot)$ be a group, $H \subseteq G$ be a subset. Then, $H$ is a \textbf{subgroup} of $G$ if
\begin{enumerate}
    \item \textbf{$H$ is closed under the group operation in $G$}. For all $a, b \in H$, $a \cdot b \in H$.
    \item \textbf{$H$ includes the identity element from $G$}. In other words, $e \in H$.
    \item \textbf{Closed under inverses}. For all $a \in H$, $a^{-1} \in H$.
\end{enumerate}
\end{definition}

\begin{corollary}
If $H$ is a subgroup of the group $G$, then $H$ with the restriction of $\cdot: H \times H \rightarrow H$  as the group operation, is a group.
\end{corollary}

\begin{example}
Let $X$ be a set, $\operatorname{Bij}(X)$ be the set of bijective maps $f: X \rightarrow X$. Then, $(\operatorname{Bij}(X), \circ)$, where $\circ$ is function composition, is an group.
\begin{align*}
    \circ: \operatorname{Bij}(X) \times \operatorname{Bij}(X) & \longrightarrow \operatorname{Bij}(X) \\
    (f,g) & \longmapsto f \circ g
\end{align*}
\begin{enumerate}
    \item The composition of two bijective functions is bijective.
    \item Let $f$, $g$, $h \in \operatorname{Bij}(X)$, $x \in X$. Then,
    \begin{align*}
        ((f \circ g) \circ h)(x) = (f \circ g)(h(x)) = f(g(h(x))) = f((g \circ h)(x)) = (f \circ (g \circ h))(x)
    \end{align*}
    \item The identity element is $Id_{X}$
    \item $\forall f \in \operatorname{Bij}(X)$, the inverse is the inverse function $f^{-1}$, which exists because $f$ is bijective.
\end{enumerate}
\end{example}

\begin{example}
Let $V$ be a vector space over $\mathbb{F}$. The set of all automorphisms $f: V \rightarrow V$, with function composition as the group operation, is a group.
\end{example}

\section*{Linear Groups over a Vector Space}
\begin{definition}
Let $V$ be a vector space over $\mathbb{F}$. Then, the \textbf{general linear group} of $V$, $\operatorname{GL}(V)$, is the set of all automorphisms, with function composition as the group operation. Note, $\operatorname{GL}(V)$ is a subgroup of $\operatorname{Bij}(V)$.
\begin{enumerate}
    \item The composition of linear maps is linear.
    \item Function composition is associative.
    \item The identity element is $Id_{V}: V \rightarrow V$.
    \item All linear maps $f$ are bijective, so $f^{-1}$ exists and is linear.
\end{enumerate}
\end{definition}

\begin{example}
Let $V$ be a vector space over $\mathbb{F}$. Then, the \textbf{special linear group} of $V$, $\operatorname{SL}(V)$, is
\begin{equation*}
    \operatorname{SL}(V) = \set{f \in \operatorname{GL}(V) : \det{f} = 1}
\end{equation*}
$SL(V) \subset GL(V)$ is a subgroup, as $\det{f \circ g} = \det{f} \cdot \det{g}$, $\det{I_n} = 1$, and if $\det{f} = 1$ then $\det{f^{-1}}$.
\end{example}

\section*{Linear Groups of Matrices}
\begin{definition}
A \textbf{general linear group} of degree $n$ over $\mathbb{F}$, $\operatorname{GL}(n, \mathbb{F})$ (or $\operatorname{GL}(\mathbb{F}^n)$) is the set of all invertible $n \times n$ matrices over $\mathbb{F}$, with matrix multiplication as the group operation.
\end{definition}

\begin{definition}
The \textbf{special linear group} of degree $n$ over $\mathbb{F}$, $\operatorname{SL}(n, \mathbb{F})$, is
\begin{equation*}
    \operatorname{SL(n, \mathbb{F}}) = \set{A \in \operatorname{GL}(n, \mathbb{F}) : \det{A} = 1}
\end{equation*}
$\operatorname{SL}(n, \mathbb{F})$ is a subgroup of $\operatorname{GL}(n, \mathbb{F})$.
\end{definition}

\section*{Coordinates}
Let $V$ be a finite-dimensional vector space over $\mathbb{F}$, $B = (v_1, \dots, v_n)$ be a basis of $V$. Let $v \in V$, then $v$ can be written as
\begin{align*}
    v = x_1v_1 + \dots x_nv_n
\end{align*}
Then, the \textbf{coordinate vector} of $v$ with respect to $B$ is
\begin{align*}
    [v]_{B} = \begin{pmatrix} x_1 \\ \dots \\ x_n \end{pmatrix} \in \mathbb{F}^n
\end{align*}

We want to convert between coordinates with respect to the standard basis, $[v]_{\mathcal{C}}$, to coordinates in another basis, $[v]_{B}$.
\\ Recall the general formula, if $\mathcal{C} = (w_1, \dots, w_n)$ is another basis of $V$, $[v]_{\mathcal{C}} = \begin{pmatrix} x_1' \\ \dots \\ x_n' \end{pmatrix}$, then
\begin{align*}
    [f(v)]_{\mathcal{C}} = [f]_{\mathcal{C}}^B [v]_{B}
\end{align*}
Then,
\begin{align*}
    [v]_{\mathcal{C}} & = [id]_{\mathcal{C}}^B [v]_{B} \\
    \begin{pmatrix} x_1' \\ \dots \\ x_n' \end{pmatrix} & = ([v_1]_{\mathcal{C}}, \dots, [v_n]_{\mathcal{C}}) \begin{pmatrix} x_1 \\ \dots \\ x_n \end{pmatrix}
\end{align*}
If we apply this to $\mathbb{F}^n$, we have $\mathcal{C} = (e_1, \dots, e_n)$ is the standard basis, $B = (v_1, \dots, v_n)$ is a different basis,
\begin{align*}
    [v]_{\mathcal{C}} & = \begin{pmatrix} x_1 \\ \dots \\ x_n \end{pmatrix} \\
    [v]_{B} & = \begin{pmatrix} x_1' \\ \dots \\ x_n' \end{pmatrix}
\end{align*}
Thus,
\begin{align*}
    \boxed{[v]_{\mathcal{C}} = [id]_{\mathcal{C}}^B [v]_{B}}
\end{align*}
$[id]_{\mathcal{C}}^B = (v_1, \dots, v_n)$ is the \textbf{change of basis matrix}.

\section*{Change of Basis matrices}


\end{document}




