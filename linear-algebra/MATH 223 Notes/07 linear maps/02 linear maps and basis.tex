\documentclass[letterpaper,12pt]{article}
\newcommand{\myname}{Cameron Geisler}
\newcommand{\mynumber}{90856741}
\usepackage{amsmath, amsfonts, amssymb, amsthm}
\usepackage[paper=letterpaper,left=25mm,right=25mm,top=3cm,bottom=25mm]{geometry}
\usepackage{fancyhdr}
\usepackage{float}
\usepackage{siunitx}
\usepackage{caption}
\usepackage{graphicx}
\pagestyle{fancy}
\usepackage{tkz-euclide} \usetkzobj{all} %% figures
\usepackage{hyperref} %% for links
\usepackage{exsheets} %% for tasks
\usepackage{systeme} %% for linear systems
\graphicspath{{../images/}} %% graphics in images folder

\lhead{Math 223} \chead{} \rhead{\myname}
\lfoot{} \cfoot{Page \thepage} \rfoot{}
\renewcommand{\headrulewidth}{0.4pt}
\renewcommand{\footrulewidth}{0.4pt}

\setlength{\parindent}{0pt}
\usepackage{enumerate}
\theoremstyle{definition}
\newtheorem*{definition}{Definition}
\newtheorem*{theorem}{Theorem}
\newtheorem*{example}{Example}
\newtheorem*{corollary}{Corollary}
\newtheorem*{lemma}{Lemma}
\newtheorem*{result}{Result}

%% Math
\newcommand{\abs}[1]{\left\lvert #1 \right\rvert}
\newcommand{\set}[1]{\left\{ #1 \right\}}
\renewcommand{\neg}{\sim}
\newcommand{\brac}[1]{\left( #1 \right)}
\newcommand{\eval}[1]{\left. #1 \right|}
\renewcommand{\vec}[1]{\mathbf{#1}}
\newenvironment{amatrix}[1]{\left[\begin{array}{@{}*{#1}{c}|c@{}}}{\end{array}\right]} %% for augmented matrix

\newcommand{\vecii}[2]{\left< #1, #2 \right>}
\newcommand{\veciii}[3]{\left< #1, #2, #3 \right>}

%% Linear algebra
\DeclareMathOperator{\Ker}{Ker}
\DeclareMathOperator{\nullity}{nullity}
\DeclareMathOperator{\Image}{Im}
\newcommand{\Span}[1]{\text{Span}\left(#1 \right)}
\DeclareMathOperator{\rank}{rank}
\DeclareMathOperator{\colrk}{colrk}
\DeclareMathOperator{\rowrk}{rowrk}
\DeclareMathOperator{\Row}{Row}
\DeclareMathOperator{\Col}{Col}
\DeclareMathOperator{\Null}{N}
\newcommand{\tr}[1]{tr\left( #1 \right)}
\DeclareMathOperator{\matref}{ref}
\DeclareMathOperator{\matrref}{rref}
\DeclareMathOperator{\sol}{Sol}
\newcommand{\inp}[2]{\left< #1, #2 \right>}
\newcommand{\norm}[1]{\left\lVert #1 \right\rVert}

%% Statistics
\newcommand{\prob}[1]{P\left( #1 \right)}
\newcommand{\overbar}[1]{\mkern 1.5mu \overline {\mkern-1.5mu#1 \mkern-1.5mu} \mkern 1.5mu}


\renewcommand{\frame}[1]{\tilde{\underline{\vec{#1}}}}

\chead{Linear Maps and Basis}

\begin{document}

\section*{Images of Basis Vectors is Basis of Image}
\begin{theorem}
Let $f: V \rightarrow W$ be a linear map, $(v_1, \dots, v_n)$ be a basis of $V$. Then, $(f(v_1), \dots, f(v_n))$ is a basis of $\Image{f}$
\end{theorem}
\begin{proof}
Let $w \in \Image{f}$. Then, $\exists v \in V$ such that $w = f(v)$. Since $(v_1, \dots, v_n)$ is a basis of $V$, there exists a unique $a_1, \dots, a_n$ such that $v = \sum_{i=1}^n a_i v_i$
\\ \\ Then,
\begin{align*}
    w & = f(v) \\
    & = f\left(\sum_{i=1}^n a_i v_i \right) \\
    & = \sum_{i=1}^n a_i f(v_i) && \text{as $f$ is linear}
\end{align*}
Thus, $w$ can be written as linear combination of $f(v_1), \dots, f(v_n)$, so $(f(v_1), \dots, f(v_n))$ is a basis of $\Image{f}$.
\end{proof}

\section*{Isomorphism if and only if Images of Basis is Basis of Codomain}
\begin{theorem}
Let $f: V \rightarrow W$ be a linear map, $(v_1, \dots, v_n)$ be a basis of $V$. $f$ is an isomorphism if and only if $(f(v_1), \dots, f(v_n))$ is a basis of $W$.
\end{theorem}
\begin{proof}
Let $f$ be an isomorphism, $\sum_{i=1}^n a_i f(v_i) = 0$. Then,
\begin{align*}
    0 & = \sum_{i=1}^n a_i f(v_i) \\
    & = f\left(\sum_{i=1}^n a_i v_i \right) && \text{as $f$ is linear}
\end{align*}
Since $f$ is injective, $\Ker{f} = \set{0}$, so
\begin{equation*}
    \sum_{i=1}^n a_i v_i = 0
\end{equation*}
Since $(v_1, \dots, v_n)$ is a basis, it is linearly independent, so $a_1 = \dots = a_n = 0$. Thus, $(f(v_1), \dots, f(v_n))$ is linearly independent.
\\ \\ Let $w \in W$. Then, since $f$ is surjective, $\exists v \in V$ such that $w = f(v)$. Since $(v_1, \dots, v_n)$ is a basis of $V$, there exists unique $a_1, \dots, a_n \in \mathbb{F}$ such that $v = \sum_{i=1}^n a_i v_i$
\\ \\ Then,
\begin{align*}
    w & = f(v) \\
    & = f\left(\sum_{i=1}^n a_i v_i \right) \\
    & = \sum_{i=1}^n a_i f(v_i) && \text{as $f$ is linear}
\end{align*}
Thus, $w$ can be written as a linear combination of $f(v_1), \dots, f(v_n)$, so $(f(v_1), \dots, f(v_n))$ is a basis of $W$.
\\ \\ Conversely, let $(f(v_1), \dots, f(v_n))$ be a basis of $W$.
\\ \\ To prove $f$ is injective, let $v \in V$ with $f(v) = 0$. Then, since $(v_1, \dots, v_n)$ is a basis of $V$, there exists unique $a_1, \dots, a_n \in \mathbb{F}$ such that $v = \sum_{i=1}^n a_i v_i$. Then,
\begin{align*}
    0 & = f(v) \\
    & = f\left(\sum_{i=1}^n a_i v_i \right) \\
    & = a_1 f(v_1) + \dots a_n f(v_n) && \text{as $f$ is linear}
\end{align*}
$(f(v_1), \dots, f(v_n))$ is linearly independent, so $a_1 = \dots = a_n = 0$. Thus, $v = 0$, so $\Ker{f} = \set{0}$, and so $f$ is injective.
\\ \\ To prove $f$ is surjective, let $w \in W$. $(f(v_1), \dots, f(v_n))$ is a basis for $W$, so there exists $a_1, \dots, a_n \in \mathbb{F}$ such that $w = \sum_{i=1}^n a_i f(v_i)$. Then, for $v = \sum_{i=1}^n a_i v_i \in V$, we have
\begin{align*}
    w & = \sum_{i=1}^n a_i f(v_i) \\
    & = f\left(\sum_{i=1}^n a_i v_i \right) \\
    & = f(v)
\end{align*}
\end{proof}

\begin{corollary}
Any two $n$-dimensional vector spaces over $\mathbb{F}$ are isomorphic, in other words, there exists an isomorphism $f: V \rightarrow W$
\end{corollary}

\section*{Linear Transformation Completely Determined by Image of Basis}
\begin{theorem}
Let $V$, $W$ be vector spaces over $\mathbb{F}$, $(v_1, \dots, v_n)$ be a basis of $V$, $w_1, \dots, w_n \in W$. Then, there exists a unique linear map $f: V \rightarrow W$ such that $f(v_i) = w_i$ for all $i = 1, \dots n$.
\end{theorem}
\begin{proof}
Since $(v_1, \dots, v_n)$ is a basis of $V$, for each $x \in V$, there exists unique $\lambda_1, \dots, \lambda_n \in \mathbb{F}$ such that
\begin{equation*}
    x = \sum_{i=1}^n \lambda_i v_i
\end{equation*}
Define $f: V \rightarrow W$, $f(x) = f\left(\sum_{i=1}^n \lambda_i v_i \right) = \sum_{i=1}^n \lambda_i w_i$
\\ \\ Let $x$, $y \in V$, $x = \sum_{i=1}^n \alpha_i v_i$, $y = \sum_{i=1}^n \beta_i v_i$
\\ \\ Then,
\begin{align*}
    f(x + y) = f\left(\sum_{i=1}^n \alpha_i v_i + \sum_{i=1}^n \beta_i v_i \right) = f\left(\sum_{i=1}^n (\alpha_i + \beta_i)v_i \right) & = \sum_{i=1}^n (\alpha_i + \beta_i) w_i && \text{by construction} \\
    & = \sum_{i=1}^n \alpha_i w_i + \sum_{i=1}^n \beta_i w_i \\
    & = f(x) + f(y)
\end{align*}
Let $\lambda \in \mathbb{F}$. Then,
\begin{align*}
    f(\lambda x) & = f\left(\lambda \sum_{i=1}^n \alpha_i v_i \right) = f\left(\sum_{i=1}^n \lambda \alpha_i v_i \right) = \lambda \sum_{i=1}^n \alpha_i w_i = \lambda f(x)
\end{align*}
\end{proof}

\begin{corollary}
A linear map is completely defined by the images of its basis vectors $(f(v_1), \dots, f(v_n))$, in other words, if $f: V \rightarrow W$, $g: V \rightarrow W$ are linear maps with $f(v_i) = g(v_i)$ for all $i = 1, \dots, n$, then $f = g$
\end{corollary}

\section*{Injective if and only if Image of Basis is Independent}
\begin{theorem}
Let $V$, $W$ be vector space over $\mathbb{F}$, $\set{v_1, \dots, v_n}$ be a basis of $V$, $f: V \rightarrow W$ be a linear map. Then, $f$ is injective if and only if $\set{f(v_1), \dots, f(v_n)}$ is linearly independent.
\end{theorem}
\begin{proof}
Let $f$ be injective, $a_1, \dots, a_n \in \mathbb{F}$ with $a_1 f(v_1) + \dots + a_n f(v_n) = 0$. Then,
\begin{align*}
    0 & = a_1 f(v_1) + \dots + a_n f(v_n) \\
    & = f(a_1 v_1 + \dots + a_n v_n) && \text{as $f$ is linear}
\end{align*}
$f$ is injective, so $\Ker{f} = \set{0}$, and so $a_1 v_1 + \dots + a_n v_n = 0$. Since $\set{v_1, \dots, v_n}$ is linearly independent, $a_1 = \dots = a_n = 0$, and so $\set{f(v_1), \dots, f(v_n)}$ is linearly independent.
\\ \\ Conversely, let $\set{v_1, \dots, v_n}$ be linearly independent, $v \in V$ with $f(v) = 0$. Since $\set{v_1, \dots, v_n}$ is a basis of $V$, $\exists a_1, \dots, a_n \in \mathbb{F}$ such that $v = a_1 v_1 + \dots + a_n v_n$. Then,
\begin{align*}
    0 & = f(v) \\
    & = f(a_1 v_1 + \dots + a_n v_n) \\
    & = a_1 f(v_1) + \dots + a_n f(v_n)
\end{align*}
Since $\set{f(v_1), \dots, f(v_n)}$ is linearly independent, $a_1 = \dots = a_n = 0$, and so $v = 0$. Thus, $\Ker{f} = \set{0}$, and so $f$ is injective.

\end{proof}

\section*{Invariant Subspace of a Linear Map}
\begin{definition}
Let $V$ be a vector space, $f: V \rightarrow V$ be an endomorphism, $U \subseteq V$ be a subspace. $U$ is \textbf{invariant} under $f$ if $f(U) \subseteq U$. In other words, for all $x \in U$, $f(x) \in U$.
\end{definition}

\section*{Restriction of a Linear Map}
\begin{definition}
Let $V$ be a vector space, $f: V \rightarrow V$ be an endomorphism, $U \subseteq V$ be a subspace that is invariant under $f$. Then, the \textbf{restriction} of $f$ to $U$ is the linear map $f_{U}: U \rightarrow U$, $f_{U}(x) = f(x)$ for all $x \in U$.
\end{definition}

\section*{Rank and Nullity}
Let $V$, $W$ be vector spaces, $f: V \rightarrow W$ be a linear map, with $\Image{f}$ and $\Ker{f}$ finite dimensional.
\begin{definition}
The \textbf{rank} of $f$, $\rank{f}$, is the dimension of $\Image{f}$
\begin{equation*}
    \rank{f} = \dim{(\Image{f})}
\end{equation*}
\end{definition}

\begin{definition}
The \textbf{nullity} of $f$, $\nullity{f}$, is the dimension of $\Ker{f}$
\begin{equation*}
    \nullity{f} = \dim{(\Ker{f})}
\end{equation*}
\end{definition}

\section*{Rank-nullity Theorem}
\begin{theorem}
Let $V$, $W$ be vector spaces, $f: V \rightarrow W$ be a linear map. Then,
\begin{equation*}
    \boxed{\rank{f} + \nullity{f} = \dim{V}}
\end{equation*}
\end{theorem}

\end{document}