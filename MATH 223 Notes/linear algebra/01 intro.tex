\documentclass[letterpaper,12pt]{article}
\newcommand{\myname}{Cameron Geisler}
\newcommand{\mynumber}{90856741}
\usepackage{amsmath, amsfonts, amssymb, amsthm}
\usepackage[paper=letterpaper,left=25mm,right=25mm,top=3cm,bottom=25mm]{geometry}
\usepackage{fancyhdr}
\usepackage{float}
\usepackage{siunitx}
\usepackage{caption}
\usepackage{graphicx}
\pagestyle{fancy}
\usepackage{tkz-euclide} \usetkzobj{all} %% figures
\usepackage{hyperref} %% for links
\usepackage{exsheets} %% for tasks
\usepackage{systeme} %% for linear systems
\graphicspath{{../images/}} %% graphics in images folder

\lhead{Math 223} \chead{} \rhead{\myname}
\lfoot{} \cfoot{Page \thepage} \rfoot{}
\renewcommand{\headrulewidth}{0.4pt}
\renewcommand{\footrulewidth}{0.4pt}

\setlength{\parindent}{0pt}
\usepackage{enumerate}
\theoremstyle{definition}
\newtheorem*{definition}{Definition}
\newtheorem*{theorem}{Theorem}
\newtheorem*{example}{Example}
\newtheorem*{corollary}{Corollary}
\newtheorem*{lemma}{Lemma}
\newtheorem*{result}{Result}

%% Math
\newcommand{\abs}[1]{\left\lvert #1 \right\rvert}
\newcommand{\set}[1]{\left\{ #1 \right\}}
\renewcommand{\neg}{\sim}
\newcommand{\brac}[1]{\left( #1 \right)}
\newcommand{\eval}[1]{\left. #1 \right|}
\renewcommand{\vec}[1]{\mathbf{#1}}
\newenvironment{amatrix}[1]{\left[\begin{array}{@{}*{#1}{c}|c@{}}}{\end{array}\right]} %% for augmented matrix

\newcommand{\vecii}[2]{\left< #1, #2 \right>}
\newcommand{\veciii}[3]{\left< #1, #2, #3 \right>}

%% Linear algebra
\DeclareMathOperator{\Ker}{Ker}
\DeclareMathOperator{\nullity}{nullity}
\DeclareMathOperator{\Image}{Im}
\newcommand{\Span}[1]{\text{Span}\left(#1 \right)}
\DeclareMathOperator{\rank}{rank}
\DeclareMathOperator{\colrk}{colrk}
\DeclareMathOperator{\rowrk}{rowrk}
\DeclareMathOperator{\Row}{Row}
\DeclareMathOperator{\Col}{Col}
\DeclareMathOperator{\Null}{N}
\newcommand{\tr}[1]{tr\left( #1 \right)}
\DeclareMathOperator{\matref}{ref}
\DeclareMathOperator{\matrref}{rref}
\DeclareMathOperator{\sol}{Sol}
\newcommand{\inp}[2]{\left< #1, #2 \right>}
\newcommand{\norm}[1]{\left\lVert #1 \right\rVert}

%% Statistics
\newcommand{\prob}[1]{P\left( #1 \right)}
\newcommand{\overbar}[1]{\mkern 1.5mu \overline {\mkern-1.5mu#1 \mkern-1.5mu} \mkern 1.5mu}


\renewcommand{\frame}[1]{\tilde{\underline{\vec{#1}}}}

\chead{}


\begin{document}

\section*{Row Space, Column Space}

\begin{definition}
Let $A$ be an $m \times n$ matrix. Then,
\begin{itemize}
    \item The \textbf{row space} of $A$, $\Row{A}$, is the subspace of $\mathbb{R}^n$ spanned by the row vectors of $A$.
    \item The \textbf{column space} of $A$, $\Col{A}$, is the subspace of $\mathbb{R}^m$ spanned by the column vectors of $A$.
\end{itemize}
\end{definition}

\begin{theorem}
If two $m \times n$ matrices $A$ and $B$ are row-equivalent, then their row spaces are equal, $\Row{A} = \Row{B}$.
\end{theorem}

Recall that elementary row operations can change the column space of a matrix.

\begin{theorem}
Let $A$ be an $m \times n$ matrix, $B$ be the row-equivalent matrix which is $A$ in row-echelon form. Then, the non-zero row vectors of $B$ form a basis for the row space of $A$.
\end{theorem}

\begin{theorem}
The row space and column space of a matrix $A$ have equal dimension.
\end{theorem}

\begin{definition}
The \textbf{rank} of a matrix $A$, $\rank{A}$, is the dimension of the row space (or column space) of $A$.
\end{definition}

\section*{Null Space, Solution to Homogeneous System of Equations}
\begin{theorem}
Let $A$ be an $m \times n$ matrix. Then, the set of solution to the homogeneous system of equations is $A \vec{x} = \vec{0}$ is a subspace of $\mathbb{R}^n$, called the \textbf{null space} of $A$, $N(A)$,
\begin{equation*}
    N(A) = \set{\vec{x} \in \mathbb{R}^n: A \vec{x} = \vec{0}}
\end{equation*}
The dimension of the null space $N(A)$ is called the \textbf{nullity} of $A$.
\end{theorem}

\begin{proof}
EXERCISE.
\end{proof}

\begin{theorem}
\textbf{Rank-nullity theorem}. Let $A$ be an $m \times n$ matrix, with rank $r$. Then, the nullity of $A$ is $n - r$. In other words,
\begin{equation*}
    \rank{A} + \nullity{A} = \dim{A}
\end{equation*}
\end{theorem}

\begin{theorem}
Let $A \vec{x} = \vec{b}$ be a non-homogeneous system of equations. If $\vec{x}_p$ is a particular solution of $A\vec{x} = \vec{b}$, then every solution of this system is of the form $\vec{x} = \vec{x}_p + \vec{x}_h$, where $\vec{x}_h$ is a solution of the corresponding homogeneous system $A\vec{x} = \vec{0}$.
\end{theorem}
\begin{proof}
EXERCISE.
\end{proof}

\begin{theorem}
The system $A\vec{x} = \vec{b}$ is consistent if and only if $\vec{b}$ is in the column space of $A$.
\end{theorem}


\section*{Summary of Equivalent Conditions}
\begin{theorem}
Let $A$ be an $n \times n$ matrix. Then, the following are equivalent:
\begin{enumerate}
    \item $A$ is invertible.
    \item $A\vec{x} = 0$ has only the trivial solution.
    \item $A\vec{x} = \vec{b}$ has a unique solution for every $\vec{b}$.
    \item $A$ is row equivalent to $I_n$.
    \item $\det{A} \neq 0$.
    \item \textbf{Full rank}, $\rank{A} = n$.
    \item The $n$ row vectors of $A$ are linearly independent.
    \item The $n$ column vectors of $A$ are linearly independent.
\end{enumerate}
\end{theorem}



\section*{Misc}

Let $A \in M(m \times n, \mathbb{F})$. Then we have the following definitions:

\begin{definition}
The \textbf{nullspace} of $A$, $N(A)$ is the kernel of the linear map associated with $A$.
\begin{itemize}
    \item $N(A) = \Ker{(A: \mathbb{F}^n \rightarrow \mathbb{F}^m)}$
\end{itemize}
\end{definition}

\begin{definition}
The \textbf{nullity} of $A$, $\nullity{A}$, is the dimension of the nullspace of $A$.
\begin{itemize}
    \item $\nullity{(A)} = \dim{(\Ker{(A: \mathbb{F}^n \rightarrow \mathbb{F}^m)})}$
\end{itemize}
\end{definition}

\begin{definition}
The \textbf{image} (or \textbf{column space}) of $A$, $\Image{A}$ or $\Col{A}$, is the vector space formed by the span of the column vectors of $A$.
\begin{itemize}
    \item $\Image{A} = \Span{(A_1, \dots, A_n)}$
    \item $\Span{(A_1, \dots, A_n)} = \Span{(Ae_1, \dots, Ae_n)}$, and since $(e_1, \dots, e_n)$ spans $\mathbb{F}^n$, $(Ae_1, \dots, Ae_n)$ spans $A(\mathbb{F}^n) = \Image{A}$
\end{itemize}
\end{definition}

\begin{definition}
The \textbf{coimage} (or \textbf{row space}) of $A$, $\Row{A}$, is the vector space formed by the span of the row vectors of $A$.
\begin{itemize}
    \item $\Row{A} = \Span{(A^1, \dots, A^m)}$
\end{itemize}
\end{definition}

\begin{definition}
The \textbf{rank} of $A$, $\rank{A}$, is the dimension of the image of the linear map associated with $A$.
\begin{itemize}
    \item $\rank{A} = \dim{(\Image{A})}$
\end{itemize}
\end{definition}

\begin{definition}
The \textbf{column rank} of $A$, $\colrk{(A)}$, is the maximal number of linearly independent columns of $A$.
\end{definition}

\begin{definition}
The \textbf{row rank} of $A$, $\rowrk{(A)}$, is the maximal number of linearly independent rows of $A$.
\end{definition}

\begin{theorem}
Let $A \in M(m \times n, \mathbb{F})$. Then,
\begin{equation*}
    \rank{A} = \colrk{A}
\end{equation*}
\end{theorem}
\begin{proof}
PROOF HERE

Note: the basic solution to the column $j$ without pivot is a linear relation among the columns of $A$. This implies that $A_j \in \Span{(\text{pivot columns of $A$})}$. Thus, the pivot columns of $A$ for a basis of $\col{A}$. Another way, we know $\dim{(\col{A})} =$ number of pivots, and $\dim{(\row{A})} = $ number of pivots, this is another proof of $\rowrk{A} = \colrk{A}$.
\end{proof}




\begin{theorem}
Let $A \in M(m \times n, \mathbb{F})$. Then,
\begin{equation*}
    \colrk{A} = \rowrk{A}
\end{equation*}
\end{theorem}
\begin{proof}
Let $A_k$ be a linear superfluous column of $A$. Then CONTINUE PROOF
\end{proof}

\begin{definition}
A row or column, $A^k$ or $A_k$ is \textbf{linearly superfluous} if and only if it can be expressed as a linear combination of the other rows/columns.
\end{definition}

\begin{theorem}
Let $A \in M(m \times n, \mathbb{F})$. If column $A_k$ is linearly superfluous and
\\ $A' = (A_1, \dots, A_{k-1}, A_{k+1}, \dots, A_n)$, then
\begin{equation*}
    \colrk{A'} = \colrk{A}
\end{equation*}
\end{theorem}
\begin{proof}
PROOF HERE.
\end{proof}

\begin{theorem}
Let $A \in M(m \times n, \mathbb{F})$. If row $A^k$ is linearly superfluous and
\\ $A' = (A^1, \dots, A^{k-1}, A^{k+1}, \dots, A^n)$, then
\begin{equation*}
    \rowrk{A'} = \rowrk{A}
\end{equation*}
\end{theorem}
\begin{proof}
PROOF HERE.
\end{proof}




\end{document}