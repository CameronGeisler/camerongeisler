\documentclass[letterpaper,12pt]{article}
\newcommand{\myname}{Cameron Geisler}
\newcommand{\mynumber}{90856741}
\usepackage{amsmath, amsfonts, amssymb}
\usepackage[paper=letterpaper,left=25mm,right=25mm,top=3cm,bottom=25mm]{geometry}
\usepackage{fancyhdr}
\usepackage{amsthm}
\usepackage{siunitx}
\pagestyle{fancy}

\lhead{Course}
\chead{Title}
\rhead{\myname \\ \mynumber}
\lfoot{\myname}
\cfoot{Page \thepage}
\rfoot{\mynumber}
\renewcommand{\headrulewidth}{0.4pt}
\renewcommand{\footrulewidth}{0.4pt}
\renewcommand\labelitemii{\textbullet} %changes 2nd level bullet to bullet

\setlength{\parindent}{0pt}
\theoremstyle{definition}
\newtheorem*{result}{Result}
\newtheorem*{definition}{Definition}
\newtheorem*{theorem}{Theorem}
\newtheorem*{example}{Example}
\newtheorem*{corollary}{Corollary}
\newtheorem*{lemma}{Lemma}
\usepackage{enumerate}
\newcommand{\ihat}{\hat{\imath}}
\newcommand{\jhat}{\hat{\jmath}}
\newcommand{\set}[1]{\left\{ #1 \right\}}
\renewcommand{\vec}[1]{\overrightarrow{#1}} %vector
\newcommand{\abs}[1]{\left\lvert #1 \right\rvert} %absolute value / magnitude of vector
\renewcommand{\neg}{\sim}

%% Linear algebra
\DeclareMathOperator{\Ker}{Ker}
\DeclareMathOperator{\nullity}{null}
\DeclareMathOperator{\Image}{Im}
\DeclareMathOperator{\Span}{Span}
\DeclareMathOperator{\rk}{rk}
\DeclareMathOperator{\colrk}{colrk}
\DeclareMathOperator{\rowrk}{rowrk}
\DeclareMathOperator{\Row}{Row}
\DeclareMathOperator{\Col}{Col}
\DeclareMathOperator{\matref}{ref}
\DeclareMathOperator{\matrref}{rref}
\DeclareMathOperator{\sol}{Sol}
\newcommand{\inp}[2]{\left< #1, #2 \right>}
\newcommand{\norm}[1]{\| #1 \|}


\begin{document}

\begin{theorem}
Let $V$ be a finite dimensional vector space, $f: V \rightarrow V$ be an endomorphism, $B = (v_1, \dots, v_n)$ be a basis for $V$ made up of eigenvectors of $f$, where the eigenvalue of $v_i$ is $\lambda_i$.
\\ \\ Then, $f(v_i) = \lambda_i v_i$, $v_i \neq 0$.
\\ \\ See OneNote
\end{theorem}

Conversely, if $f: V \rightarrow V$ is an endomorphism, there exists a basis $B$ such that $[f]_B = A$ is diagonal (say with $\lambda_i$s on the diagonal), then
\begin{align*}
    A(e_i) & = \lambda_i e_i \\
    \Phi^{-1} f \Phi (e_i) & = \lambda_i e_i \\
    f(\Phi(e_i)) & = \lambda_i \Phi(e_i)
\end{align*}
so $B = (\Phi(e_1), \dots, \Phi(e_n))$ is a basis consisting of eigenvectors of $f$.

\section*{Diagonalization}
Let $V$ be a vector space over $\mathbb{F}$, $f: V \rightarrow V$ be an endomorphism. $\lambda \in \mathbb{F}$ is an eigenvalue of $f$ if $\exists v \in V$, $v \neq 0$, such that $f(v) = \lambda v$.
\\ $\iff \ker{(\lambda Id_{V} - f)} \neq \set{0}$
\\ $\iff \det{(\lambda Id_{V} - f)} = 0$ (the characteristic equation)
\\ the eigenspace of the eigenvalue $\lambda$ is $E_{\lambda} = \ker{(\lambda Id_{V} - f)} = \set{0} \cup \set{\text{eigenvalues corresponding to $\lambda$}}$
\\ $\dim{E_{\lambda}}$ is the geometric multiplicity of $\lambda$



To determine if $f$ is diagonalizable,
\begin{enumerate}
    \item Find all eigenvalues, i.e. roots of $\lambda \longmapsto \det{(\lambda Id_{V} - f)}$
    \item For each eigenvalue, find a basis for $E_{\lambda}$ using the row reduction algorithm.
    \item Put them all together and see if you have $\dim{V}$ eigenvectors
\end{enumerate}

Assume $f$ is infinite, then for $P_n \longrightarrow Map(\mathbb{F}, \mathbb{F})$, the union of images is the space of polynomials.

\begin{lemma}
Let $p(t)$ be a polynomial of degree $n$, with root $\lambda \in \mathbb{F}$, i.e. $p(\lambda) = 0$. Then, $p(t) = (t - \lambda)q(t)$ for a unique polynomial $t$ of degree $n-1$.
\end{lemma}
\begin{proof}
$p(t+\lambda)$ is another polynomial of degree $n$ without constant term, because $0$ is a root. Thus, $p(t + \lambda) = tq(t)$, and so $p(t - \lambda + \lambda) = (t - \lambda) q(t - \lambda)$.
\end{proof}


If $p_f(t)$ is the characteristic polynomial of $f: V \rightarrow V$, with $\lambda$ eigenvalue. Then we can factor out the linear factors of $\lambda$, $p_f(t) = (t- \lambda)^k q(t)$, where $q(\lambda) \neq 0$. Here, $k$ is the algebraic multiplicity of $\lambda$.

\begin{corollary}
If $\lambda$ is an eigenvalue of $f: V \rightarrow V$, then the geometric multiplicity of $\lambda$ is less than or equal to the algebraic multiplicity of $\lambda$.
\end{corollary}

An eigenvalue is deficient if geometric mult. less than algebraic mult.
\\ Diagonalizability can go wrong in two ways:
\begin{enumerate}
    \item The characteristic polynomial may not have enough roots.
    \item Any of the eigenvalues may be deficient.
\end{enumerate}

\begin{theorem}
Every polynomial with complex coefficients has a root.
\end{theorem}

\begin{corollary}
Let $f: V \rightarrow V$ over $\mathbb{C}$, $\lambda_i, dots, \lambda_k$ be eigenvalues of $f$. Then,
\begin{align*}
    P_f(t) & = \Pi_{i=1}^k (t - t_i)^{m_i}
\end{align*}
where $m_i$ is the algebraic multiplicity of $\lambda_i$.
\\ Also, $\sum m_i = \sum \text{alg. mult.} = \dim{V}$
\\ Also, $f$ is diagonalizable if and only if none of the eigenvalues are deficient.
Thus, #1 is not a problem over $\mathbb{C}$, and for #2, 
\end{corollary}






\end{document}