\documentclass[letterpaper,12pt]{article}
\newcommand{\myname}{Cameron Geisler}
\newcommand{\mynumber}{90856741}
\usepackage{amsmath, amsfonts, amssymb}
\usepackage[paper=letterpaper,left=25mm,right=25mm,top=3cm,bottom=25mm]{geometry}
\usepackage{fancyhdr}
\pagestyle{fancy}

\lhead{Math 223}
\chead{Affine Spaces}
\rhead{ \myname \\ \mynumber }
\lfoot{\myname}
\cfoot{Page \thepage}
\rfoot{\mynumber}

\renewcommand{\headrulewidth}{0.4pt}
\renewcommand{\footrulewidth}{0.4pt}

\usepackage{amsthm}
\usepackage{siunitx}
\setlength{\parindent}{0pt}
\theoremstyle{definition}
\newtheorem*{result}{Result}
\newtheorem*{definition}{Definition}
\newtheorem*{theorem}{Theorem}
\newtheorem*{example}{Example}
\newtheorem*{corollary}{Corollary}
\newtheorem*{lemma}{Lemma}
\usepackage{enumerate}
\newcommand{\ihat}{\hat{\imath}}
\newcommand{\jhat}{\hat{\jmath}}
\newcommand{\set}[1]{\left\{ #1 \right\}}
\renewcommand{\vec}[1]{\overrightarrow{#1}} %vector
\newcommand{\abs}[1]{\lvert #1 \rvert} %absolute value / magnitude of vector
\renewcommand\labelitemii{\textbullet} %changes 2nd level bullet to bullet
\renewcommand{\neg}{\sim}

\DeclareMathOperator{\ker}{Ker}
\DeclareMathOperator{\nullity}{null}
\DeclareMathOperator{\Image}{Im}
\DeclareMathOperator{\Span}{Span}
\DeclareMathOperator{\rk}{rk}
\DeclareMathOperator{\colrk}{colrk}
\DeclareMathOperator{\rowrk}{rowrk}
\DeclareMathOperator{\Row}{Row}
\DeclareMathOperator{\Col}{Col}
\DeclareMathOperator{\Sol}{Sol}

\begin{document}

\begin{itemize}
    \item Affine transformations, applications to image morphing/processing?
\end{itemize}

\section*{Affine Spaces}
Recall: Let $A \in M(m \times n, \mathbb{F})$, $b \in \mathbb{F}^m$. If $x_0 \in \Soln{(A,b)}$ is a solution, then the solution set is
\begin{equation*}
    \Sol{(A,b)} = \set{x \in \mathbb{F}^n: Ax = b} = x_0 + \ker{A}
\end{equation*}
$\Sol{(A,b)}$ forms a bijection to $\ker{A}$, a vector space, which depends on the choice of $x_0$.
\begin{align*}
    \Sol{(A,b)} & \longleftrightarrow \ker{A} \\
    x & \longmapsto x - x_0
\end{align*}
Intuitively, $\Sol{(A,b)}$ is similar to a vector space, but without an origin. Affine spaces model this situation.

\begin{definition}
Let $V$ be a vector space over $\mathbb{F}$. An \textbf{affine space} over $V$ is a pair $(X, \oplus)$ where $X$ is a set, and $\oplus$ is the map
\begin{align*}
    \oplus : X \times V & \longrightarrow X \\
    (x,v) & \longmapsto x \oplus v
\end{align*}
which satisfies the following properties
\begin{enumerate}
    \item $X$ is \textbf{non-empty}, $X \neq \emptyset$.
    \item \textbf{Additive identity}: $\forall x \in X$, $x \oplus 0 = x$.
    \item \textbf{``Associativity"}: $\forall x \in X$, $v$, $w \in V$, $(x \oplus v) + w = x \oplus (v + w)$.
    \item $\forall x$, $y \in X$, there exists a unique vector $v \in V$ such that $x \oplus v = y$.
\end{enumerate}
\end{definition}
\begin{itemize}
    \item The elements of $X$ are called \textbf{points} (or ``locations" or ``positions").
    \item The elements of $V$ are called \textbf{vectors} (or \textbf{translations}, or \textbf{displacement vectors}).
\end{itemize}

\begin{example}
Time is modeled by an affine space $D$ over a vector space $L$.
\begin{itemize}
    \item The $1$-dimensional points of $D$ represent \textbf{dates} in time.
    \begin{itemize}
        \item Dates cannot be added to each other.
        \item For any two dates $d_1$ and $d_2$, there exists a unique duration $d$ such that $d = d_2 - d_1$.
    \end{itemize}
    \item The $1$-dimensional vectors in $L$ represent \textbf{durations}. They can be added and multiplied by real numbers.
    \begin{itemize}
        \item A choice of basis for $L$ represents a choice of units to measure durations, e.g. min $\in L$, min $\neq 0$ is a basis
    \end{itemize}
\end{itemize}
\end{example}

\begin{example}
"Physical space" or "mathematical space" is modeled by an affine space $X$ over a 3-dimensional vector space $V$ over $\mathbb{R}$
\end{example}

\begin{example}
Points, lines, and planes are affine spaces of dimension $1$, $2$, and $3$, respectively.
\end{example}


Affine spaces are helpful to make precise, solution sets to inhomogeneous systems of equations.
\begin{example}[Connection to Solution set]
Let $U$ be a vector space over $\mathbb{F}$, $V \subset U$ be a subspace, $x_0 \in U$ be a vector. Then,
\begin{align*}
    X = x_0 + V = \set{x_0 + v \in U : v \in V} \subset U
\end{align*}
\begin{align*}
    X \times V & \longrightarrow X \\
    (x,v) & \longmapsto x + v
\end{align*}
where $+$ is the addition from $U$.
\\ \\ $+$ is well defined because if $x \in X$, $v \in V$, then $x = x_0 + w$ for some $w \in V$. Then, $x + v = (x_0 + w) + v = x_0 + (w + v) \in X$, as $(w + v) \in V$.
\end{example}

\begin{enumerate}
    \item $X \neq \emptyset$ because $x_0 \in X$
    \item $x + 0 = x$, because $+$ is the addition in $U$ and property of $0 \in U$.
    \item associativity follows from associativity in $U$
    \item given $x$, $y \in X$. If $v$, $w \in V$ such that $y = x + v$ and $y = x + w$, then $v = w$ (calculations happens within vector space $U$). Existence, let $x$, $y \in X$. $x = x_0 + v$ and $y = x_0 + w$, for $v$, $w \in V$. Then, $y - x = w - v$, so $y = x + (w - v)$ and since $(w - v) \in V$ because $V$ is a subspace.
\end{enumerate}
In particular, for $V = U$ and $x_0 = 0$, $U$ is an affine space modeled on $U$. Every vector space is an affine space over itself.


\begin{theorem}
Let $X$ be an affine space over $V$. If $y = x + v$ for $x \in X$, $v \in V$. Then we write $v = y - x$. This defines a map
\begin{align*}
    X \times X & \longrightarrow V \\
    (y,x) & \longmapsto y - x
\end{align*}
\end{theorem}
For the example $X = x_0 + V$, for $x = x_0 + v$ and $y = y_0 + w$, we have $(x_0 + v) - (x_0 + w) = v - w$ because $(x_0 + v) = (x_0 + w) + (v - w)$ in $U$.
\\ Properties
\begin{itemize}
    \item $\forall x$, $y$, $z \in X$, $(y - x) + (x - z) = y - z$
\end{itemize}
\begin{proof}
Let $y = x + v$, $x = z + w$. Then,
\begin{align*}
    y & = x + v \\
    y & = (z + w) + v \\
    y & = z + (w + v) \\
    y - z & = w + v
\end{align*}
\end{proof}

Sometimes, people write for $P$, $Q \in X$ (P and Q are points), $\vec{PQ}$ is the unique vector $Q = P \oplus \vec{PQ}$. Then, $\vec{PQ} + \vec{QR} = \vec{PR}$, which follows from the axioms of affine spaces.

\begin{enumerate}
    \item[2] $(x + v) - x = v$. Exercise. 
\end{enumerate}



\\ \\ Given a vector $\vec{PQ}$ and points $P$ and $Q$, $Q$ can be written in terms of $\vec{PQ}$ and $P$. i.e. $Q = P + \vec{PQ}$. We can add displacement vectors to position to get positions.
\\ \\ For any two positions $P$ and $Q$, there is a unique displacement vector $\vec{PQ}$ such that $Q = P + \vec{PQ}$.
\\ \\ We can add displacements to each other. $\vec{PQ} + \vec{QR} = \vec{PR}$. We can also scale displacements by elements of $\mathbb{R}$. Thus, displacements form a vector space over $\mathbb{R}$.
\\ \\ Choice of origin $P_0 \in X$ sets up a bijection
\begin{align*}
    V & \longrightarrow X \\
    v & \longmapsto P_0 \oplus v \\
    \vec{P_0 Q} \longleftarrow Q
\end{align*}
Choosing a basis for $V$ (in physical case, also involves choosing units). This makes $V \cong \mathbb{R}^3$. Then $X \approx V$ (choice of origin in X) $\approx \mathbb{R}^3$ (choice of basis).

\section*{Affine Spaces}

\begin{itemize}
    \item $\forall x, y \in X$, $\exists ! v \in V$ such that $y = x \oplus v$. We write $v = y \ominus x$ or $v = \vec{xy}$.
\end{itemize}

\begin{definition}
Let $X$ be an affine space $V$, $\mathbb{F}$. An \textbf{affine subspace} of $X$ is a subset $Y \subset X$ such that there exists a vector subspace $W \subset V$ and
\begin{enumerate}
    \item $Y \neq \emptyset$
    \item $\forall y \in Y$, if $w \in W$, then $y \oplus w \in Y$
    \item $\forall y, y' \in Y$, $y \oplus y' \in W$
\end{enumerate}
\begin{itemize}
    \item $W$ is uniquely determined by $Y$, $\dim{Y} = \dim{W}$.
\end{itemize}
\end{definition}

\begin{itemize}
    \item If $Y \subset X$ is an affine subspace, then $Y$ is an affine space for $W$. By restriction,
    \begin{align*}
        Y \times W & \longrightarrow Y \\
    \end{align*}
    is a subset of 
    \begin{align*}
        X \times V & \longrightarrow X
    \end{align*}
\end{itemize}

\begin{theorem}
Let $Y \subset X$ be an affine subspace. If $x_0 \in Y$, then $Y = y_0 \oplus W$.
\end{theorem}


\begin{example}
If $f: V \rightarrow U$ is a linear map of $\mathbb{F}$ vector spaces $V$ and $U$. If $b \in U$, then $f^{-1}\set{b}$ is either empty, or is an affine subspace of $V$.
\begin{itemize}
    \item This is because if $f(x_0) = b$, then $f^{-1}{b} = x_0 + \ker{f}$.
\end{itemize}
\end{example}


\begin{definition}
Let $X$ be an affine space over $V$, $Y$ be an affine space over $U$. An \textbf{affine transformation} from $X$ to $Y$, $\phi$, is a map
\begin{align*}
    \phi : X \longrightarrow Y
\end{align*}
such that there exists a linear map $f: V \longrightarrow U$ with the property, $\forall x \in X, v \in V$, $\phi(x + v) = \phi(x) + f(v)$.
\begin{itemize}
    \item $f$ is determined by $\phi$.
\end{itemize}
\end{definition}

\begin{example}
$\phi: X \rightarrow X$ is a \textbf{translation} if the corresponding map $f: V \rightarrow V$ is $f = Id_{V}$.
\end{example}

\begin{lemma}
If $\phi : X \rightarrow X$ is a translation, then $\forall x \in X$, $\exists ! v \in V$ such that $\phi(x) = x \oplus v$.
\end{lemma}

\begin{proof}
Let $x \in X$, $v \in V$. Now pick $x_0 \in X$. Then, $\phi(x + v) = \phi(x) + v$. Let $v = \phi(x_0) - x_0$. We want to show that $\phi(x) = x + v$.
\\ \\ $\phi(x) = \phi(x_0 + (x - x_0)) = \phi(x_0) + x - x_0$ (Here $x - x_0$ represents $v$.
\begin{align*}
    & \phi(x_0) + x - x_0 \\
    & = (x_0 + v) + (x - x_0) \\
    & = (x_0 + (x - x_0)) + v \\
    & = x + v
\end{align*}
As desired.
\end{proof}

\begin{corollary}
Let $X$ be an affine space for Euclidian space $V$. The \textbf{distance} between $P$ and $Q$, $d(P,Q)$, is
\begin{align*}
    d(P,Q) & = \norm{Q - P} \in \mathbb{R} \\
    & = \norm{\vec{PQ}}
\end{align*}
\begin{itemize}
    \item If $ABC$ is a triangle, then the angle $BAC$ is the angle $(\vec{AB}, \vec{AC})$ for $\vec{AB}$, $\vec{AC} \in V$ (and $A \neq B$ and $A \neq C$). So, in $X$ we can do Euclidian geometry (using vectors).
\end{itemize}
\end{corollary}




\end{document}