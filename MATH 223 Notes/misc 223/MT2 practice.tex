\documentclass[letterpaper,12pt]{article}
\newcommand{\myname}{Cameron Geisler}
\newcommand{\mynumber}{90856741}
\usepackage{amsmath, amsfonts, amssymb}
\usepackage[paper=letterpaper,left=25mm,right=25mm,top=3cm,bottom=25mm]{geometry}
\usepackage{fancyhdr}
\usepackage{amsthm}
\usepackage{siunitx}
\pagestyle{fancy}

\lhead{Math 223}
\chead{Midterm 2 Practice}
\rhead{\myname \\ \mynumber}
\lfoot{\myname}
\cfoot{Page \thepage}
\rfoot{\mynumber}
\renewcommand{\headrulewidth}{0.4pt}
\renewcommand{\footrulewidth}{0.4pt}
\renewcommand\labelitemii{\textbullet} %changes 2nd level bullet to bullet

\setlength{\parindent}{0pt}
\theoremstyle{definition}
\newtheorem*{result}{Result}
\newtheorem*{definition}{Definition}
\newtheorem*{theorem}{Theorem}
\newtheorem*{example}{Example}
\newtheorem*{corollary}{Corollary}
\newtheorem*{lemma}{Lemma}
\usepackage{enumerate}
\newcommand{\ihat}{\hat{\imath}}
\newcommand{\jhat}{\hat{\jmath}}
\newcommand{\set}[1]{\left\{ #1 \right\}}
\renewcommand{\vec}[1]{\overrightarrow{#1}} %vector
\newcommand{\abs}[1]{\left\lvert #1 \right\rvert} %absolute value / magnitude of vector
\renewcommand{\neg}{\sim}

%% Linear algebra
\DeclareMathOperator{\Ker}{Ker}
\DeclareMathOperator{\nullity}{null}
\DeclareMathOperator{\Image}{Im}
\DeclareMathOperator{\Span}{Span}
\DeclareMathOperator{\rk}{rk}
\DeclareMathOperator{\colrk}{colrk}
\DeclareMathOperator{\rowrk}{rowrk}
\DeclareMathOperator{\Row}{Row}
\DeclareMathOperator{\Col}{Col}
\DeclareMathOperator{\matref}{ref}
\DeclareMathOperator{\matrref}{rref}
\DeclareMathOperator{\sol}{Sol}
\newcommand{\inp}[2]{\left< #1, #2 \right>}
\newcommand{\norm}[1]{\| #1 \|}

\newenvironment{amatrix}[1]{\left(\begin{array}{@{}*{#1}{c}|c@{}}}{\end{array}\right)}

\begin{document}

\begin{example}
Prove that $\mathcal{B} = \set{\begin{pmatrix} 1 \\ 1 \end{pmatrix}, \begin{pmatrix} -1 \\ 0 \end{pmatrix}}$ is a basis of $\mathbb{R}^2$
\\ \\ Let $\begin{pmatrix} x \\ y \end{pmatrix} \in \mathbb{R}^2$, $a$, $b \in \mathbb{R}$, with
\begin{equation*}
    \begin{pmatrix} x \\ y \end{pmatrix} = a \begin{pmatrix} 1 \\ 1 \end{pmatrix} + b \begin{pmatrix} -1 \\ 0 \end{pmatrix} = \begin{pmatrix} a - b \\ a \end{pmatrix}
\end{equation*}
Thus, $x = a - b$ and $y = a$, so the solutions for $a$ and $b$ are $a = y$ and $b = y - x$.
\end{example}

\begin{example}
Let $R: \mathbb{R}^2 \rightarrow \mathbb{R}^2$ be the rotation by $\pi/2$ about the origin, $\mathcal{B} = \set{\begin{pmatrix} 1 \\ 1 \end{pmatrix}, \begin{pmatrix} -1 \\ 0 \end{pmatrix}}$.
\\ Find the matrix of $R$ with respect to the basis$[R]$
\end{example}

\begin{example}
Let $k \in \mathbb{Q}$, $A_k = \begin{pmatrix} 1 & 1 & 1 \\ 1 & 2 & k \\ 1 & 4 & k^2 \end{pmatrix}$. Determine the values of $k$ such make $A_k$ invertible.
\begin{align*}
    0 & = \det{\begin{pmatrix} 1 & 1 & 1 \\ 1 & 2 & k \\ 1 & 4 & k^2 \end{pmatrix}} \\
    & = \det{\begin{pmatrix} 2 & k \\ 4 & k^2 \end{pmatrix}} - \det{\begin{pmatrix} 1 & k \\ 1 & k^2 \end{pmatrix}} + \det{\begin{pmatrix} 1 & 2 \\ 1 & 4 \end{pmatrix}} \\
    & = 2k^2 - 4k - (k^2 - k) + 2 \\
    & = k^2 - 3k + 2 \\
    & = (k-2)(k-1)
\end{align*}
Thus, if $k \neq 1$, $2$, $\det{A_k} \neq 0$, so $A_k$ is invertible.
\end{example}


\begin{example}
Let $\mathbb{F}_3$ be the field with $3$ elements. Consider the linear transformation
\begin{align*}
    f: \mathbb{F}_3^2 & \longrightarrow \mathbb{F}_3^3 \\
    (x,y) & \longmapsto (x+y,x-y,x)
\end{align*}
\end{example}

\begin{example}
Let $x \in \mathbb{R}$, $A_x = \begin{pmatrix} 3 & 5 & 0 \\ 5 & 0 & x \\ 0 & 3 & 3 \end{pmatrix}$. Determine the values of $x$ such that $A_x$ is invertible.
\begin{align*}
    0 & = \det{\begin{pmatrix} 3 & 5 & 0 \\ 5 & 0 & x \\ 0 & 3 & 3 \end{pmatrix}} \\
    & = 3 \det{\begin{pmatrix} 0 & x \\ 3 & 3 \end{pmatrix}} - 5 \det{\begin{pmatrix} 5 & x \\ 0 & 3 \end{pmatrix}} \\
    & = 3(-3x) - 5(15) \\
    & = -9x - 75
\end{align*}
Thus, $-9x - 75 = 0$, so $x = 25/3$.
\end{example}


\begin{example}
Let $Ax = b$ be an inhomogeneous system of equations over $\mathbb{R}$, with $b$ orthogonal to the columns of $A$. Then, $Ax = b$ has no solutions.
\end{example}
\begin{proof}
By contradiction, suppose that $Ax = b$ has a solution. In other words, $\exists x \in \mathbb{R}^n$ such that $Ax = b$. Then,
\begin{align*}
    b = Ax = x_1A_1 + \dots + x_nA_n
\end{align*}
Thus, mapping both sides with $\inp{\cdot}{b}$, we get
\begin{align*}
    \inp{b}{b} & = \inp{x_1A_1}{b} + \dots + \inp{x_nA_n}{b} \\
    & = x_1 \inp{A_1}{b} + \dots + x_n \inp{A_n}{b} \\
    & = 0 && \text{as $A_i$ is orthogonal to $b$}
\end{align*}
Thus, $\inp{b}{b} = 0$, so $b = 0$, which contradicts our assumption that $b \neq 0$. Thus, $Ax = b$ has no solution.
\end{proof}


\begin{example}
Let $V$ be a Euclidian vector space, $U \subset V$ be a subspace. The \textbf{reflection across $U$}, $R_U$, is defined as the linear map
\begin{align*}
    R_U: V & \longrightarrow V \\
    v & \longmapsto 2 \operatorname{proj}_{U}(v) - v
\end{align*}
\end{example}


\begin{example}
Let $A = \begin{pmatrix} 1 & 5 & 0 & -2 \\ 2 & 10 & -2 & 0 \\ 3 & 15 & -1 & 1 \end{pmatrix}$. Determine an invertible $3x3$ matrix $E$ such that $EA$ is in reduced row echelon form.
\\ \\ Consider the augmented matrix $(A \mid I_3)$
\begin{equation*}
    \left(\begin{array}{cccc|ccc}
    1 & 5 & 0 & -2 & 1 & 0 & 0 \\
    2 & 10 & -2 & 0 & 0 & 1 & 0 \\
    3 & 15 & -1 & 1 & 0 & 0 & 1
    \end{array} \right)
\end{equation*}
Converting to reduced row echelon form, we get
\begin{equation*}
    \left(\begin{array}{cccc|ccc}
    1 & 5 & 0 & 0 & 1/5 & -1/5 & 2/5 \\
    0 & 0 & 1 & 0 & 1/5 & -7/10 & 2/5 \\
    0 & 0 & 0 & 1 & -2/5 & -1/10 & 1/5
    \end{array} \right)
\end{equation*}
Thus,
\begin{equation*}
    E = \begin{pmatrix} 1/5 & -1/5 & 2/5 \\ 1/5 & -7/10 & 2/5 \\ -2/5 & -1/10 & 1/5 \end{pmatrix}
\end{equation*}
\end{example}


\begin{example}
Find the determinant of the matrix
\begin{equation*}
    A = \begin{pmatrix} 3 & 5 & 5 & 1 \\ 4 & 5 & 2 & 3 \\ 0 & 1 & 2 & 3 \\ 4 & 4 & 2 & 1 \end{pmatrix}
\end{equation*}
Using row operations, convert $A$ to row echelon form (i.e. an upper triangular matrix)
\begin{equation*}
    \begin{pmatrix} 3 & 5 & 5 & 1 \\ 0 & -5/3 & -14/3 & 5/3 \\ 0 & 0 & 24/5 & 2 \\ 0 & 0 & 0 & 21/12 \end{pmatrix}
\end{equation*}
Thus, $\det{A} = 3 (-5/3)(24/5)(21/12) = -42$.
\end{example}

\begin{example}
Let $V$, $W \subseteq \mathbb{R}^4$ be subspaces, with
\begin{align*}
    V & = \Span{\left(\begin{bmatrix} 1 \\ 0 \\ 1 \\ 1 \end{bmatrix}, \begin{bmatrix} 1 \\ 1 \\ 0 \\ 1 \end{bmatrix} \right)} && W = \Span{\left(\begin{bmatrix} 2 \\ 0 \\ 2 \\ 0 \end{bmatrix}, \begin{bmatrix} 0 \\ 1 \\ -1 \\ 2 \end{bmatrix} \right)}
\end{align*}
Determine a basis for $V \cap W$, and determine the dimension of $V \cap W$.
\end{example}




\end{document}




