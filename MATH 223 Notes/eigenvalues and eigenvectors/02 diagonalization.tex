\documentclass[letterpaper,12pt]{article}
\newcommand{\myname}{Cameron Geisler}
\newcommand{\mynumber}{90856741}
\usepackage{amsmath, amsfonts, amssymb, amsthm}
\usepackage[paper=letterpaper,left=25mm,right=25mm,top=3cm,bottom=25mm]{geometry}
\usepackage{fancyhdr}
\usepackage{float}
\usepackage{siunitx}
\usepackage{caption}
\usepackage{graphicx}
\pagestyle{fancy}
\usepackage{tkz-euclide} \usetkzobj{all} %% figures
\usepackage{hyperref} %% for links
\usepackage{exsheets} %% for tasks
\usepackage{systeme} %% for linear systems
\graphicspath{{../images/}} %% graphics in images folder

\lhead{Math 223} \chead{} \rhead{\myname}
\lfoot{} \cfoot{Page \thepage} \rfoot{}
\renewcommand{\headrulewidth}{0.4pt}
\renewcommand{\footrulewidth}{0.4pt}

\setlength{\parindent}{0pt}
\usepackage{enumerate}
\theoremstyle{definition}
\newtheorem*{definition}{Definition}
\newtheorem*{theorem}{Theorem}
\newtheorem*{example}{Example}
\newtheorem*{corollary}{Corollary}
\newtheorem*{lemma}{Lemma}
\newtheorem*{result}{Result}

%% Math
\newcommand{\abs}[1]{\left\lvert #1 \right\rvert}
\newcommand{\set}[1]{\left\{ #1 \right\}}
\renewcommand{\neg}{\sim}
\newcommand{\brac}[1]{\left( #1 \right)}
\newcommand{\eval}[1]{\left. #1 \right|}
\renewcommand{\vec}[1]{\mathbf{#1}}
\newenvironment{amatrix}[1]{\left[\begin{array}{@{}*{#1}{c}|c@{}}}{\end{array}\right]} %% for augmented matrix

\newcommand{\vecii}[2]{\left< #1, #2 \right>}
\newcommand{\veciii}[3]{\left< #1, #2, #3 \right>}

%% Linear algebra
\DeclareMathOperator{\Ker}{Ker}
\DeclareMathOperator{\nullity}{nullity}
\DeclareMathOperator{\Image}{Im}
\newcommand{\Span}[1]{\text{Span}\left(#1 \right)}
\DeclareMathOperator{\rank}{rank}
\DeclareMathOperator{\colrk}{colrk}
\DeclareMathOperator{\rowrk}{rowrk}
\DeclareMathOperator{\Row}{Row}
\DeclareMathOperator{\Col}{Col}
\DeclareMathOperator{\Null}{N}
\newcommand{\tr}[1]{tr\left( #1 \right)}
\DeclareMathOperator{\matref}{ref}
\DeclareMathOperator{\matrref}{rref}
\DeclareMathOperator{\sol}{Sol}
\newcommand{\inp}[2]{\left< #1, #2 \right>}
\newcommand{\norm}[1]{\left\lVert #1 \right\rVert}

%% Statistics
\newcommand{\prob}[1]{P\left( #1 \right)}
\newcommand{\overbar}[1]{\mkern 1.5mu \overline {\mkern-1.5mu#1 \mkern-1.5mu} \mkern 1.5mu}


\renewcommand{\frame}[1]{\tilde{\underline{\vec{#1}}}}

\chead{Diagonalization}

\begin{document}

\begin{definition}
Matrices $A$ and $B$ are \textbf{similar} if and only if there exists an invertible matrix $P$ such that
\begin{equation*}
    B = P^{-1} A P
\end{equation*}
\end{definition}



\section*{Diagonalization}
\begin{definition}
An $n \times n$ matrix $A$ is \textbf{diagonalizable} if it is similar to a diagonal matrix. In other words, there exists an invertible matrix $P$ such that $P^{-1}AP$ is diagonal.
\end{definition}

\begin{example}
Diagonal matrices $D$ are trivially diagonalizable, as $D = I^{-1} D I$, where $I$ is the identity matrix.
\end{example}

\begin{theorem}
If $A, B$ are similar $n \times n$ matrices, then they have the same eigenvalues.
\end{theorem}
\begin{proof}
If $A, B$ are similar, then there exists an invertible matrix $P$ such that $B = P^{-1} A P$. Then,
\end{proof}


















\section*{Misc Math 223}

\begin{corollary}
$A \in M(n \times n, \mathbb{F})$ is diagonalizable if and only if
\begin{itemize}
    \item The sum of the geometric multiplicities of the eigenvalues of $A$ is equal to $n$.
    \item There exists a basis of $\mathbb{F}^n$, $(v_1, \dots, v_n)$ consisting of eigenvectors of $A$
\end{itemize}
\end{corollary}

\begin{corollary}
If $(v_1, \dots, v_n)$ is an eigenbasis of $\mathbb{F}^n$, then $P$ can be formed with the basis vectors as columns, and the diagonal entries of the resulting diagonal matrix, $P^{-1}AP$, will be the eigenvalues of $A$.
\end{corollary}

\begin{definition}
Let $V$ be a finite-dimensional vector space, $f: V \rightarrow V$ be an endomorphism. $f$ is \textbf{diagonalizable} if and only if there exists a basis of $V$, $B = (v_1, \dots, v_n)$, such that $[f]_{B}$ is diagonalizable.
\end{definition}

\begin{corollary}
$f$ is diagonalizable if and only if
\begin{itemize}
    \item The sum of the geometric multiplicities of the eigenvalues of $f$ is equal to $\dim{V}$.
    \item There exists a basis of $V$, $(v_1, \dots, v_n)$, consisting of eigenvectors of $f$
\end{itemize}
\end{corollary}

\begin{corollary}
If $B = (v_1, \dots, v_n)$ is an eigenbasis of $V$, then the matrix of $f$ with respect to basis $B$, $[f]_{B}$, is diagonal, and the diagonal entries of $[f]_{B}$ are the eigenvalues of $f$. In other words, $a_{ii} = \lambda_i$, the eigenvalue of $v_i$.
\end{corollary}

\section*{Diagonalizing a Real Matrix}
Let $A \in M(n \times n, \mathbb{R})$.
\begin{enumerate}
    \item Determine the eigenvalues of $A$, $\lambda_1, \dots, \lambda_n$
    \item For each eigenvalue $\lambda_i$, find a basis for the eigenspace of $E_{\lambda}$
    \item Combine all the basis vectors $(v_1, \dots, v_n)$, and form the matrix $P$ with the basis vectors as columns.
    \item Form the diagonal matrix $D = (d_{ij})$ with $d_{ii} = \lambda_i$
    \item Then, $A = PDP^{-1}$
\end{enumerate}



\section*{Examples}
\begin{example}
Let $A = \begin{pmatrix} 0 & 2 \\ 1 & 1 \end{pmatrix}$. Then,
\begin{align*}
    \det{(A - \lambda I_n)} & = \det{\begin{pmatrix} -\lambda & 2 \\ 1 & 1 - \lambda \end{pmatrix}} = \lambda^2 - \lambda - 2 = (\lambda - 2)(\lambda + 1)
\end{align*}
Thus, the eigenvalues of $A$ are $\lambda = 2, -1$
\begin{align*}
    E_2 & = \Span{\begin{pmatrix} 1 \\ 1 \end{pmatrix}} = \ker{\begin{pmatrix} 2 & -2 \\ -1 & 1 \end{pmatrix}} \\
    E_{-1} & = \Span{\begin{pmatrix} 2 \\ -1 \end{pmatrix}} = \ker{\begin{pmatrix} -1 & -2 \\ -1 & -2 \end{pmatrix}}
\end{align*}
$\begin{pmatrix} 1 \\ 1 \end{pmatrix}$, $\begin{pmatrix} 2 \\ -1 \end{pmatrix}$ is a basis of $\mathbb{R}^2$ consisting of eigenvectors.
\end{example}























\end{document}