\documentclass[letterpaper,12pt]{article}
\newcommand{\myname}{Cameron Geisler}
\newcommand{\mynumber}{90856741}
\usepackage{amsmath, amsfonts, amssymb, amsthm}
\usepackage[paper=letterpaper,left=25mm,right=25mm,top=3cm,bottom=25mm]{geometry}
\usepackage{fancyhdr}
\usepackage{float}
\usepackage{siunitx}
\usepackage{caption}
\usepackage{graphicx}
\pagestyle{fancy}
\usepackage{tkz-euclide} \usetkzobj{all} %% figures
\usepackage{hyperref} %% for links
\usepackage{exsheets} %% for tasks
\usepackage{systeme} %% for linear systems
\graphicspath{{../images/}} %% graphics in images folder

\lhead{Math 223} \chead{} \rhead{\myname}
\lfoot{} \cfoot{Page \thepage} \rfoot{}
\renewcommand{\headrulewidth}{0.4pt}
\renewcommand{\footrulewidth}{0.4pt}

\setlength{\parindent}{0pt}
\usepackage{enumerate}
\theoremstyle{definition}
\newtheorem*{definition}{Definition}
\newtheorem*{theorem}{Theorem}
\newtheorem*{example}{Example}
\newtheorem*{corollary}{Corollary}
\newtheorem*{lemma}{Lemma}
\newtheorem*{result}{Result}

%% Math
\newcommand{\abs}[1]{\left\lvert #1 \right\rvert}
\newcommand{\set}[1]{\left\{ #1 \right\}}
\renewcommand{\neg}{\sim}
\newcommand{\brac}[1]{\left( #1 \right)}
\newcommand{\eval}[1]{\left. #1 \right|}
\renewcommand{\vec}[1]{\mathbf{#1}}
\newenvironment{amatrix}[1]{\left[\begin{array}{@{}*{#1}{c}|c@{}}}{\end{array}\right]} %% for augmented matrix

\newcommand{\vecii}[2]{\left< #1, #2 \right>}
\newcommand{\veciii}[3]{\left< #1, #2, #3 \right>}

%% Linear algebra
\DeclareMathOperator{\Ker}{Ker}
\DeclareMathOperator{\nullity}{nullity}
\DeclareMathOperator{\Image}{Im}
\newcommand{\Span}[1]{\text{Span}\left(#1 \right)}
\DeclareMathOperator{\rank}{rank}
\DeclareMathOperator{\colrk}{colrk}
\DeclareMathOperator{\rowrk}{rowrk}
\DeclareMathOperator{\Row}{Row}
\DeclareMathOperator{\Col}{Col}
\DeclareMathOperator{\Null}{N}
\newcommand{\tr}[1]{tr\left( #1 \right)}
\DeclareMathOperator{\matref}{ref}
\DeclareMathOperator{\matrref}{rref}
\DeclareMathOperator{\sol}{Sol}
\newcommand{\inp}[2]{\left< #1, #2 \right>}
\newcommand{\norm}[1]{\left\lVert #1 \right\rVert}

%% Statistics
\newcommand{\prob}[1]{P\left( #1 \right)}
\newcommand{\overbar}[1]{\mkern 1.5mu \overline {\mkern-1.5mu#1 \mkern-1.5mu} \mkern 1.5mu}


\renewcommand{\frame}[1]{\tilde{\underline{\vec{#1}}}}

\chead{Inverse of a Matrix}

\begin{document}

\section*{Inverse of a Matrix}
\begin{definition}
Let $A \in M(m \times n, \mathbb{F})$. The matrix inverse of $A$, $A^{-1}$, is the inverse map associated with the linear map of $A$.

$A$ is invertible, or $A^{-1}$ exists, if and only if
\begin{itemize}
    \item The linear map associated with $A$, $A: \mathbb{F}^n \rightarrow \mathbb{F}^n$, is an isomorphism.
\end{itemize}
\end{definition}

\section*{Inverse of a Matrix}
\begin{definition}
Two square matrices $A, B \in M(n \times n)$ are \textbf{inverses} of each other if their product is the identity matrix, or
\begin{equation*}
    AB = BA = I_n
\end{equation*}
\end{definition}

\begin{definition}
An $n \times n$ matrix $A$ is \textbf{invertible} if there exists an $n \times n$ matrix, denoted $A^{-1}$ called the \textbf{inverse} of $A$, such that
\begin{align*}
    AA^{-1} = I_n && A^{-1}A = I_n
\end{align*}
\begin{itemize}
    \item A matrix that is not invertible is called \textbf{singular}, and a matrix that is invertible is called \textbf{nonsingular}.
\end{itemize}
\end{definition}
In fact, for a square matrix $A$, $AA^{-1} = I_n$ if and only if $A^{-1}A = I_n$. Thus, to determine if a matrix $B$ is the inverse of $A$, one only needs to check either if $AB = I_n$, or if $BA = I_n$, as each implies the other.

\section*{Inverse of a $2 \times 2$ Matrix}
\begin{theorem}
Let $A = \begin{bmatrix} a & b \\ c & d \end{bmatrix}$ be a $2 \times 2$ matrix. If $ad - bc \neq 0$, then $A$ is invertible, and
\begin{equation*}
    A^{-1} = \dfrac{1}{ad - bc} \begin{bmatrix} d & -b \\ -c & a \end{bmatrix}
\end{equation*}
\end{theorem}
\begin{proof}
\begin{align*}
    AA^{-1} & = \dfrac{1}{ad - bc} \begin{bmatrix} a & b \\ c & d \end{bmatrix} \begin{bmatrix} d & -b \\ -c & a \end{bmatrix} \\
    & = \dfrac{1}{ad - bc} \begin{bmatrix} ad - bc & 0 \\ 0 & ad - bc \end{bmatrix} \\
    & = \begin{bmatrix} 1 & 0 \\ 0 & 1 \end{bmatrix}
\end{align*}
\end{proof}

Notice that if $ad - bc = 0$, the matrix $A$ does not have an inverse.

\begin{example}
Determine the inverse of the matrix
\begin{equation*}
    \begin{bmatrix} -4 & -8 \\ -1 & -2 \end{bmatrix}
\end{equation*}
The inverse does not exist.
\end{example}

\begin{example}
Let $A = \begin{bmatrix} 5 & -4 \\ -6 & 5 \end{bmatrix}, B = \begin{bmatrix} 5 & 4 \\ 6 & 5 \end{bmatrix}$. Determine if $A$ and $B$ are inverses.
\begin{align*}
    \begin{bmatrix} 5 & -4 \\ -6 & 5 \end{bmatrix} \begin{bmatrix} 5 & 4 \\ 6 & 5 \end{bmatrix} = \begin{bmatrix} 1 & 0 \\ 0 & 1 \end{bmatrix} \\
    \begin{bmatrix} 5 & 4 \\ 6 & 5 \end{bmatrix} \begin{bmatrix} 5 & -4 \\ -6 & 5 \end{bmatrix} = \begin{bmatrix} 1 & 0 \\ 0 & 1 \end{bmatrix}
\end{align*}
Thus, $A$ and $B$ are inverses of each other.
\end{example}

\begin{example}
Determine the inverse of the matrix
\begin{equation*}
    \begin{bmatrix} -1 & -3 \\ 4 & 5 \end{bmatrix}
\end{equation*}
is
\begin{equation*}
    \dfrac{1}{7} \begin{bmatrix} 5 & 3 \\ -4 & -1 \end{bmatrix}
\end{equation*}
\end{example}

\begin{example}
Determine the inverse of the matrix
\begin{equation*}
    \begin{bmatrix} 6 & 3 \\ 3 & 7 \end{bmatrix}
\end{equation*}
is
\begin{equation*}
    \begin{bmatrix} 7/33 & -1/11 \\ -1/11 & 2/11 \end{bmatrix}
\end{equation*}
\end{example}

\end{document}