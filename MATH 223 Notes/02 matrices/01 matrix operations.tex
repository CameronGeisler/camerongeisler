\documentclass[letterpaper,12pt]{article}
\newcommand{\myname}{Cameron Geisler}
\newcommand{\mynumber}{90856741}
\usepackage{amsmath, amsfonts, amssymb, amsthm}
\usepackage[paper=letterpaper,left=25mm,right=25mm,top=3cm,bottom=25mm]{geometry}
\usepackage{fancyhdr}
\usepackage{float}
\usepackage{siunitx}
\usepackage{caption}
\usepackage{graphicx}
\pagestyle{fancy}
\usepackage{tkz-euclide} \usetkzobj{all} %% figures
\usepackage{hyperref} %% for links
\usepackage{exsheets} %% for tasks
\usepackage{systeme} %% for linear systems
\graphicspath{{../images/}} %% graphics in images folder

\lhead{Math 223} \chead{} \rhead{\myname}
\lfoot{} \cfoot{Page \thepage} \rfoot{}
\renewcommand{\headrulewidth}{0.4pt}
\renewcommand{\footrulewidth}{0.4pt}

\setlength{\parindent}{0pt}
\usepackage{enumerate}
\theoremstyle{definition}
\newtheorem*{definition}{Definition}
\newtheorem*{theorem}{Theorem}
\newtheorem*{example}{Example}
\newtheorem*{corollary}{Corollary}
\newtheorem*{lemma}{Lemma}
\newtheorem*{result}{Result}

%% Math
\newcommand{\abs}[1]{\left\lvert #1 \right\rvert}
\newcommand{\set}[1]{\left\{ #1 \right\}}
\renewcommand{\neg}{\sim}
\newcommand{\brac}[1]{\left( #1 \right)}
\newcommand{\eval}[1]{\left. #1 \right|}
\renewcommand{\vec}[1]{\mathbf{#1}}
\newenvironment{amatrix}[1]{\left[\begin{array}{@{}*{#1}{c}|c@{}}}{\end{array}\right]} %% for augmented matrix

\newcommand{\vecii}[2]{\left< #1, #2 \right>}
\newcommand{\veciii}[3]{\left< #1, #2, #3 \right>}

%% Linear algebra
\DeclareMathOperator{\Ker}{Ker}
\DeclareMathOperator{\nullity}{nullity}
\DeclareMathOperator{\Image}{Im}
\newcommand{\Span}[1]{\text{Span}\left(#1 \right)}
\DeclareMathOperator{\rank}{rank}
\DeclareMathOperator{\colrk}{colrk}
\DeclareMathOperator{\rowrk}{rowrk}
\DeclareMathOperator{\Row}{Row}
\DeclareMathOperator{\Col}{Col}
\DeclareMathOperator{\Null}{N}
\newcommand{\tr}[1]{tr\left( #1 \right)}
\DeclareMathOperator{\matref}{ref}
\DeclareMathOperator{\matrref}{rref}
\DeclareMathOperator{\sol}{Sol}
\newcommand{\inp}[2]{\left< #1, #2 \right>}
\newcommand{\norm}[1]{\left\lVert #1 \right\rVert}

%% Statistics
\newcommand{\prob}[1]{P\left( #1 \right)}
\newcommand{\overbar}[1]{\mkern 1.5mu \overline {\mkern-1.5mu#1 \mkern-1.5mu} \mkern 1.5mu}


\renewcommand{\frame}[1]{\tilde{\underline{\vec{#1}}}}

\chead{Matrix Operations}

\begin{document}

Recall that an $m \times n$ matrix $A$ is a rectangular array of $m$ rows and $n$ columns of real numbers,
\begin{equation*}
    A = \begin{bmatrix}
    a_{11} & a_{12} & \dots & a_{1n} \\
    a_{21} & a_{22} & \dots & a_{2n} \\
    \vdots & \vdots & \ddots & \vdots \\
    a_{m1} & a_{n2} & \dots & a_{mn}
    \end{bmatrix}
\end{equation*}
The $j$th column of $A$ is the column vector $\vec{a}_j$. Then, the matrix can be written as,
\begin{equation*}
    A = \begin{bmatrix} \vec{a}_1 & \cdots & \vec{a}_n \end{bmatrix}
\end{equation*}

\begin{definition}
\begin{itemize}
    \item[]
    \item The \textbf{diagonal} entries of a matrix $A = \begin{bmatrix} a_{ij} \end{bmatrix}$ are $a_{11}, a_{22}, a_{33}, \dots$ which form the \textbf{main diagonal} of $A$.
    \item A \textbf{diagonal} matrix is a square $n \times n$ matrix whose non-diagonal entries are zero.
\end{itemize}
\end{definition}

\begin{example}
The $n \times n$ identity matrix $I_n$ is diagonal.
\end{example}

\section*{Equality of Matrices}
\begin{definition}
Two matrices $A$ and $B$ are \textbf{equal} if they have equal dimensions $m \times n$ and each corresponding entries are equal, $a_{ij} = b_{ij}$ for all $i, j$.
\end{definition}

\begin{example}
If
\begin{equation*}
    \begin{bmatrix} w & x \\ y & z \end{bmatrix} = \begin{bmatrix} -4 & 1 \\ -5 & 4 \end{bmatrix}
\end{equation*}
then since these two matrices are equal, their corresponding entries must be equal. So, $w = -4, x = 1, y = -5, z = 4$.
\end{example}

\section*{Matrix Addition, Subtraction, and Scalar Multiplication}
\begin{definition}
\textbf{Matrix addition}. Let $A, B$ be $m \times n$ matrices. Then, the \textbf{sum} of $A$ and $B$, $A + B$, is the matrix given by adding the corresponding entries of $A$ and $B$. In other words,
\begin{equation*}
    \boxed{(A + B)_{ij} = \begin{bmatrix} a_{ij} + b_{ij} \end{bmatrix}}
\end{equation*}
\end{definition}

Note that addition is only defined for matrices that have equal dimensions. Subtraction of matrices is defined similarly.

\begin{definition}
\textbf{Scalar multiplication}. Let $A$ be a $m \times n$ matrix, $k \in \mathbb{R}$. Then, $kA$ is the matrix formed by multiplying each of the entries of $A$ by $k$, or
\begin{equation*}
    \boxed{(kA)_{ij} = kA_{ij}}
\end{equation*}
\end{definition}

These operations can also be thought of in terms of column vectors.
\begin{itemize}
    \item The sum $A + B$ is the matrix whose columns are the sums of the corresponding columns of $A$ and $B$, or
    \begin{equation*}
        A + B = \begin{bmatrix} \vec{a}_1 + \vec{b}_1 & \cdots & \vec{a}_n + \vec{b}_n \end{bmatrix}
    \end{equation*}
    \item The scalar multiple $kA$ is the matrix whose columns are the scalar multiples of vectors $k \vec{a}_j$, or
    \begin{equation*}
        kA = \begin{bmatrix} k \vec{a}_1 & \cdots & k \vec{a}_n \end{bmatrix}
    \end{equation*}
\end{itemize}

Analogously to vectors $(-1)A$ is denoted by $-A$, and subtraction of matrices $A - B$ is just $A + (-1)B$.

\begin{example}
Add the matrices
\begin{equation*}
    \begin{bmatrix} 9 & 5 \\ -3 & -7 \end{bmatrix} + \begin{bmatrix} -1 & 7 \\ -1 & 8 \end{bmatrix}
\end{equation*}
The resulting matrix is
\begin{align*}
    \begin{bmatrix} 9 & 5 \\ -3 & -7 \end{bmatrix} + \begin{bmatrix} -1 & 7 \\ -1 & 8 \end{bmatrix} & = \begin{bmatrix} 9 + (-1) & 5 + 7 \\ -3 + (-1) & -7 + 8 \end{bmatrix} \\
    & = \begin{bmatrix} 8 & 12 \\ -4 & 1 \end{bmatrix}
\end{align*}
\end{example}

\begin{example}
Subtract the matrices
\begin{equation*}
    \begin{bmatrix} -1 & -3 \\ 6 & 4 \end{bmatrix} - \begin{bmatrix} -4 & 2 \\ 2 & 0 \end{bmatrix}
\end{equation*}
The resulting matrix is
\begin{align*}
    \begin{bmatrix} -1 & -3 \\ 6 & 4 \end{bmatrix} - \begin{bmatrix} -4 & 2 \\ 2 & 0 \end{bmatrix} & = \begin{bmatrix} -1 - (-4) & -3 - 2 \\ 6 - 2 & 4 - 0 \end{bmatrix} \\
    & = \begin{bmatrix} 3 & -5 \\ 4 & 4 \end{bmatrix}
\end{align*}
\end{example}

\begin{example}
Subtract the matrices
\begin{equation*}
    \begin{bmatrix} 9 & 6 \\ -3 & 1 \\ 4 & -9 \end{bmatrix} - \begin{bmatrix} -4 & -4 \\ -2 & -7 \\ -2 & -6 \end{bmatrix}
\end{equation*}
\begin{align*}
    \begin{bmatrix} 9 & 6 \\ -3 & 1 \\ 4 & -9 \end{bmatrix} - \begin{bmatrix} -4 & -4 \\ -2 & -7 \\ -2 & -6 \end{bmatrix} & = \begin{bmatrix} 9 - (-4) & 6 - (-4) \\ -3 - (-2) & 1 - (-7) \\ 4 - (-2) & -9 - (-6) \end{bmatrix} \\
    & = \begin{bmatrix} 13 & 10 \\ -1 & 8 \\ 6 & -3 \end{bmatrix}
\end{align*}
\end{example}

\begin{example}
Let $A = \begin{bmatrix} 0 & 4 \\ 4 & 3 \end{bmatrix}$. Determine $-5A$.
\begin{align*}
    -5A = -5 \begin{bmatrix} 0 & 4 \\ 4 & 3 \end{bmatrix} & = \begin{bmatrix} -5(0) & -5(4) \\ -5(4) & -5(3) \end{bmatrix} \\
    & = \begin{bmatrix} 0 & -20 \\ -20 & -15 \end{bmatrix}
\end{align*}
\end{example}

\section*{Properties of Matrix Addition}
\begin{definition}
An $m \times n$ matrix of which the entries are all zero is called a \textbf{zero matrix}, and is typically denoted by 0.
\end{definition}

The size of a zero matrix is usually clear from the context. The zero matrix has the property that $A + 0 = A$, i.e. it is the \textbf{additive identity} for the set of all $n \times m$ matrices. Note it is clear in this equation that 0 is a matrix, since a matrix $A$ cannot be added to the number 0.

\begin{theorem}
Let $A, B, C$ be $m \times n$ matrices, $0$ be the $m \times n$ zero matrix. Then, addition has the following properties,
\begin{enumerate}[(a)]
    \item \textbf{Commutative property}.
    \begin{equation*}
        A + B = B + A
    \end{equation*}
    \item \textbf{Associative property}.
    \begin{equation*}
        (A + B) + C = A + (B + C)
    \end{equation*}
    \item \textbf{Additive identity property}.
    \begin{equation*}
        A + 0 = A
    \end{equation*}
    \item \textbf{Additive inverse property}. For a matrix $A$, the matrix $B = \begin{bmatrix} -a_{ij} \end{bmatrix}$ consisting of the additive inverses of the entries of $A$, is the \textbf{additive inverse} of $A$, as $A + B = 0$, and $B$ is denoted by $-A$,
    \begin{equation*}
        A + (-A) = 0
    \end{equation*}
\end{enumerate}
\end{theorem}

The associative property means that sums of three or more matrices can be written without brackets, for example $A + B + C$.

\begin{theorem}
Let $A, B$ be $m \times n$ matrices, $r, s \in \mathbb{R}$. Then,
\begin{enumerate}[(a)]
    \item \textbf{Scalar distributes over matrix addition}.
    \begin{equation*}
        r(A + B) = rA + rB
    \end{equation*}
    \item \textbf{Matrix distributes over scalar addition}.
    \begin{equation*}
        (r + s)A = rA + sA
    \end{equation*}
    \item \textbf{Associativity of scalar multiplication}.
    \begin{equation*}
        r(sA) = (rs) A
    \end{equation*}
\end{enumerate}
\end{theorem}


\section*{Arithmetic of Matrices}
\begin{example}
Let $A = \begin{bmatrix} 8 & -6 \\ 9 & 3 \end{bmatrix}, B = \begin{bmatrix} 8 & -9 \\ -5 & -3 \end{bmatrix}$. Determine $-4A - 3B$.
\begin{align*}
    -4A - 3B & = -4 \begin{bmatrix} 8 & -6 \\ 9 & 3 \end{bmatrix} - 3 \begin{bmatrix} 8 & -9 \\ -5 & -3 \end{bmatrix} \\
    & = \begin{bmatrix} -32 & 24 \\ -36 & -12 \end{bmatrix} - \begin{bmatrix} 24 & -27 \\ -15 & -9 \end{bmatrix} \\
    & = \begin{bmatrix} -56 & 51 \\ -21 & -3 \end{bmatrix}
\end{align*}
\end{example}

\begin{example}
Let $A = \begin{bmatrix} 6 & -5 \\ 2 & 2 \end{bmatrix}, B = \begin{bmatrix} -7 & -6 \\ -2 & 9 \end{bmatrix}$. Determine $-2A + 4B$.
\begin{align*}
    -2A + 4B & = -2 \begin{bmatrix} 6 & -5 \\ 2 & 2 \end{bmatrix} + 4 \begin{bmatrix} -7 & -6 \\ -2 & 9 \end{bmatrix} \\
    & = \begin{bmatrix} -12 & 10 \\ -4 & -4 \end{bmatrix} + \begin{bmatrix} -28 & -24 \\ -8 & 36 \end{bmatrix} \\
    & = \begin{bmatrix} -40 & -14 \\ -12 & 32 \end{bmatrix}
\end{align*}
\end{example}



\section*{Misc}
\begin{itemize}
    \item The \textbf{ith row} of a matrix $A$, is the row vector (or row matrix) $\begin{bmatrix} a_{i1} & \dots & a_{in} \end{bmatrix}$
    \item The \textbf{jth column} of a matrix $A$, is the column vector (or column matrix) $\begin{bmatrix} a_{1j} \\ \vdots \\ a_{mj} \end{bmatrix}$
    \item The set of all $m \times n$ matrices with coefficients in $\mathbb{F}$ is $M(m \times n, \mathbb{F})$
\end{itemize}



\end{document}