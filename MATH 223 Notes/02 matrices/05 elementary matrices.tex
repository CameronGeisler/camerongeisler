\documentclass[letterpaper,12pt]{article}
\newcommand{\myname}{Cameron Geisler}
\newcommand{\mynumber}{90856741}
\usepackage{amsmath, amsfonts, amssymb, amsthm}
\usepackage[paper=letterpaper,left=25mm,right=25mm,top=3cm,bottom=25mm]{geometry}
\usepackage{fancyhdr}
\usepackage{float}
\usepackage{siunitx}
\usepackage{caption}
\usepackage{graphicx}
\pagestyle{fancy}
\usepackage{tkz-euclide} \usetkzobj{all} %% figures
\usepackage{hyperref} %% for links
\usepackage{exsheets} %% for tasks
\usepackage{systeme} %% for linear systems
\graphicspath{{../images/}} %% graphics in images folder

\lhead{Math 223} \chead{} \rhead{\myname}
\lfoot{} \cfoot{Page \thepage} \rfoot{}
\renewcommand{\headrulewidth}{0.4pt}
\renewcommand{\footrulewidth}{0.4pt}

\setlength{\parindent}{0pt}
\usepackage{enumerate}
\theoremstyle{definition}
\newtheorem*{definition}{Definition}
\newtheorem*{theorem}{Theorem}
\newtheorem*{example}{Example}
\newtheorem*{corollary}{Corollary}
\newtheorem*{lemma}{Lemma}
\newtheorem*{result}{Result}

%% Math
\newcommand{\abs}[1]{\left\lvert #1 \right\rvert}
\newcommand{\set}[1]{\left\{ #1 \right\}}
\renewcommand{\neg}{\sim}
\newcommand{\brac}[1]{\left( #1 \right)}
\newcommand{\eval}[1]{\left. #1 \right|}
\renewcommand{\vec}[1]{\mathbf{#1}}
\newenvironment{amatrix}[1]{\left[\begin{array}{@{}*{#1}{c}|c@{}}}{\end{array}\right]} %% for augmented matrix

\newcommand{\vecii}[2]{\left< #1, #2 \right>}
\newcommand{\veciii}[3]{\left< #1, #2, #3 \right>}

%% Linear algebra
\DeclareMathOperator{\Ker}{Ker}
\DeclareMathOperator{\nullity}{nullity}
\DeclareMathOperator{\Image}{Im}
\newcommand{\Span}[1]{\text{Span}\left(#1 \right)}
\DeclareMathOperator{\rank}{rank}
\DeclareMathOperator{\colrk}{colrk}
\DeclareMathOperator{\rowrk}{rowrk}
\DeclareMathOperator{\Row}{Row}
\DeclareMathOperator{\Col}{Col}
\DeclareMathOperator{\Null}{N}
\newcommand{\tr}[1]{tr\left( #1 \right)}
\DeclareMathOperator{\matref}{ref}
\DeclareMathOperator{\matrref}{rref}
\DeclareMathOperator{\sol}{Sol}
\newcommand{\inp}[2]{\left< #1, #2 \right>}
\newcommand{\norm}[1]{\left\lVert #1 \right\rVert}

%% Statistics
\newcommand{\prob}[1]{P\left( #1 \right)}
\newcommand{\overbar}[1]{\mkern 1.5mu \overline {\mkern-1.5mu#1 \mkern-1.5mu} \mkern 1.5mu}


\renewcommand{\frame}[1]{\tilde{\underline{\vec{#1}}}}

\chead{Elementary Matrices}

\begin{document}

\section*{Elementary Matrices}
In fact, performing row operations on a matrix $A$ is equivalent to multiplying $A$ by a particular matrix. Further, such matrices are simple to determine.

\begin{definition}
An \textbf{elementary matrix} is a matrix which is obtained by performing a single elementary row operation on an identity matrix.
\end{definition}

\begin{theorem}
Let $A$ be an $m \times n$ matrix. If an elementary row operation is performed on $A$, the resulting matrix can be written as $EA$ for an $m \times m$ elementary matrix $E$ formed by performing the same row operation on $I_m$.
\end{theorem}

In other words, every elementary row operation corresponds to left-multiplication by some elementary matrix.
\\ \\ In addition, every elementary matrix is invertible, since row operations are reversable, so if $E$ corresponds to some row operation, then the reverse operation has a corresponding elementary matrix $F$ such that $EF = FE = I$. In summary,

\begin{theorem}
Every elementary matrix $E$ is invertible, and the inverse of $E$ is the elementary matrix of the same type as $E$ which transforms $E$ back into $I$.
\end{theorem}


\section*{Inverting an $n \times n$ Matrix}
\begin{theorem}
An $n \times n$ matrix $A$ is invertible if and only if $A$ is row equivalent to $I_n$, and in this case, any sequence of row operations that reduces $A$ to $I_n$ also transforms $I_n$ into $A^{-1}$.
\end{theorem}

\begin{proof}
If $A$ is invertible, then the equation $A\vec{x} = \vec{b}$ has a solution for every $\vec{b}$, and $A$ has a pivot position in every row ($n$ rows). Since $A$ is square (in particular only has $n$ columns), this implies that the pivot positions are on the main diagonal. Thus, the RREF of $A$ is $I_n$, or $A \sim I_n$.
\\ \\ Conversely, suppose that $A \sim I_n$. Then, each step in the row reduction of $A$ corresponds to left-multiplication by some elementary matrix, so if there are say $p$ steps to reduce $A$ to $I_n$, then there exist elementary matrices $E_1, \dots, E_p$ such that
\begin{equation*}
    A \sim E_1 A \sim E_2(E_1 A) \sim \cdots \sim E_p(E_{p-1} \cdots E_1 A) = I_n
\end{equation*}
In other words,
\begin{equation*}
    E_p \cdots E_1 A = I_n
\end{equation*}
In other words, $(E_p \cdots E_1) A = I_n$, i.e. $A^{-1} = E_p \cdots E_1$. In other words, $A^{-1} = E_p \cdots E_1 \cdot I_n$, so $A^{-1}$ is obtained by applying $E_1, \dots, E_p$ successively to $I_n$, the same sequence required to transform $A$ to $I_n$.
\end{proof}

The previous theorem leads to an algorithm for determining matrix inverses,

\begin{enumerate}
    \item Form the augmented matrix $\begin{bmatrix} A \mid I \end{bmatrix}$.
    \item Row reduce $\begin{bmatrix} A \mid I \end{bmatrix}$.
    \item If $A$ is row equivalent to $I$, then $\begin{bmatrix} A \mid I \end{bmatrix}$ will be of the form $\begin{bmatrix} I \mid B \end{bmatrix}$. Then, $B = A^{-1}$.
    \item Otherwise, $A$ does not have an inverse.
\end{enumerate}

After using this procedure, the result can be verified by checking that $AA^{-1} = I$.

\end{document}