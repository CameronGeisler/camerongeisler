\documentclass[letterpaper,12pt]{article}
\newcommand{\myname}{Cameron Geisler}
\newcommand{\mynumber}{90856741}
\usepackage{amsmath, amsfonts, amssymb, amsthm}
\usepackage[paper=letterpaper,left=25mm,right=25mm,top=3cm,bottom=25mm]{geometry}
\usepackage{fancyhdr}
\usepackage{float}
\usepackage{siunitx}
\usepackage{caption}
\usepackage{graphicx}
\pagestyle{fancy}
\usepackage{tkz-euclide} \usetkzobj{all} %% figures
\usepackage{hyperref} %% for links
\usepackage{exsheets} %% for tasks
\usepackage{systeme} %% for linear systems
\graphicspath{{../images/}} %% graphics in images folder

\lhead{Math 223} \chead{} \rhead{\myname}
\lfoot{} \cfoot{Page \thepage} \rfoot{}
\renewcommand{\headrulewidth}{0.4pt}
\renewcommand{\footrulewidth}{0.4pt}

\setlength{\parindent}{0pt}
\usepackage{enumerate}
\theoremstyle{definition}
\newtheorem*{definition}{Definition}
\newtheorem*{theorem}{Theorem}
\newtheorem*{example}{Example}
\newtheorem*{corollary}{Corollary}
\newtheorem*{lemma}{Lemma}
\newtheorem*{result}{Result}

%% Math
\newcommand{\abs}[1]{\left\lvert #1 \right\rvert}
\newcommand{\set}[1]{\left\{ #1 \right\}}
\renewcommand{\neg}{\sim}
\newcommand{\brac}[1]{\left( #1 \right)}
\newcommand{\eval}[1]{\left. #1 \right|}
\renewcommand{\vec}[1]{\mathbf{#1}}
\newenvironment{amatrix}[1]{\left[\begin{array}{@{}*{#1}{c}|c@{}}}{\end{array}\right]} %% for augmented matrix

\newcommand{\vecii}[2]{\left< #1, #2 \right>}
\newcommand{\veciii}[3]{\left< #1, #2, #3 \right>}

%% Linear algebra
\DeclareMathOperator{\Ker}{Ker}
\DeclareMathOperator{\nullity}{nullity}
\DeclareMathOperator{\Image}{Im}
\newcommand{\Span}[1]{\text{Span}\left(#1 \right)}
\DeclareMathOperator{\rank}{rank}
\DeclareMathOperator{\colrk}{colrk}
\DeclareMathOperator{\rowrk}{rowrk}
\DeclareMathOperator{\Row}{Row}
\DeclareMathOperator{\Col}{Col}
\DeclareMathOperator{\Null}{N}
\newcommand{\tr}[1]{tr\left( #1 \right)}
\DeclareMathOperator{\matref}{ref}
\DeclareMathOperator{\matrref}{rref}
\DeclareMathOperator{\sol}{Sol}
\newcommand{\inp}[2]{\left< #1, #2 \right>}
\newcommand{\norm}[1]{\left\lVert #1 \right\rVert}

%% Statistics
\newcommand{\prob}[1]{P\left( #1 \right)}
\newcommand{\overbar}[1]{\mkern 1.5mu \overline {\mkern-1.5mu#1 \mkern-1.5mu} \mkern 1.5mu}


\renewcommand{\frame}[1]{\tilde{\underline{\vec{#1}}}}

\chead{Determinants}

\begin{document}



\section*{Determinant of Upper Triangular Matrix}
\begin{theorem}
If $A \in M(n \times n, \mathbb{F})$ in an upper triangular matrix, then $\det{A}$ is given by the product of the diagonal elements.
\begin{equation*}
    \det{A} = a_{11} \cdot \dots \cdot a_{nn}
\end{equation*}
\end{theorem}
\begin{proof}
By induction, let $n = 1$. Then, if $A \in M(1 \times 1, \mathbb{F})$, $A = \begin{pmatrix} a_{11} \end{pmatrix}$ is upper triangular, and $\det{A} = a_{11}$.
\\ \\ Then, assume that for $A \in M((n-1) \times (n-1), \mathbb{F})$, if $A$ is upper triangular, then $\det{A} = a_{11} \cdot \dots \cdot a_{(n-1)(n-1)}$. Then, for $A \in M(n \times n, \mathbb{F})$,
\begin{align*}
    \det{A} & = \sum_{i=1}^n (-1)^{i+1} a_{i 1} A_{i 1} \\
    & = a_{11} \det{A_{11}} && \text{as $a_{i1} = 0$ for $i \neq 1$} \\
    & = a_{11} (a_{22} \cdot \dots \cdot a_{nn}) && \text{by the induction hypothesis for an $(n-1)\times(n-1)$ matrix} \\
    \det{A} & = a_{11} \cdot \dots \cdot a_{nn}
\end{align*}
Thus, by induction, the result is true for all $n \in \mathbb{N}$.
\end{proof}

\begin{corollary}
Especially for matrices $4 \times 4$ or larger, to determine $\det{A}$, use row operations R1 and R3 to convert $A$ into row echelon form, $A'$ (since $A$ is square, this will make $A$ upper triangular). Then, if $r$ row swaps were used,
\begin{equation*}
    \det{A} = (-1)^r a'_{11} \cdot \dots \cdot a'_{nn}
\end{equation*}
\end{corollary}

\begin{corollary}
$A$ is invertible (i.e. $\rank{A} = n$) if and only if $\det{A} \neq 0$.
\end{corollary}
\begin{proof}
The product of the diagonal entries of a matrix in upper triangular form is equal to zero if and only if at least one of them is zero.
\begin{itemize}
    \item If $\rank{A} < n$, then by property $2$, $\det{A} = 0$.
    \item If $\rank{A} = n$, then row operations can convert $A$ into an upper triangular matrix which has a pivot in every row and column. Thus, all of the diagonal entries are non-zero, so $\det{A} \neq 0$.
\end{itemize}
\end{proof}

Strictly speaking, the $\det$ we have discussed is the "row determinant", as we required that $\det$ is linear for each row. We can define, similarly, a "column determinant", which requires $\operatorname{coldet}{A}$ to be linear in each column (along with properties $2$ and $3$).
\\ \\ Since $\colrk{A} = \rank{A}$, $\colrk{A} < n$ if and only if $\rank{A} < n$ (property 2). Also, the identity matrix is symmetric as $(I_n)^t = I_n$, so $\operatorname{coldet}{I_n} = \det{I_n} = 1$ (property 3). Finally, we need to show that $\operatorname{coldet}{A}$ is linear in each row.

\begin{theorem}
The determinant, $\det: M(n \times n, \mathbb{F}) \rightarrow \mathbb{F}$, is linear in the each column.
\end{theorem}
\begin{proof}
Consider the column expansion formula for the $j$th column,
\begin{align*}
    \det{A} = \sum_{i=1}^n (-1)^{i+j} a_{ij} A_{ij}
\end{align*}
$A_{ij}$ is independent of the $j$th column (as the $j$th column has been removed). Thus, $A$ is linear in row $j$.
\end{proof}

Therefore, similarly to the equality of $\rowrk$ and $\colrk$, "row determinant" and "column determinant" are equivalent. In other words, \textbf{the determinant is linear in each row and each column}.

\section*{Determinant of a Transpose}
\begin{theorem}
Let $A \in M(n \times n, \mathbb{F})$. Then,
\begin{equation*}
    \boxed{\det{A^t} = \det{A}}
\end{equation*}
\end{theorem}
\begin{proof}
Let $A \in M(n \times n, \mathbb{F})$. Then,
\begin{align*}
    \det{A^t} & = \operatorname{rowdet}{A^t} && \text{definition of determinant} \\
    & = \operatorname{coldet}{A^t} && \text{equality of $\operatorname{rowdet}$ and $\operatorname{coldet}$} \\
    & = \operatorname{rowdet}{A} && \text{definition of transpose} \\
    \det{A^t} & = \det{A} && \text{definition of determinant}
\end{align*}
\end{proof}

\section*{Laplace Expansion along a Row}
\begin{corollary}
Let $A \in M(n \times n, \mathbb{F})$, $j \in \set{1, \dots, n}$. Then,
\begin{equation*}
    \boxed{\det{A} = \sum_{i=1}^n (-1)^{i+j} a_{ij} \det{A_{ij}}}
\end{equation*}
\end{corollary}

\section*{Adjugate of a Matrix}
\begin{definition}
Let $A \in M(n \times n, \mathbb{F})$. The \textbf{adjugate} of $A$, $\operatorname{adj}{A} \in M(n \times n, \mathbb{F})$ is the transpose of its cofactor matrix.
\begin{equation*}
    \operatorname{adj}{A} = (-1)^{i+j} \det{A_{ji}}
\end{equation*}
\end{definition}

\begin{theorem}
Let $A \in M(n \times n, \mathbb{F})$. If $\det{A} \neq 0$, then,
\begin{equation*}
    A^{-1} = \dfrac{1}{\det{A}} \adj{A}
\end{equation*}
\end{theorem}

\begin{corollary}
Let $A = \begin{pmatrix} a & b \\ c & d \end{pmatrix}$. Then,
\begin{equation*}
    \boxed{A^{-1} = \dfrac{1}{\det{A}} \begin{pmatrix} d & -b \\ -c & a \end{pmatrix}}
\end{equation*}
\end{corollary}

\section*{Determinant of a Product}
\begin{theorem}[Determinant of a product]
Let $A$, $B \in M(n \times n, \mathbb{F})$. Then,
\begin{equation*}
    \boxed{\det{AB} = \det{A} \cdot \det{B}}
\end{equation*}
\end{theorem}
\begin{proof}
Let $B \in M(n \times n, \mathbb{F})$. Consider the map
\begin{align*}
    f: M(n \times n, \mathbb{F}) & \longrightarrow \mathbb{F} \\
    A & \longmapsto \dfrac{1}{\det{B}} \det{AB}
\end{align*}
$f$ is linear in each row, as changing the $i$th row of $A$ only affects the $i$th row of $AB$. In other words, the following map is linear:
\begin{align*}
    \mathbb{F}^n & \longrightarrow \mathbb{F}^n \\
    A^i & \longmapsto (AB)^i
\end{align*}
Thus, $f$ satisfies property $1$.
\\ \\ Also, if $\rank{A} < n$, then since $\Image{AB} \subset \Image{A}$, we have $\rank{AB} \leq \rank{A} < n$. Thus,
\begin{equation*}
    f(A) = \dfrac{1}{\det{B}} \det{AB} = \dfrac{1}{\det{B}} \cdot 0 = 0
\end{equation*}
Thus, $f$ satisfies property $2$.
\\ \\ Finally,
\begin{equation*}
    f(I_n) = \dfrac{1}{\det{B}} \det(I_n B) = \dfrac{1}{\det{B}} \det{B} = 1
\end{equation*}
Thus, $f$ satisfies property $3$, and so $f$ is the determinant map. In other words, $f(A) = \det{A}$. Thus,
\begin{itemize}
    \item If $\det{B} \neq 0$, then we have
    \begin{align*}
        \det{A} & = \dfrac{1}{\det{B}} \det{AB} \\
        \det{AB} & = \det{A} \cdot \det{B}
    \end{align*}
    \item If $\det{B} = 0$, then $\rank{B} < n$. Then, since $\Image{AB} \subset \Image{A}$, we have $\rank{AB} < n$, and so $\det{AB} = 0$.
\end{itemize}
\end{proof}

\begin{corollary}
Let $A \in M(n \times n, \mathbb{F})$. If $A$ is invertible, then
\begin{equation*}
    \boxed{\det{(A^{-1})} = \dfrac{1}{\det{A}}}
\end{equation*}
\end{corollary}
\begin{proof}
\begin{equation*}
    1 = \det{I_n} = \det{(AA^{-1})} = \det{A} \cdot \det{A^{-1}}
\end{equation*}
Thus, $\det{A^{-1}} = \dfrac{1}{\det{A}}$.
\end{proof}

\begin{definition}
Let $f: V \rightarrow V$ be an endomorphism, $[f]$ be the matrix corresponding to $f$, $B = (v_1, \dots, v_n)$ be a basis of $V$.
\\ Then, the \textbf{determinant of $f$ with respect to $B$} is $\det{f} = \det{[f]_B^B}$
\end{definition}

\section*{Determinant of an Endomorphism}
\begin{corollary}
We can define $\det{F}$ for any endomorphism $F: V \longrightarrow V$, where $V$ is a finite dimensional vector space over $\mathbb{F}$.
\end{corollary}
\begin{proof}
Let $B$ be a basis of $V$. Define $\det{F} = \det{[f]_B^B}$. If $C$ is another basis, then since $[f]_C^B [g]_B^C = [f \circ g]_C^A$ and $[id]_C^B [id]_B^C = I_n$
\begin{equation*}
    [id]_C^B [F]_B^B [id]_B^C = [F]_C^C
\end{equation*}
Then,
\begin{equation*}
    \det{[id]_C^B [F]_B^B [id]_B^C} = \det{[id]_C^B} + \det{[F]_B^B} + \det{[id]_B^C} = \det{[F]_C^C}
\end{equation*}
\end{proof}




\section*{Examples}
\begin{example}
Let $T: \mathbb{R}^2 \rightarrow \mathbb{R}^2$ be a linear transformation such that
\begin{align*}
    T\left(\begin{bmatrix} 1 \\ 1 \end{bmatrix} \right) & = \begin{bmatrix} 1 \\ 2 \end{bmatrix} && T\left(\begin{bmatrix} 1 \\ 2 \end{bmatrix} \right) = \begin{bmatrix} 1 \\ 3 \end{bmatrix}
\end{align*}
Determine the determinant of $T$.
\\ \\ Let $[T] = \begin{bmatrix} a & b \\ c & d \end{bmatrix}$. Then,
\begin{align*}
    \begin{bmatrix} a & b \\ c & d \end{bmatrix} \begin{bmatrix} 1 \\ 1 \end{bmatrix} & = \begin{bmatrix} 1 \\ 2 \end{bmatrix} \\
    \begin{bmatrix} a & b \\ c & d \end{bmatrix} \begin{bmatrix} 1 \\ 2 \end{bmatrix} & = \begin{bmatrix} 1 \\ 3 \end{bmatrix}
\end{align*}
This forms the system of equations
\begin{align*}
    a + b & = 1 \\
    c + d & = 2 \\
    a + 2b & = 1 \\
    c + 2d & = 3
\end{align*}
Forming the augmented matrix, we get
\begin{equation*}
    \begin{bmatrix}
    1 & 1 & 0 & 0 & 1 \\
    0 & 0 & 1 & 1 & 2 \\
    1 & 2 & 0 & 0 & 1 \\
    0 & 0 & 1 & 2 & 3
    \end{bmatrix}
\end{equation*}
Converting to reduced row echelon form, we get
\begin{equation*}
    \begin{bmatrix}
    1 & 0 & 0 & 0 & 1 \\
    0 & 1 & 0 & 0 & 0 \\
    0 & 0 & 1 & 0 & 1 \\
    0 & 0 & 0 & 1 & 1
    \end{bmatrix}
\end{equation*}
Thus, $[T] = \begin{bmatrix} 1 & 0 \\ 1 & 1 \end{bmatrix}$
\\ \\ Therefore, $\det{[T]} = \det{\begin{bmatrix} 1 & 0 \\ 1 & 1 \end{bmatrix}} = 1$
\end{example}

\section*{Area of a Triangle Using Determinants}
\begin{theorem}
A triangle in the plane with vertices $(a,b), (c,d), (e,f)$ has area
\begin{equation*}
    A = \frac{1}{2} \begin{vmatrix} a & b & 1 \\ c & d & 1 \\ e & f & 1 \end{vmatrix}
\end{equation*}
\end{theorem}

\section*{Evaluating Determinants Using a Graphing Calculator}

\end{document}