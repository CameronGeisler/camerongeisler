\documentclass[letterpaper,12pt]{article}
\newcommand{\myname}{Cameron Geisler}
\newcommand{\mynumber}{90856741}
\usepackage{amsmath, amsfonts, amssymb, amsthm}
\usepackage[paper=letterpaper,left=25mm,right=25mm,top=3cm,bottom=25mm]{geometry}
\usepackage{fancyhdr}
\usepackage{float}
\usepackage{siunitx}
\usepackage{caption}
\usepackage{graphicx}
\pagestyle{fancy}
\usepackage{tkz-euclide} \usetkzobj{all} %% figures
\usepackage{hyperref} %% for links
\usepackage{exsheets} %% for tasks
\usepackage{systeme} %% for linear systems
\graphicspath{{../images/}} %% graphics in images folder

\lhead{Math 223} \chead{} \rhead{\myname}
\lfoot{} \cfoot{Page \thepage} \rfoot{}
\renewcommand{\headrulewidth}{0.4pt}
\renewcommand{\footrulewidth}{0.4pt}

\setlength{\parindent}{0pt}
\usepackage{enumerate}
\theoremstyle{definition}
\newtheorem*{definition}{Definition}
\newtheorem*{theorem}{Theorem}
\newtheorem*{example}{Example}
\newtheorem*{corollary}{Corollary}
\newtheorem*{lemma}{Lemma}
\newtheorem*{result}{Result}

%% Math
\newcommand{\abs}[1]{\left\lvert #1 \right\rvert}
\newcommand{\set}[1]{\left\{ #1 \right\}}
\renewcommand{\neg}{\sim}
\newcommand{\brac}[1]{\left( #1 \right)}
\newcommand{\eval}[1]{\left. #1 \right|}
\renewcommand{\vec}[1]{\mathbf{#1}}
\newenvironment{amatrix}[1]{\left[\begin{array}{@{}*{#1}{c}|c@{}}}{\end{array}\right]} %% for augmented matrix

\newcommand{\vecii}[2]{\left< #1, #2 \right>}
\newcommand{\veciii}[3]{\left< #1, #2, #3 \right>}

%% Linear algebra
\DeclareMathOperator{\Ker}{Ker}
\DeclareMathOperator{\nullity}{nullity}
\DeclareMathOperator{\Image}{Im}
\newcommand{\Span}[1]{\text{Span}\left(#1 \right)}
\DeclareMathOperator{\rank}{rank}
\DeclareMathOperator{\colrk}{colrk}
\DeclareMathOperator{\rowrk}{rowrk}
\DeclareMathOperator{\Row}{Row}
\DeclareMathOperator{\Col}{Col}
\DeclareMathOperator{\Null}{N}
\newcommand{\tr}[1]{tr\left( #1 \right)}
\DeclareMathOperator{\matref}{ref}
\DeclareMathOperator{\matrref}{rref}
\DeclareMathOperator{\sol}{Sol}
\newcommand{\inp}[2]{\left< #1, #2 \right>}
\newcommand{\norm}[1]{\left\lVert #1 \right\rVert}

%% Statistics
\newcommand{\prob}[1]{P\left( #1 \right)}
\newcommand{\overbar}[1]{\mkern 1.5mu \overline {\mkern-1.5mu#1 \mkern-1.5mu} \mkern 1.5mu}


\renewcommand{\frame}[1]{\tilde{\underline{\vec{#1}}}}

\chead{Properties of Determinants}

\begin{document}

\section*{Determinants and Row Operations}
Consider a $2 \times 2$ matrix $A = \begin{bmatrix} a & b \\ c & d \end{bmatrix}$, with determinant $\det{A} = ad - bc$. Consider how the determinant is affected by the elementary row operations.

\begin{enumerate}[(a)]
    \item For replacement (say $R_2 + kR_1 \rightarrow R_2$),
    \begin{align*}
        \det{\begin{bmatrix} a & b \\ c + ka & d + kb \end{bmatrix}} & = a(d + kb) - b(c + ka) \\
        & = ad - bc + abk - abk \\
        & = ad - bc = \det{A}
    \end{align*}
    In other words, the determinant is unchanged.
    \item For scaling, say row 1 is scaled by $k$,
    \begin{align*}
        \det{\begin{bmatrix} ka & kb \\ c & d \end{bmatrix}} & = ka(d) - kb(c) \\
        & = k(ad - bc) = k \det{A}
    \end{align*}
    i.e. the determinant is also scaled by $k$.
    \item For row interchange,
    \begin{align*}
        \det{\begin{bmatrix} c & d \\ a & b \end{bmatrix}} & = cb - ad \\
        & = -(ad - bc) = -\det{A}
    \end{align*}
    i.e. it changes the sign of the determinant.
\end{enumerate}

These properties are true for determinants of any size matrix.

\begin{theorem}
Let $A$ be a square matrix, $B$ be the matrix $A$ after some row operation.
\begin{enumerate}[(a)]
    \item If a multiple of one row is added to another row, then $\det{B} = \det{A}$.
    \item If two rows are interchanged, then $\det{B} = -\det{A}$.
    \item If one row is scaled by $k$, then $\det{B} = k\det{A}$.
\end{enumerate}
\end{theorem}

In particular, only row interchanges and scalings affect the value of a determinant.

\section*{Evaluating Determinants of Higher Order}
One strategy for evaluating higher-order determinants i.e. for a larger matrix $A$, is to use row reduction to convert them to REF, say $U$. Recall that matrices in REF are upper triangular. Then, the determinant of $U$ is the product of its diagonal entries. Then, the determinant of $A$ is given by,
\begin{equation*}
    \det{A} = c \det{U}
\end{equation*}
where $c$ is the product of all of the constant factors and sign changes accumulated by row interchanges and scalings. In particular, if no scalings are used (which is possible), then
\begin{equation*}
    \det{A} = (-1)^r \det{U}
\end{equation*}
where $r$ is the number of row interchanges.

\section*{Invertible if and only if Non-zero Determinant}
\begin{theorem}
A square matrix $A$ is invertible if and only if $\det{A} \neq 0$.
\end{theorem}



\section*{Determinant of a Transpose}
\begin{theorem}
Let $A$ be a square matrix. Then,
\begin{equation*}
    \boxed{\det{A^T} = \det{A}}
\end{equation*}
\end{theorem}

\section*{Determinant of a Product}

\begin{theorem}
\textbf{Multiplicative property}. Let $A, B$ be $n \times n$ matrices. Then,
\begin{equation*}
    \boxed{\det{(AB)} = \det{A} \cdot \det{B}}
\end{equation*}
More generally,
\begin{equation*}
    \boxed{\det{(A_1 \cdots A_k)} = \det{A_1} \cdots \det{A_k}}
\end{equation*}
\end{theorem}


Note that the analogous statement does not apply for a sum of matrices, that is, the determinant of a sum of matrices is not in general equal to the sum of the determinants, or $\det{(A + B)} \neq \det{A} + \det{B}$.



\end{document}