\documentclass[letterpaper,12pt]{article}
\newcommand{\myname}{Cameron Geisler}
\newcommand{\mynumber}{90856741}
\usepackage{amsmath, amsfonts, amssymb, amsthm}
\usepackage[paper=letterpaper,left=25mm,right=25mm,top=3cm,bottom=25mm]{geometry}
\usepackage{fancyhdr}
\usepackage{float}
\usepackage{siunitx}
\usepackage{caption}
\usepackage{graphicx}
\pagestyle{fancy}
\usepackage{tkz-euclide} \usetkzobj{all} %% figures
\usepackage{hyperref} %% for links
\usepackage{exsheets} %% for tasks
\usepackage{systeme} %% for linear systems
\graphicspath{{../images/}} %% graphics in images folder

\lhead{Math 223} \chead{} \rhead{\myname}
\lfoot{} \cfoot{Page \thepage} \rfoot{}
\renewcommand{\headrulewidth}{0.4pt}
\renewcommand{\footrulewidth}{0.4pt}

\setlength{\parindent}{0pt}
\usepackage{enumerate}
\theoremstyle{definition}
\newtheorem*{definition}{Definition}
\newtheorem*{theorem}{Theorem}
\newtheorem*{example}{Example}
\newtheorem*{corollary}{Corollary}
\newtheorem*{lemma}{Lemma}
\newtheorem*{result}{Result}

%% Math
\newcommand{\abs}[1]{\left\lvert #1 \right\rvert}
\newcommand{\set}[1]{\left\{ #1 \right\}}
\renewcommand{\neg}{\sim}
\newcommand{\brac}[1]{\left( #1 \right)}
\newcommand{\eval}[1]{\left. #1 \right|}
\renewcommand{\vec}[1]{\mathbf{#1}}
\newenvironment{amatrix}[1]{\left[\begin{array}{@{}*{#1}{c}|c@{}}}{\end{array}\right]} %% for augmented matrix

\newcommand{\vecii}[2]{\left< #1, #2 \right>}
\newcommand{\veciii}[3]{\left< #1, #2, #3 \right>}

%% Linear algebra
\DeclareMathOperator{\Ker}{Ker}
\DeclareMathOperator{\nullity}{nullity}
\DeclareMathOperator{\Image}{Im}
\newcommand{\Span}[1]{\text{Span}\left(#1 \right)}
\DeclareMathOperator{\rank}{rank}
\DeclareMathOperator{\colrk}{colrk}
\DeclareMathOperator{\rowrk}{rowrk}
\DeclareMathOperator{\Row}{Row}
\DeclareMathOperator{\Col}{Col}
\DeclareMathOperator{\Null}{N}
\newcommand{\tr}[1]{tr\left( #1 \right)}
\DeclareMathOperator{\matref}{ref}
\DeclareMathOperator{\matrref}{rref}
\DeclareMathOperator{\sol}{Sol}
\newcommand{\inp}[2]{\left< #1, #2 \right>}
\newcommand{\norm}[1]{\left\lVert #1 \right\rVert}

%% Statistics
\newcommand{\prob}[1]{P\left( #1 \right)}
\newcommand{\overbar}[1]{\mkern 1.5mu \overline {\mkern-1.5mu#1 \mkern-1.5mu} \mkern 1.5mu}


\renewcommand{\frame}[1]{\tilde{\underline{\vec{#1}}}}

\chead{Advanced Determinants, Linear Maps}

\begin{document}

\begin{definition}
The \textbf{determinant} is the unique map
\begin{equation*}
    \det: M(n \times n, \mathbb{F}) \longrightarrow \mathbb{F}
\end{equation*}
with the following properties:
\begin{enumerate}
    \item The map $\det$ is linear in each row.
    \begin{itemize}
        \item Let $A$ be a matrix with all rows, $A^1, \dots, A^{k-1}, A^{k+1}, \dots, A^n \in \mathbb{F}$, held constant except the $k$th row, $x$. Then, $A$ is linear in row $k$.
        \\ \\ In other words, the following map is linear:
        \begin{align*}
            \mathbb{F}^n & \longrightarrow \mathbb{F} \\
            x = (x_1, \dots, x_n) & \longmapsto \det{(A^1, \dots, A^{k-1}, x, A^{k+1}, \dots, A^n)}
        \end{align*}
    \end{itemize}
    \item If $\rank{A} < n$, then $\det{A} = 0$.
    \begin{corollary}
    If $A$ has two equal rows, then $\det{A} = 0$.
    \end{corollary}
    \item $\det{I_n} = 1$
\end{enumerate}
\end{definition}
Note that the determinant takes in a square matrix, and returns a number.

\begin{theorem}
Let $\det: M(n \times n, \mathbb{F}) \rightarrow \mathbb{F}$ be a map, such that $\det$ is linear in each row, and if $\rank{A} < n$, then $\det{A} = 0$. Then,
\begin{enumerate}
    \item If $A'$ is a matrix obtained from $A$ by interchanging two rows, then $\det{A'} = - \det{A}$.
    \item If $A'$ is a matrix obtained from $A$ by multiplying a row by $\lambda \in \mathbb{F}$, then $\det{A'} = \lambda \det{A}$.
    \item If $A'$ is a matrix obtained from $A$ by adding a multiple of a row to another row, then $\det{A'} = \det{A}$.
\end{enumerate}
\end{theorem}
\begin{proof}
\begin{enumerate}
    \item Let $A = \begin{pmatrix} \dots \\ A^i \\ \dots \\ A^j \\ \dots \end{pmatrix}$, such that $A^i$ and $A^j$ are rows to be exchanged. Then, consider
    \begin{align*}
        0 & = \det{\begin{pmatrix} A^i + A^j \\ \dots \\ A^i + A^j \end{pmatrix}} && \text{two equal rows} \\
        & = \det{\begin{pmatrix} A^i \\ \dots \\ A^i + A^j \end{pmatrix}} + \det{\begin{pmatrix} A^j \\ \dots \\ A^i + A^j \end{pmatrix}} && \text{linear in top row} \\
        & = \det{\begin{pmatrix} A^i \\ \dots \\ A^i \end{pmatrix}} + \det{\begin{pmatrix} A^i \\ \dots \\ A^j \end{pmatrix}} + \det{\begin{pmatrix} A^j \\ \dots \\ A^i \end{pmatrix}} + \det{\begin{pmatrix} A^j \\ \dots \\ A^j \end{pmatrix}} && \text{linear in bottom row} \\
        & = \det{\begin{pmatrix} A^i \\ \dots \\ A^j \end{pmatrix}} + \det{\begin{pmatrix} A^j \\ \dots \\ A^i \end{pmatrix}} \\
        & = \det{A} + \det{A'}
    \end{align*}
    Thus,
    \begin{equation*}
        \det{A'} = -\det{A}
    \end{equation*}
    
    \item Let $A = \begin{pmatrix} \dots \\ A^i \\ \dots \end{pmatrix}$, $A' = \begin{pmatrix} \dots \\ \lambda A^i \\ \dots \end{pmatrix}$. Then, since $\det{A'}$ is linear in each row,
    \begin{equation*}
        \det{A'} = \det{\begin{pmatrix} \dots \\ \lambda A^i \\ \dots \end{pmatrix}} = \lambda \det{\begin{pmatrix} \dots \\ A^i \\ \dots \end{pmatrix}} = \lambda \det{A}
    \end{equation*}
    
    \item Let $A = \begin{pmatrix} A^i \\ \dots \\ A^j \end{pmatrix}$, $A' = \begin{pmatrix} A^i + \lambda A^j \\ \dots \\ A^j \end{pmatrix}$. Then,
    \begin{align*}
        \det{A'} & = \begin{pmatrix} A^i + \lambda A^j \\ \dots \\ A^j \end{pmatrix} \\
        & = \det{\begin{pmatrix} A^i \\ \dots \\ A^j \end{pmatrix}} + \lambda \det{\begin{pmatrix} A^j \\ \dots \\ A^j \end{pmatrix}} && \text{linearity in row $i$} \\
        & = \det{\begin{pmatrix} A^i \\ \dots \\ A^j \end{pmatrix}} && \text{two equal rows} \\
        \det{A'} & = \det{A}
    \end{align*}
\end{enumerate}
\end{proof}

\begin{theorem}
There exists a unique map $\det$ with above properties $1$, $2$, and $3$.
\end{theorem}
\begin{proof}
\textbf{Uniqueness}. Let $\det$ and $\det'$ be two maps that both satisfy properties $1$, $2$, and $3$. Let $A \in M(n \times n, \mathbb{F})$. We want to show that $\det{A} = \det'{A}$.
    \\ \\ First, note that if we let $A'$ be a matrix obtained from $A$ after elementary row operations, then since elementary row operations are reversible, $\det{A} = \det'{A}$ if and only if $\det{A'} = \det'{A'}$.
    \begin{itemize}
        \item If $\rank{A} < n$, then $0 = \det{A} = \det{A'}$.
        \item If $\rank{A} = n$, then $A$ can be converted into the identity matrix $I_n$, by a sequence of row operations. Since $1 = \det{I_n} = \det'{I_n}$, it follows that $\det{A} = \det{A'}$, as desired.
    \end{itemize}
\textbf{Existance}. By induction, let $n = 1$. Then, the map $\det: M(1 \times 1, \mathbb{F}) \rightarrow \mathbb{F}, \begin{pmatrix} a \end{pmatrix} \mapsto a$ satisfies properties $1$, $2$, and $3$.
\\ \\ Then, assume that $\det$ exists for all $A \in M((n-1) \times (n-1), \mathbb{F})$, with properties $1$, $2$, and $3$. Let $A_{ij}$ represent the matrix obtained from $A$ by removing the $i$th row and $j$th column. Then, let $k \in \set{1, \dots, n}$, and define $\det$ as,
\begin{align*}
    \det{A} & = \sum_{i=1}^n (-1)^{i+k} a_{ik} \det{A_{ik}}
\end{align*}
We want to show that $\det$ satisfies properties 1, 2, and 3, for $A \in M(n \times n, \mathbb{F})$.
\begin{enumerate}
    \item We want to show that $\det{A}$ is linear in the $j$th row of $A$, for all $j \in \set{1, \dots, n}$. We can show this by showing that each summand of $\det{A}$, $(-1)^{i+k} a_{ik} \det{A_{ik}}$, is linear.
    \begin{itemize}
        \item For $j \neq i$, $a_{ik}$ is independent of the $j$th row, so it can be treated as a constant. And, since $\det: M((n-1) \times (n-1), \mathbb{F}) \rightarrow \mathbb{F}$ has property $1$, $\det{A_{ik}}$ has property $1$.
        \item For $j = i$ (i.e. the term with the $j$th row removed), note that the map $M(n \times n, \mathbb{F}) \rightarrow \mathbb{F}$, $A \mapsto a_{ik}$ is linear, so $a_{ik}$ is linear. Also, $A_{ik}$ is independent of the $j$th row of $A$, (as it was omitted), so it can be treated as a constant.
    \end{itemize}
    \item We want to show that if $\rank{A} < n$, then $\det{A} = 0$.
    \\ \\ Let $\rank{A} < n$. Then, there exists a row $A^k$ that can be written as a linear combination of the other rows. In other words, $\exists \lambda_1, \dots, \lambda_n \in \mathbb{F}$ such that
    \begin{equation*}
        A^k = \sum_{i \neq k} \lambda_i A^i = \lambda_1 A^1 + \dots + \lambda_{k-1} A^{k-1} + \lambda_{k+1} + \dots + \lambda_n A^n
    \end{equation*}
    Then,
    \begin{align*}
        \det{A} & = \det{\begin{pmatrix} \dots \\ A^k \\ \dots \end{pmatrix}} \\
        & = \det{\begin{pmatrix} \dots \\ \lambda_1 A^1 + \dots + \lambda_{k-1} A^{k-1} + \lambda_{k+1} A^{k+1} + \dots + \lambda_n A^n \\ \dots \end{pmatrix}} \\
        & = \lambda_1 \det{\begin{pmatrix} \dots \\ A^1 \\ \dots \end{pmatrix}} + \dots + \lambda_{k-1} \det{\begin{pmatrix} \dots \\ A^{k-1} \\ \dots \end{pmatrix}} + \lambda_{k+1} \det{\begin{pmatrix} \dots \\ A^{k+1} \\ \dots \end{pmatrix}} + \dots + \lambda_n \det{\begin{pmatrix} \dots \\ A^n \\ \dots \end{pmatrix}} \\
        & \text{linear in row $k$} \\
        & = 0 \\
        & \text{as each determinant has two repeated rows}
    \end{align*}
    \item First, $\det{I_1} = \det{1} = 1$. Then, assume that $\det{I_{n-1}} = 1$, and we want to show that $\det{I_n} = 1$. We write the identity matrix as $(I_n)_{ij} = \delta_{ij} = \begin{cases} 0 & i \neq j \\ 1 & i = j \end{cases}$. Thus,
    \begin{align*}
        \det{I_n} & = \sum_{i = 1}^n (-1)^{i+k} \delta_{ik} \det{((I_n)_{kk})} \\
        & = (-1)^{k + k} \delta_{kk} \det{((I_n)_{kk})} && \text{as $\delta_{ij} = 0$ for $i \neq j$} \\
        & = \det{((I_n)_{kk})} \\
        & = \det{I_{n-1}} \\
        & = 1
    \end{align*}
\end{enumerate}
Thus, the formula for $\det{A}$ satisfies all three properties.
\end{proof}

\end{document}