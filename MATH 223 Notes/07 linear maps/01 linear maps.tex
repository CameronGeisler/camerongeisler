\documentclass[letterpaper,12pt]{article}
\newcommand{\myname}{Cameron Geisler}
\newcommand{\mynumber}{90856741}
\usepackage{amsmath, amsfonts, amssymb, amsthm}
\usepackage[paper=letterpaper,left=25mm,right=25mm,top=3cm,bottom=25mm]{geometry}
\usepackage{fancyhdr}
\usepackage{float}
\usepackage{siunitx}
\usepackage{caption}
\usepackage{graphicx}
\pagestyle{fancy}
\usepackage{tkz-euclide} \usetkzobj{all} %% figures
\usepackage{hyperref} %% for links
\usepackage{exsheets} %% for tasks
\usepackage{systeme} %% for linear systems
\graphicspath{{../images/}} %% graphics in images folder

\lhead{Math 223} \chead{} \rhead{\myname}
\lfoot{} \cfoot{Page \thepage} \rfoot{}
\renewcommand{\headrulewidth}{0.4pt}
\renewcommand{\footrulewidth}{0.4pt}

\setlength{\parindent}{0pt}
\usepackage{enumerate}
\theoremstyle{definition}
\newtheorem*{definition}{Definition}
\newtheorem*{theorem}{Theorem}
\newtheorem*{example}{Example}
\newtheorem*{corollary}{Corollary}
\newtheorem*{lemma}{Lemma}
\newtheorem*{result}{Result}

%% Math
\newcommand{\abs}[1]{\left\lvert #1 \right\rvert}
\newcommand{\set}[1]{\left\{ #1 \right\}}
\renewcommand{\neg}{\sim}
\newcommand{\brac}[1]{\left( #1 \right)}
\newcommand{\eval}[1]{\left. #1 \right|}
\renewcommand{\vec}[1]{\mathbf{#1}}
\newenvironment{amatrix}[1]{\left[\begin{array}{@{}*{#1}{c}|c@{}}}{\end{array}\right]} %% for augmented matrix

\newcommand{\vecii}[2]{\left< #1, #2 \right>}
\newcommand{\veciii}[3]{\left< #1, #2, #3 \right>}

%% Linear algebra
\DeclareMathOperator{\Ker}{Ker}
\DeclareMathOperator{\nullity}{nullity}
\DeclareMathOperator{\Image}{Im}
\newcommand{\Span}[1]{\text{Span}\left(#1 \right)}
\DeclareMathOperator{\rank}{rank}
\DeclareMathOperator{\colrk}{colrk}
\DeclareMathOperator{\rowrk}{rowrk}
\DeclareMathOperator{\Row}{Row}
\DeclareMathOperator{\Col}{Col}
\DeclareMathOperator{\Null}{N}
\newcommand{\tr}[1]{tr\left( #1 \right)}
\DeclareMathOperator{\matref}{ref}
\DeclareMathOperator{\matrref}{rref}
\DeclareMathOperator{\sol}{Sol}
\newcommand{\inp}[2]{\left< #1, #2 \right>}
\newcommand{\norm}[1]{\left\lVert #1 \right\rVert}

%% Statistics
\newcommand{\prob}[1]{P\left( #1 \right)}
\newcommand{\overbar}[1]{\mkern 1.5mu \overline {\mkern-1.5mu#1 \mkern-1.5mu} \mkern 1.5mu}


\renewcommand{\frame}[1]{\tilde{\underline{\vec{#1}}}}

\chead{Linear Maps}

\begin{document}


Linear maps are functions defined on vector spaces that are ``structure preserving", in that the images they produce still have ``linear properties" like addition and scalar multiplication.

\section*{Linear Maps Between Vector Spaces}
\begin{definition}
Let $V$, $W$ be vector spaces over $\mathbb{F}$. A map $f: V \rightarrow W$ is a \textbf{linear map} (or a \textbf{linear transformation}, or is an \textbf{homomorphism}), if $f$ is operation preserving. In other words, for all $x, y \in V$, $\lambda \in \mathbb{F}$,
\begin{enumerate}[(a)]
    \item $f(x + y) = f(x) + f(y)$
    \item $f(\lambda x) = \lambda f(x)$
\end{enumerate}
\end{definition}


\begin{theorem}
Linear maps have the following properties:
\begin{enumerate}[(a)]
    \item $f(0) = 0$, as $f(0) = f(0 \cdot x) = 0 \cdot f(x) = 0$
    \item $f(\lambda x + y) = \lambda f(x) + f(y)$
    \item $f(x - y) = f(x) - f(y)$
    \item $f\left(\sum_{i=1}^n \lambda_i x_i \right) = f(\lambda_1 x_1 + \dots + \lambda_n x_n) = \lambda_1 f(x_1) + \dots + \lambda_n f(x_n)$
\end{enumerate}
\end{theorem}

\begin{example}
The identity map $\operatorname{Id}_{V}: V \rightarrow V$ is linear.
\end{example}
\begin{proof}
Let $x, y \in \mathbb{V}$, $\lambda \in \mathbb{F}$. Then,
\begin{equation*}
    \operatorname{Id}_{V}(x + y) = x + y = \operatorname{Id}_{V}(x) + \operatorname{Id}_{V}(y)
\end{equation*}
\begin{equation*}
    \operatorname{Id}_{V}(\lambda x) = \lambda x = \lambda \operatorname{Id}_{V}(x)
\end{equation*}
\end{proof}

\begin{example}
\begin{align*}
    f: \mathbb{R}^2 & \longrightarrow \mathbb{R}^2 \\
    (x,y) & \longmapsto (x - y, y - x)
\end{align*}
is a linear map.
\end{example}
\begin{proof}
Let $x, y \in \mathbb{R}^2$, $\lambda \in \mathbb{R}$, with $x = (x_1, x_2)$, $y = (y_1, y_2)$. Then,
\begin{align*}
    f(x + y) & = f(x_1 + y_1, x_2 + y_2) \\
    & = (x_1 + y_1 - (x_2 + y_2), x_2 + y_2 - (x_1 + y_1)) \\
    & = (x_1 - x_2 + y_1 - y_2, x_2 - x_1 + y_2 - y_1) \\
    & = (x_1 - x_2, x_2 - x_1) + (y_1 - y_2, y_2 - y_1) \\
    & = f(x) + f(y)
\end{align*}
\begin{align*}
    f(\lambda x) & = f(\lambda x_1, \lambda x_2) \\
    & = (\lambda x_2 - \lambda x_1, \lambda x_1 - \lambda x_2) \\
    & = \lambda (x_2 - x_1, x_1 - x_2) \\
    & = \lambda f(x)
\end{align*}
\end{proof}

\begin{example}
\begin{align*}
    f: \mathbb{R}^2 & \longrightarrow \mathbb{R}^2 \\
    (x,y) & \longmapsto (2x + y, x)
\end{align*}
is a linear map.
\end{example}

\begin{example}
The derivative of polynomials map is a linear map, $f: P_n \rightarrow P_{n-1}$, $T(f(x)) = f'(x)$
\end{example}

\begin{example}
The definite integral from $a$ to $b$ defines a linear map from the set of continuous functions to the real numbers
\end{example}

\section*{Rotation Transformation}
\begin{theorem}
The \textbf{rotation transformation} by $\theta$ is the linear map $f_{\theta}: \mathbb{R}^2 \rightarrow \mathbb{R}^2$, $f(x,y) = (x \cos{\theta} - y \sin{\theta}, x \sin{\theta} + y \cos{\theta})$
\end{theorem}

\section*{Reflection Transformation}
\begin{example}
The \textbf{reflection about the x-axis} is the linear map $f: \mathbb{R}^2 \rightarrow \mathbb{R}^2$, $f(x,y) = f(x,-y)$
\end{example}

\section*{Projection Transformation}
\begin{example}
The \textbf{projection on the axis} is the linear map $f: \mathbb{R}^2 \rightarrow \mathbb{R}^2$, $f(x,y) = (x,0)$
\end{example}

\section*{Set of All Homomorphisms Between Two Vector Spaces}
\begin{definition}
Let $V$, $W$ be vector spaces over $\mathbb{F}$. The set of all linear maps from $V$ to $W$ is $\operatorname{Hom}(V, W)$.
\end{definition}

\begin{corollary}
$\operatorname{Hom}(V, W)$, with operations $+$ and $\cdot$ defined in the obvious way, is a vector space over $\mathbb{F}$.
\end{corollary}

\section*{Types of Linear Maps}
\begin{definition}
A linear map $f: V \rightarrow W$ is a
\begin{itemize}
    \item \textbf{monomorphism} if $f$ is injective
    \item \textbf{epimorphism} if $f$ is surjective
    \item \textbf{isomorphism} if $f$ is bijective
    \item \textbf{endomorphism} (or is a linear \textbf{operator}) if $V = W$, i.e. $f: V \rightarrow V$ maps to and from the same vector space
    \item \textbf{automorphism} if $f$ is bijective and $V = W$, i.e. $f$ is an isomorphism and an endomorphism
\end{itemize}
\end{definition}

\section*{Composition of Linear Maps is Linear}
\begin{theorem}
Let $V$, $W$, $Y$ be vector spaces over $\mathbb{F}$, $f: V \rightarrow W$, $g: W \rightarrow Y$ are linear maps. Then, their composition $g \circ f: V \rightarrow Y$ is a linear map.
\end{theorem}
\begin{proof}
\begin{align*}
    (g \circ f)(x + y) & = g(f(x + y)) \\
    & = g(f(x) + f(y)) && \text{as $f$ is linear} \\
    & = g(f(x)) + g(f(y)) && \text{as $g$ is linear} \\
    & = (g \circ f)(x) + (g \circ f)(y)
\end{align*}
Let $\lambda \in \mathbb{F}$. Then,
\begin{align*}
    (g \circ f)(\lambda x) & = g(f(\lambda x)) \\
    & = g(\lambda f(x)) && \text{as $f$ is linear} \\
    & = \lambda g(f(x)) && \text{as $g$ is linear} \\
    & = \lambda (g \circ f)(x)
\end{align*}
\end{proof}

\begin{theorem}
Let $V$ be a vector space over $\mathbb{F}$, $f$, $g$, $h \in \operatorname{Hom}(V,V)$, $a \in \mathbb{F}$. Then,
\begin{enumerate}
    \item $f \circ (g + h) = f \circ g + f \circ h$ and $(g + h) \circ f = g \circ f + h \circ f$
    \item $f \circ (g \circ h) = (f \circ g) \circ h$
    \item $f \circ Id_{V} = Id_{V} \circ f = f$
    \item $a(f \circ g) = (af) \circ g = f \circ (ag)$
\end{enumerate}
\end{theorem}

\section*{Inverse of a Linear Map is Linear}
\begin{theorem}
Let $V$ be a vector space over $\mathbb{F}$, $f: V \rightarrow W$ be a linear map. If $f$ is an isomorphism, then $f^{-1}: W \rightarrow V$ is also an isomorphism.
\end{theorem}
\begin{proof}
Let $x$, $y \in W$. Then,
\begin{align*}
    f^{-1}(x + y) & = f^{-1}(f(f^{-1}(x)) + f(f^{-1}(y))) \\
    & = f^{-1}(f(f^{-1}(x) + f^{-1}(y))) && \text{as $f$ is linear} \\
    & = f^{-1}(x) + f^{-1}(y)
\end{align*}
Let $\lambda \in \mathbb{F}$. Then,
\begin{align*}
    f^{-1}(\lambda x) & = f^{-1}(\lambda f(f^{-1}(x))) \\
    & = f^{-1}(f(\lambda f^{-1}(x))) && \text{as $f$ is linear} \\
    & = \lambda f^{-1}(x)
\end{align*}
Thus, $f^{-1}$ is linear. Also, since $f$ is bijective, $f^{-1}$ is bijective, and so $f^{-1}$ is an isomorphism.
\end{proof}

\section*{Image and Kernel}
Let $V$, $W$ be vector spaces over $\mathbb{F}$, $f: V \rightarrow W$ be a linear map.
\begin{definition}
The \textbf{image} of $f$, $\Image{f}$, is the set of $w \in W$ that $f$ maps from $V$
\begin{equation*}
    \Image{f} = \set{f(x) \in W: x \in V}
\end{equation*}
\begin{itemize}
    \item $f$ is surjective if and only if $\Image{f} = W$
\end{itemize}
\end{definition}

\begin{definition}
The \textbf{kernel} of $f$, $\Ker{f}$, is the set of all vectors that $f$ maps to the zero vector.
\begin{equation*}
    \Ker{f} = \set{x \in V: f(x) = 0}
\end{equation*}
\begin{itemize}
    \item A kernel is \textbf{trivial} if $\Ker{f} = \set{0}$
\end{itemize}
\end{definition}

\section*{Image is a Subspace}
\begin{theorem}
$\Image{f}$ is a subspace of $V$
\end{theorem}
\begin{proof}
Let $w_1$, $w_2 \in \Image{f} \subset W$. Then, $\exists v_1$, $v_2 \in V$ such that $w_1 = f(v_1)$ and $w_2 = f(v_2)$
\begin{align*}
    w_1 + w_2 = f(v_1) + f(v_2) = f(v_1 + v_2)
\end{align*}
Since $v_1 + v_2 \in V$, $f(v_1 + v_2) \in W$.
\\ \\ Let $\lambda \in \mathbb{F}$. Then,
\begin{align*}
    \lambda w_1 = \lambda f(v_1) = f(\lambda v_1)
\end{align*}
Since $\lambda v_1 \in V$, $f(\lambda v_1) \in W$.
\end{proof}

\section*{Kernel is a Subspace}
\begin{theorem}
$\Ker{f}$ is a subspace of $V$
\end{theorem}
\begin{proof}
Let $x$, $y \in \Ker{f}$. Then,
\begin{align*}
    f(x + y) & = f(x) + f(y) && \text{as $f$ is linear} \\
    & = 0 + 0 && \text{as $x$, $y \in \Ker{f}$} \\
    & = 0
\end{align*}
Thus, $x + y \in \Ker{f}$. Let $\lambda \in \mathbb{F}$.
\begin{align*}
    f(\lambda x) & = \lambda f(x) && \text{as $f$ is linear} \\
    & = \lambda \cdot 0 \\
    & = 0
\end{align*}
Thus, $\lambda x \in \Ker{f}$.
\end{proof}

\section*{Injective if and only if Trivial Kernel}
\begin{theorem}
Let $f: V \rightarrow W$ be a linear map. Then, $f$ is injective if and only if $\Ker{f} = \set{0}$
\end{theorem}
\begin{proof}
Let $\Ker{f} = \set{0}$, $x$, $y \in V$ with $f(x) = f(y)$. Then,
\begin{align*}
    f(x) - f(y) & = 0 \\
    f(x - y) & = 0
\end{align*}
Thus, $x - y \in \Ker{f}$, so $x - y = 0$, and $x = y$.
\\ \\ Conversely, let $f$ be injective. $f$ is linear, so $0 \in \Ker{f}$. Let $x \in \Ker{f}$. Then, $0 = f(x) = f(0)$, and since $f$ is injective, $x = 0$.
\end{proof}



\end{document}