\documentclass[letterpaper,12pt]{article}
\newcommand{\myname}{Cameron Geisler}
\newcommand{\mynumber}{90856741}
\usepackage{amsmath, amsfonts, amssymb, amsthm}
\usepackage[paper=letterpaper,left=25mm,right=25mm,top=3cm,bottom=25mm]{geometry}
\usepackage{fancyhdr}
\usepackage{float}
\usepackage{siunitx}
\usepackage{caption}
\usepackage{graphicx}
\pagestyle{fancy}
\usepackage{tkz-euclide} \usetkzobj{all} %% figures
\usepackage{hyperref} %% for links
\usepackage{exsheets} %% for tasks
\usepackage{systeme} %% for linear systems
\graphicspath{{../images/}} %% graphics in images folder

\lhead{Math 223} \chead{} \rhead{\myname}
\lfoot{} \cfoot{Page \thepage} \rfoot{}
\renewcommand{\headrulewidth}{0.4pt}
\renewcommand{\footrulewidth}{0.4pt}

\setlength{\parindent}{0pt}
\usepackage{enumerate}
\theoremstyle{definition}
\newtheorem*{definition}{Definition}
\newtheorem*{theorem}{Theorem}
\newtheorem*{example}{Example}
\newtheorem*{corollary}{Corollary}
\newtheorem*{lemma}{Lemma}
\newtheorem*{result}{Result}

%% Math
\newcommand{\abs}[1]{\left\lvert #1 \right\rvert}
\newcommand{\set}[1]{\left\{ #1 \right\}}
\renewcommand{\neg}{\sim}
\newcommand{\brac}[1]{\left( #1 \right)}
\newcommand{\eval}[1]{\left. #1 \right|}
\renewcommand{\vec}[1]{\mathbf{#1}}
\newenvironment{amatrix}[1]{\left[\begin{array}{@{}*{#1}{c}|c@{}}}{\end{array}\right]} %% for augmented matrix

\newcommand{\vecii}[2]{\left< #1, #2 \right>}
\newcommand{\veciii}[3]{\left< #1, #2, #3 \right>}

%% Linear algebra
\DeclareMathOperator{\Ker}{Ker}
\DeclareMathOperator{\nullity}{nullity}
\DeclareMathOperator{\Image}{Im}
\newcommand{\Span}[1]{\text{Span}\left(#1 \right)}
\DeclareMathOperator{\rank}{rank}
\DeclareMathOperator{\colrk}{colrk}
\DeclareMathOperator{\rowrk}{rowrk}
\DeclareMathOperator{\Row}{Row}
\DeclareMathOperator{\Col}{Col}
\DeclareMathOperator{\Null}{N}
\newcommand{\tr}[1]{tr\left( #1 \right)}
\DeclareMathOperator{\matref}{ref}
\DeclareMathOperator{\matrref}{rref}
\DeclareMathOperator{\sol}{Sol}
\newcommand{\inp}[2]{\left< #1, #2 \right>}
\newcommand{\norm}[1]{\left\lVert #1 \right\rVert}

%% Statistics
\newcommand{\prob}[1]{P\left( #1 \right)}
\newcommand{\overbar}[1]{\mkern 1.5mu \overline {\mkern-1.5mu#1 \mkern-1.5mu} \mkern 1.5mu}


\renewcommand{\frame}[1]{\tilde{\underline{\vec{#1}}}}

\chead{Introduction to Linear Transformations}

\begin{document}

A matrix equation can arise by considering a matrix $A$ as an object that ``acts" on a vector $\vec{x}$ to produce a new vector $A\vec{x}$. In other words, matrices can be thought of as a map or function from one set of vectors to another. In particular, a map from $\mathbb{R}^n$ to $\mathbb{R}^m$. This is a generalization of a real-valued function.

\begin{definition}
A \textbf{transformation} (or \textbf{mapping}) $T$ from $\mathbb{R}^n$ to $\mathbb{R}^m$ is a rule which assigns to each vector $\vec{x} \in \mathbb{R}^n$ a vector $T(\vec{x})$ in $\mathbb{R}^m$.
\begin{itemize}
    \item The set $\mathbb{R}^n$ is called the \textbf{domain} of $T$ (more generally, a subset of $\mathbb{R}^n$, denoted by $D(T)$), and $\mathbb{R}^m$ is called the \textbf{codomain} of $T$.
    \item For $\vec{x} \in \mathbb{R}^n$, the vector $T(\vec{x})$ is called the \textbf{image} of $\vec{x}$.
    \item The \textbf{range} of $T$ is the set of all images $T(\vec{x})$.
\end{itemize}
\end{definition}

\section*{Matrix Transformations}
If $A$ is an $m \times n$ matrix, then matrix multiplication by $A$ is a transformation. In particular, the map $T(\vec{x}) = A\vec{x}$ is a transformation from $\mathbb{R}^n$ to $\mathbb{R}^m$.

\section*{Linear Transformations}
Recall that if $A$ is an $m \times n$ matrix, and $\vec{u}, \vec{v} \in \mathbb{R}^n$, then
\begin{equation*}
    A(\vec{u} + \vec{v}) = A\vec{u} + A\vec{v} \quad \text{and} \quad A(c\vec{u}) = cA\vec{u}
\end{equation*}

These properties together make matrix multiplication a ``linear" transformation, and are one of the most important properties in linear algebra.

\begin{definition}
A transformation $T$ is \textbf{linear} if
\begin{enumerate}[(a)]
    \item $T(\vec{u} + \vec{v}) = T(\vec{u}) + T(\vec{v})$ for all $\vec{u}, \vec{v} \in D(T)$.
    \item $T(c\vec{u}) = cT(\vec{u})$ for all $c \in \mathbb{R}, \vec{u} \in D(T)$.
\end{enumerate}
\end{definition}

With this definition, every matrix transformation is a linear transformation. Linear transformations are said to be \textbf{operation preserving}, in particular preserving the operations of addition and scalar multiplication. This is because for a linear map, adding $\vec{u} + \vec{v}$ in $\mathbb{R}^n$ and then applying $T$ is equivalent to applying $T$ to $\vec{u}$ and $\vec{v}$ and then adding their result in $\mathbb{R}^m$.

\begin{theorem}
\textbf{Maps 0 to 0}. If $T$ is a linear transformation, then $T(\vec{0}) = \vec{0}$.
\end{theorem}
\begin{proof}
By linearity, $T(\vec{0}) = T(0\vec{u}) = 0 \cdot T(\vec{u}) = \vec{0}$.
\end{proof}

\begin{theorem}
\textbf{General linear property}. Let $T$ be a linear transformation, $\vec{v}_1, \dots, \vec{v}_n \in D(T)$, $a_1, \dots, a_n \in \mathbb{R}$. Then,
\begin{equation*}
    \boxed{T(a_1 \vec{v}_1 + \dots + a_n \vec{v}_n) = a_1 T(\vec{v}_1) + \dots + a_n T(\vec{v}_n)}
\end{equation*}
More compactly,
\begin{equation*}
    T\brac{\sum_{k=1}^n a_k \vec{v}_k} = \sum_{k=1}^n a_k T(\vec{v}_k)
\end{equation*}
\end{theorem}



\end{document}