\documentclass[letterpaper,12pt]{article}
\newcommand{\myname}{Cameron Geisler}
\newcommand{\mynumber}{90856741}
\usepackage{amsmath, amsfonts, amssymb, amsthm}
\usepackage[paper=letterpaper,left=25mm,right=25mm,top=3cm,bottom=25mm]{geometry}
\usepackage{fancyhdr}
\usepackage{float}
\usepackage{siunitx}
\usepackage{caption}
\usepackage{graphicx}
\pagestyle{fancy}
\usepackage{tkz-euclide} \usetkzobj{all} %% figures
\usepackage{hyperref} %% for links
\usepackage{exsheets} %% for tasks
\usepackage{systeme} %% for linear systems
\graphicspath{{../images/}} %% graphics in images folder

\lhead{Math 223} \chead{} \rhead{\myname}
\lfoot{} \cfoot{Page \thepage} \rfoot{}
\renewcommand{\headrulewidth}{0.4pt}
\renewcommand{\footrulewidth}{0.4pt}

\setlength{\parindent}{0pt}
\usepackage{enumerate}
\theoremstyle{definition}
\newtheorem*{definition}{Definition}
\newtheorem*{theorem}{Theorem}
\newtheorem*{example}{Example}
\newtheorem*{corollary}{Corollary}
\newtheorem*{lemma}{Lemma}
\newtheorem*{result}{Result}

%% Math
\newcommand{\abs}[1]{\left\lvert #1 \right\rvert}
\newcommand{\set}[1]{\left\{ #1 \right\}}
\renewcommand{\neg}{\sim}
\newcommand{\brac}[1]{\left( #1 \right)}
\newcommand{\eval}[1]{\left. #1 \right|}
\renewcommand{\vec}[1]{\mathbf{#1}}
\newenvironment{amatrix}[1]{\left[\begin{array}{@{}*{#1}{c}|c@{}}}{\end{array}\right]} %% for augmented matrix

\newcommand{\vecii}[2]{\left< #1, #2 \right>}
\newcommand{\veciii}[3]{\left< #1, #2, #3 \right>}

%% Linear algebra
\DeclareMathOperator{\Ker}{Ker}
\DeclareMathOperator{\nullity}{nullity}
\DeclareMathOperator{\Image}{Im}
\newcommand{\Span}[1]{\text{Span}\left(#1 \right)}
\DeclareMathOperator{\rank}{rank}
\DeclareMathOperator{\colrk}{colrk}
\DeclareMathOperator{\rowrk}{rowrk}
\DeclareMathOperator{\Row}{Row}
\DeclareMathOperator{\Col}{Col}
\DeclareMathOperator{\Null}{N}
\newcommand{\tr}[1]{tr\left( #1 \right)}
\DeclareMathOperator{\matref}{ref}
\DeclareMathOperator{\matrref}{rref}
\DeclareMathOperator{\sol}{Sol}
\newcommand{\inp}[2]{\left< #1, #2 \right>}
\newcommand{\norm}[1]{\left\lVert #1 \right\rVert}

%% Statistics
\newcommand{\prob}[1]{P\left( #1 \right)}
\newcommand{\overbar}[1]{\mkern 1.5mu \overline {\mkern-1.5mu#1 \mkern-1.5mu} \mkern 1.5mu}


\renewcommand{\frame}[1]{\tilde{\underline{\vec{#1}}}}

\chead{Vector Equations}

\begin{document}

\begin{definition}
A matrix $A$ is \textbf{square} if it has the same number of rows and columns. In other words, $A$ is an $n \times n$ matrix.
\end{definition}

\section*{Vectors}

The term \textit{vector} is used widely in mathematics and physics. Here, a vector will simply mean an ordered list of real numbers. In linear algebra, we represent vectors most commonly as a \textbf{column matrix}, a matrix with dimensions $1 \times n$, i.e. are a matrix with 1 row and $n$ ``columns". Similarly, a \textbf{row vector} is a $m \times 1$ matrix, a matrix which has 1 columns and $m$ ``rows".
\\ \\ First, we will consider vectors with two entries. The set of all vectors with two entries is denoted by $\mathbb{R}^2$ (read ``r-squared" or simply ``r-two").

\begin{definition}
Two vectors in $\mathbb{R}^2$ are \textbf{equal} if their corresponding entries are equal.
\end{definition}

\begin{definition}
Let $\vec{u}, \vec{v}$ be vectors. Their \textbf{sum} $\vec{u} + \vec{v}$ is the vector obtained by adding the corresponding entries of $\vec{u}$ and $\vec{v}$.
\end{definition}

\begin{definition}
Let $\vec{u}$ be a vector, $c \in \mathbb{R}$. Then, the \textbf{scalar multiple} of $\vec{u}$ by $c$, $c\vec{u}$, is the vector obtained by multiply each entry of $\vec{u}$ by $c$.
\begin{itemize}
    \item The number $c \in \mathbb{R}$ is called a scalar.
\end{itemize}
\end{definition}

Often, it is simpler to write vectors horizontally rather than vertically, but it is necessary to keep the vertical convention. In this case, we \textit{identify} a column vector with an ordered tuple of numbers. For example,
\begin{equation*}
    \begin{bmatrix} 1 \\ -3 \end{bmatrix} \quad \text{is equivalent to} \quad (1,-3)
\end{equation*}
More generally,
\begin{equation*}
    (b_1, \dots, b_n) \quad \text{is equivalent to} \quad \begin{bmatrix} b_1 \\ \vdots \\ b_n \end{bmatrix}
\end{equation*}
Notice that $(b_1, \dots, b_n)$ is not a row vector.

\section*{Geometry of Vectors in $\mathbb{R}^2$}
Every point in the plane is completely determined by an ordered pair of numbers, so there is a one-to-one correspondence between points in the plane $(a,b)$ and vectors $\begin{bmatrix} a \\ b \end{bmatrix}$.

\section*{Vectors in $\mathbb{R}^3$}
Vectors in $\mathbb{R}^3$ are $3 \times 1$ column matrices, and can be represented geometrically as points in space.

\section*{Vectors in $\mathbb{R}^n$}
More generally,

\begin{definition}
The set of all ordered $n$-tuples of $n$ real numbers is denoted by $\mathbb{R}^n$ (read ``r-n").
\end{definition}

The definitions of equality, sum, and scalar multiple are analogous to that for $\mathbb{R}^2$ and $\mathbb{R}^3$,
\begin{itemize}
    \item Vectors $\vec{u}, \vec{v}$ in $\mathbb{R}^n$ are \textbf{equal} if their corresponding entries are equal.
    \item The \textbf{sum} $\vec{u} + \vec{v}$ is obtained by adding entries componentwise.
    \item The \textbf{scalar multiple} $c \vec{u}$ is obtained by multiplying every entry of $\vec{u}$ by $c$.
    \item The \textbf{zero vector}, denoted by $\vec{0}$, is the unique vector with entries which are all zeros.
\end{itemize}

\begin{theorem}
Let $\vec{u}, \vec{v}, \vec{w} \in \mathbb{R}^n$ be vectors, $c, d \in \mathbb{R}$ be scalars. Then,
\begin{enumerate}[(a)]
    \item \textbf{Addition commutative}, $\vec{u} + \vec{v} = \vec{v} + \vec{u}$.
    \item \textbf{Addition associative}, $(\vec{u} + \vec{v}) + \vec{w} = \vec{u} + (\vec{v} + \vec{w})$.
    \item \textbf{Additive identity}, $\vec{u} + \vec{0} = \vec{0} + \vec{u} = \vec{u}$.
    \item \textbf{Additive inverse}, $\vec{u} + (-1\vec{u}) = \vec{0}$.
    \item \textbf{Scalar multiplication associative}, $c(d\vec{u}) = (cd) \vec{u}$.
    \item \textbf{Multiplicative identity}, $1 \vec{u} = \vec{u}$.
    \item \textbf{Scalar multiplication distributes over addition}, $c(\vec{u} + \vec{v}) = c\vec{u} + c\vec{v}$.
    \item \textbf{Vectors distribute over addition}, $(c + d) \vec{u} = c\vec{u} + c\vec{v}$.
\end{enumerate}
\end{theorem}

For simplicity, $-1\vec{u}$ is written as $-\vec{u}$, and is called the additive inverse of $\vec{u}$. These properties all essentially follow from the corresponding properties of real numbers.

\section*{Linear Combinations of Vectors}
\begin{definition}
Let $\vec{v}_1, \dots, \vec{v}_n \in \mathbb{R}^n$ be vectors, $a_1, \dots, a_n$ be scalars. Then, the vector $\vec{v}$ given by
\begin{equation*}
    v = a_1 v_1 + \dots + a_n v_n = \sum_{i=1}^n a_i v_i
\end{equation*}
is a \textbf{linear combination} of $\vec{v}_1, \dots, \vec{v}_n \in \mathbb{R}^n$, with \textbf{weights} $a_1, \dots, a_n$.
\end{definition}

\begin{example}
Given any collection of vectors, the \textbf{trivial} linear combination is that with all weights being 0, i.e.
\begin{equation*}
    \vec{0} = 0 \vec{v}_1 + \dots + 0 \vec{v}_n
\end{equation*}
\end{example}

\section*{Vector Equations}
A \textbf{vector equation}, naturally, is an equation involving vectors.

\begin{example}
Let
\begin{equation*}
    \vec{a}_1 = \begin{bmatrix} 1 \\ -2 \\ -5 \end{bmatrix} \quad \vec{a}_2 = \begin{bmatrix} 2 \\ 5 \\ 6 \end{bmatrix} \quad \vec{b} = \begin{bmatrix} 7 \\ 4 \\ -3 \end{bmatrix}
\end{equation*}
and consider the equation,
\begin{equation*}
    x_1 \vec{a}_1 + x_2 \vec{a}_2 = \vec{b}
\end{equation*}
where the unknowns are $x_1, x_2$. Intuitively, this represents the question of whether $\vec{b}$ can be written as a linear combination of $\vec{a}_1$ and $\vec{a}_2$. In other words, if there exists weights $x_1, x_2$ such that the linear combination $x_1 \vec{a}_1 + x_2 \vec{a}_2$ is equal to $\vec{b}$. Using the properties of vectors,
\begin{align*}
    x_1 \begin{bmatrix} 1 \\ -2 \\ -5 \end{bmatrix} + x_2 \begin{bmatrix} 2 \\ 5 \\ 6 \end{bmatrix} & = \begin{bmatrix} 7 \\ 4 \\ -3 \end{bmatrix} \\
    \begin{bmatrix} x_1 \\ -2x_1 \\ -5x_1 \end{bmatrix} + \begin{bmatrix} 2x_2 \\ 5x_2 \\ 6x_2 \end{bmatrix} & = \begin{bmatrix} 7 \\ 4 \\ -3 \end{bmatrix} \\
    \begin{bmatrix} x_1 + 2x_2 \\ -2x_1 + 5x_2 \\ -5x_1 + 6x_2 \end{bmatrix} & = \begin{bmatrix} 7 \\ 4 \\ -3 \end{bmatrix}
\end{align*}
These two vectors are equal if and only if their corresponding entries are equal. In other words, if $x_1, x_2$ are a solution of the linear system,
\begin{equation*}
    \systeme{x_1 + 2x_2 = 7, -2x_1 + 5x_2 = 4, -5x_1 + 6x_2 = -3}
\end{equation*}
Using row reduction,
\begin{equation*}
    \begin{amatrix}{2} 1 & 2 & 7 \\ -2 & 5 & 4 \\ -5 & 6 & -3 \end{amatrix}
\end{equation*}
which has RREF,
\begin{equation*}
    \begin{amatrix}{2} 1 & 0 & 3 \\ 0 & 1 & 2 \\ 0 & 0 & 0 \end{amatrix}
\end{equation*}
and so the system is consistent with $x_1 = 3, x_2 = 2$. Thus,
\begin{equation*}
    3\vec{a}_1 + 2\vec{a}_2 = \vec{b}
\end{equation*}
\end{example}

More generally, to solve a vector equation,
\begin{equation*}
    x_1 \vec{a}_1 + \dots + x_n \vec{a}_n = \vec{b}
\end{equation*}
Form the associated system of linear equations in $x_1, \dots, x_n$, and form its augmented matrix, whose columns will be $\vec{a}_1, \dots, \vec{a}_n, \vec{b}$. This is denoted by,
\begin{equation*}
    \begin{bmatrix} \vec{a}_1 & \vec{a}_2 & \dots & \vec{a}_n & \vec{b} \end{bmatrix}
\end{equation*}
Intuitively, this is a row vector with entries which are themselves column vectors, forming a matrix. Then, use row reduction to determine the solution. In summary,

\begin{theorem}
\textbf{Vector equations as a linear system}. The vector equation,
\begin{equation*}
    x_1 \vec{a}_1 + \dots + x_n \vec{a}_n = \vec{b}
\end{equation*}
has the same solution set as the system of linear equations in $x_1, \dots, x_n$ which has augmented matrix given by,
\begin{equation*}
    \begin{bmatrix} \vec{a}_1 & \vec{a}_2 & \dots & \vec{a}_n & \vec{b} \end{bmatrix}
\end{equation*}
\end{theorem}

\section*{Span of Vectors}

\begin{definition}
Let $\vec{v}_1, \dots, \vec{v}_n \in \mathbb{R}^n$ be vectors. The \textbf{span} of $\vec{v}_1, \dots, \vec{v}_n$, denoted by $\Span{\vec{v}_1, \dots, \vec{v}_n}$, is the set of all linear combinations of $\vec{v}_1, \dots, \vec{v}_n$. In other words,
\begin{equation*}
    \Span{\vec{v}_1, \dots, \vec{v}_n} = \set{a_1 \vec{v}_1 + \dots a_n \vec{v}_n : a_1, \dots, a_n \in \mathbb{R}}
\end{equation*}
\end{definition}

By definition, a vector $\vec{v}$ is in $\Span{\vec{v}_1, \dots, \vec{v}_n}$ if and only if $\vec{v}$ can be written as a linear combination of $\vec{v}_1, \dots, \vec{v}_n$. From the previous result, this is true if and only if the linear system with augmented matrix
\begin{equation*}
    \begin{bmatrix} \vec{v}_1 & \dots & \vec{v}_n & \vec{v} \end{bmatrix}
\end{equation*}
has a solution.

\section*{Geometric Interpretation of Span}

If $\vec{v} \in \mathbb{R}^3$, then $\Span{\vec{v}}$ is precisely the set of all scalar multiples of $\vec{v}$. Equivalently, the set of points on the line in $\mathbb{R}^3$ through $\vec{v}$ and $\vec{0}$.
\\ \\ If $\vec{u}, \vec{v}$ are non-zero vectors, and $\vec{v}$ is not a multiple of $\vec{v}$, then $\Span{\vec{u}, \vec{v}}$ is the plane in $\mathbb{R}^3$ which contains $\vec{u}, \vec{v}$, and $\vec{0}$.

\end{document}