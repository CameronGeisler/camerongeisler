\documentclass[letterpaper,12pt]{article}
\newcommand{\myname}{Cameron Geisler}
\newcommand{\mynumber}{90856741}
\usepackage{amsmath, amsfonts, amssymb, amsthm}
\usepackage[paper=letterpaper,left=25mm,right=25mm,top=3cm,bottom=25mm]{geometry}
\usepackage{fancyhdr}
\usepackage{float}
\usepackage{siunitx}
\usepackage{caption}
\usepackage{graphicx}
\pagestyle{fancy}
\usepackage{tkz-euclide} \usetkzobj{all} %% figures
\usepackage{hyperref} %% for links
\usepackage{exsheets} %% for tasks
\usepackage{systeme} %% for linear systems
\graphicspath{{../images/}} %% graphics in images folder

\lhead{Math 223} \chead{} \rhead{\myname}
\lfoot{} \cfoot{Page \thepage} \rfoot{}
\renewcommand{\headrulewidth}{0.4pt}
\renewcommand{\footrulewidth}{0.4pt}

\setlength{\parindent}{0pt}
\usepackage{enumerate}
\theoremstyle{definition}
\newtheorem*{definition}{Definition}
\newtheorem*{theorem}{Theorem}
\newtheorem*{example}{Example}
\newtheorem*{corollary}{Corollary}
\newtheorem*{lemma}{Lemma}
\newtheorem*{result}{Result}

%% Math
\newcommand{\abs}[1]{\left\lvert #1 \right\rvert}
\newcommand{\set}[1]{\left\{ #1 \right\}}
\renewcommand{\neg}{\sim}
\newcommand{\brac}[1]{\left( #1 \right)}
\newcommand{\eval}[1]{\left. #1 \right|}
\renewcommand{\vec}[1]{\mathbf{#1}}
\newenvironment{amatrix}[1]{\left[\begin{array}{@{}*{#1}{c}|c@{}}}{\end{array}\right]} %% for augmented matrix

\newcommand{\vecii}[2]{\left< #1, #2 \right>}
\newcommand{\veciii}[3]{\left< #1, #2, #3 \right>}

%% Linear algebra
\DeclareMathOperator{\Ker}{Ker}
\DeclareMathOperator{\nullity}{nullity}
\DeclareMathOperator{\Image}{Im}
\newcommand{\Span}[1]{\text{Span}\left(#1 \right)}
\DeclareMathOperator{\rank}{rank}
\DeclareMathOperator{\colrk}{colrk}
\DeclareMathOperator{\rowrk}{rowrk}
\DeclareMathOperator{\Row}{Row}
\DeclareMathOperator{\Col}{Col}
\DeclareMathOperator{\Null}{N}
\newcommand{\tr}[1]{tr\left( #1 \right)}
\DeclareMathOperator{\matref}{ref}
\DeclareMathOperator{\matrref}{rref}
\DeclareMathOperator{\sol}{Sol}
\newcommand{\inp}[2]{\left< #1, #2 \right>}
\newcommand{\norm}[1]{\left\lVert #1 \right\rVert}

%% Statistics
\newcommand{\prob}[1]{P\left( #1 \right)}
\newcommand{\overbar}[1]{\mkern 1.5mu \overline {\mkern-1.5mu#1 \mkern-1.5mu} \mkern 1.5mu}


\renewcommand{\frame}[1]{\tilde{\underline{\vec{#1}}}}

\chead{Geometry Using Matrices}

\begin{document}

Points $(x,y)$ in the plane can be represented as a column matrix. A polygon can be represented by putting all of the column matrices for its vertices into a single matrix, called the \textbf{vertex matrix} or \textbf{coordinate matrix}.

\section*{Contraction/Expansion}
\begin{itemize}
    \item Horizontal contraction/expansion $\begin{bmatrix} k & 0 \\ 0 & 1 \end{bmatrix}$, contraction if $0 < k < 1$, expansion if $k > 1$.
    \item Vertical contraction/expansion, $\begin{bmatrix} 1 & 0 \\ 0 & k \end{bmatrix}$, contraction if $0 < k < 1$, expansion if $k > 1$.
\end{itemize}

\section*{Contraction/Dilation as Scalar Multiplication}
\begin{definition}
Let $T: \mathbb{R}^2 \rightarrow \mathbb{R}^2$ be a linear transformation defined by $T(\vec{x}) = r\vec{x}$ for $r > 0$. Then $T$ is a \textbf{contraction} if $0 \leq r \leq 1$ and a \textbf{dilation} if $r > 1$.
\end{definition}

\section*{Projection Transformations}


\section*{Reflection as Matrix Multiplication}
\begin{itemize}
    \item Reflection over the $x$-axis $\begin{bmatrix} 1 & 0 \\ 0 & -1 \end{bmatrix}$.
    \item Reflection over $y$-axis $\begin{bmatrix} -1 & 0 \\ 0 & 1 \end{bmatrix}$.
    \item Reflection over the line $y = x$, $\begin{bmatrix} 0 & 1 \\ 1 & 0 \end{bmatrix}$.
    \item Reflection over $y = -x$, $\begin{bmatrix} 0 & -1 \\ -1 & 0 \end{bmatrix}$.
\end{itemize}
Rotation matrices,
\begin{itemize}
    \item $\ang{90}$ ccw $\begin{bmatrix} 0 & -1 \\ 1 & 0 \end{bmatrix}$
    \item $\ang{180}$ ccw $\begin{bmatrix} -1 & 0 \\ 0 & -1 \end{bmatrix}$
    \item $\ang{270}$ ccw $\begin{bmatrix} 0 & 1 \\ -1 & 0 \end{bmatrix}$
\end{itemize}
Glide reflection is a reflection and a translation.

\section*{Rotation Transformations}

Let $T: \mathbb{R}^2 \rightarrow \mathbb{R}^2$ be a transformation which rotates each point in $\mathbb{R}^2$ about the origin by an angle $\theta$, where counterclockwise rotation is considered positive. Then, consider the images of the columns of $I_2$ under $T$. First, $\begin{bmatrix} 1 \\ 0 \end{bmatrix}$ represents the point $(1,0)$, and so rotates to $(\cos{\theta}, \sin{\theta})$ or $\begin{bmatrix} \cos{\theta} \\ \sin{\theta} \end{bmatrix}$. Also, $\begin{bmatrix} 0 \\ 1 \end{bmatrix}$ represents $(0,1)$ and becomes $(\cos(\theta + \pi/2), \sin(\theta + \pi/2)$, which is equivalent to $(-\sin{\theta}, \cos{\theta})$, or $\begin{bmatrix} -\sin{\theta} \\ \cos{\theta} \end{bmatrix}$. In summary,

\begin{definition}
The \textbf{rotation transformation} $T_{\theta}: \mathbb{R}^2 \rightarrow \mathbb{R}^2$ given by rotation by $\theta$ counterclockwise, has matrix
\begin{equation*}
    A = \begin{bmatrix} \cos{\theta} & -\sin{\theta} \\ \sin{\theta} & \cos{\theta} \end{bmatrix}
\end{equation*}
\end{definition}

\section*{Shear Transformations}
Shear transformations. Applications to physics, geology, and crystallography.

\begin{itemize}
    \item Horizontal shear, $\begin{bmatrix} 1 & k \\ 0 & 1 \end{bmatrix}$, right if $k > 0$ and left if $k < 0$.
\end{itemize}




\end{document}