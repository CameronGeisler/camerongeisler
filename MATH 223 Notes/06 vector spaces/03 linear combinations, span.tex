\documentclass[letterpaper,12pt]{article}
\newcommand{\myname}{Cameron Geisler}
\newcommand{\mynumber}{90856741}
\usepackage{amsmath, amsfonts, amssymb, amsthm}
\usepackage[paper=letterpaper,left=25mm,right=25mm,top=3cm,bottom=25mm]{geometry}
\usepackage{fancyhdr}
\usepackage{float}
\usepackage{siunitx}
\usepackage{caption}
\usepackage{graphicx}
\pagestyle{fancy}
\usepackage{tkz-euclide} \usetkzobj{all} %% figures
\usepackage{hyperref} %% for links
\usepackage{exsheets} %% for tasks
\usepackage{systeme} %% for linear systems
\graphicspath{{../images/}} %% graphics in images folder

\lhead{Math 223} \chead{} \rhead{\myname}
\lfoot{} \cfoot{Page \thepage} \rfoot{}
\renewcommand{\headrulewidth}{0.4pt}
\renewcommand{\footrulewidth}{0.4pt}

\setlength{\parindent}{0pt}
\usepackage{enumerate}
\theoremstyle{definition}
\newtheorem*{definition}{Definition}
\newtheorem*{theorem}{Theorem}
\newtheorem*{example}{Example}
\newtheorem*{corollary}{Corollary}
\newtheorem*{lemma}{Lemma}
\newtheorem*{result}{Result}

%% Math
\newcommand{\abs}[1]{\left\lvert #1 \right\rvert}
\newcommand{\set}[1]{\left\{ #1 \right\}}
\renewcommand{\neg}{\sim}
\newcommand{\brac}[1]{\left( #1 \right)}
\newcommand{\eval}[1]{\left. #1 \right|}
\renewcommand{\vec}[1]{\mathbf{#1}}
\newenvironment{amatrix}[1]{\left[\begin{array}{@{}*{#1}{c}|c@{}}}{\end{array}\right]} %% for augmented matrix

\newcommand{\vecii}[2]{\left< #1, #2 \right>}
\newcommand{\veciii}[3]{\left< #1, #2, #3 \right>}

%% Linear algebra
\DeclareMathOperator{\Ker}{Ker}
\DeclareMathOperator{\nullity}{nullity}
\DeclareMathOperator{\Image}{Im}
\newcommand{\Span}[1]{\text{Span}\left(#1 \right)}
\DeclareMathOperator{\rank}{rank}
\DeclareMathOperator{\colrk}{colrk}
\DeclareMathOperator{\rowrk}{rowrk}
\DeclareMathOperator{\Row}{Row}
\DeclareMathOperator{\Col}{Col}
\DeclareMathOperator{\Null}{N}
\newcommand{\tr}[1]{tr\left( #1 \right)}
\DeclareMathOperator{\matref}{ref}
\DeclareMathOperator{\matrref}{rref}
\DeclareMathOperator{\sol}{Sol}
\newcommand{\inp}[2]{\left< #1, #2 \right>}
\newcommand{\norm}[1]{\left\lVert #1 \right\rVert}

%% Statistics
\newcommand{\prob}[1]{P\left( #1 \right)}
\newcommand{\overbar}[1]{\mkern 1.5mu \overline {\mkern-1.5mu#1 \mkern-1.5mu} \mkern 1.5mu}


\renewcommand{\frame}[1]{\tilde{\underline{\vec{#1}}}}

\chead{Linear Combinations}

\begin{document}

\section*{Family of Vectors}
\begin{definition}
A \textbf{family} (or \textbf{indexed family}, or \textbf{indexed set}) is an $n$-tuple of elements, with an index mapped to each element.
\begin{itemize}
    \item A family of vectors $v_1, \dots, v_n \in V$ is $(v_1, \dots, v_n)$
    \item A family of scalars $a_1, \dots, a_n \in \mathbb{F}$
\end{itemize}
\end{definition}

\section*{Linear Combinations}
\begin{definition}
Let $V$ be a vector space over $\mathbb{F}$. A vector $v \in V$ is a \textbf{linear combination} of vectors $v_1, \dots, v_n \in V$ if there exists $a_1, \dots, a_n \in \mathbb{F}$ such that
\begin{equation*}
    v = a_1 v_1 + \dots + a_n v_n = \sum_{i=1}^n a_i v_i
\end{equation*}
\begin{itemize}
    \item $a_1, \dots, a_n$ are the called the \textbf{coefficients} of the linear combination
    \item By convention, the zero vector $0$ is a linear combination of the empty family of vectors.
\end{itemize}
\end{definition}

\section*{Span of a Set of Vectors}
\begin{definition}
Let $V$ be a vector space over $\mathbb{F}$, $S = \set{v_1, \dots, v_n} \subseteq V$. The \textbf{span} of $S$, $\Span{S}$, is the set of all linear combinations of the vectors in $S$,
\begin{equation*}
    \boxed{\Span{S} = \Span{v_1, \dots, v_n} = \set{a_1 v_1 + \dots a_n v_n \mid a_i \in \mathbb{F}}}
\end{equation*}
\begin{itemize}
    \item By convention, $\Span{\emptyset} = \set{0}$
\end{itemize}
\end{definition}

\section*{Span is a Vector Subspace}
\begin{theorem}
The span $\Span{S}$ is a subspace of $V$. In particular, the sum of linear combinations of vectors in $V$ is a linear combination of the same vectors in $V$.
\end{theorem}
\begin{proof}
\begin{itemize}
    \item[]
    \item If $S = \emptyset$, then $\Span{S} = \set{0}$, which is a subspace of any vector space $V$.
    \item If $S \neq \emptyset$, let $x$, $y \in \Span{S}$. Then, there exists unique $a_1, \dots, a_n, b_1, \dots, b_n \in \mathbb{F}$ such that
    \begin{align*}
        x & = a_1 v_1 + \dots + a_n v_n = \sum_{i=1}^n a_i v_i \\
        y & = b_1 v_1 + \dots + b_n v_n = \sum_{i=1}^n b_i v_i
    \end{align*}
    Then,
    \begin{align*}
        x + y & = \sum_{i=1}^n a_i v_i + \sum_{i=1}^n b_i v_i = \sum_{i=1}^n (a_i + b_i) v_i
    \end{align*}
    which is a linear combination of $v_1, \dots, v_n$, so $x + y \in \Span{S}$. Then, let $\lambda \in \mathbb{F}$. Then,
    \begin{align*}
        \lambda x = \lambda \sum_{i=1}^n a_i v_i = \sum_{i=1}^n (\lambda a_i) v_i
    \end{align*}
    which is a linear combination of $v_1, \dots, v_n$, so $\lambda x \in \Span{S}$.
\end{itemize}
\end{proof}

\begin{theorem}
Further, $\Span{S}$ is the smallest subspace of $V$ which contains $S$.
\end{theorem}
\begin{proof}

\end{proof}

\begin{definition}
Let $V$ be a vector space, $S \subseteq V$. $S$ \textbf{spans} (or \textbf{generates}) $V$ if $V = \Span{S}$.
\begin{itemize}
    \item Alternatively, the vectors of $S$ spans $V$
\end{itemize}
\end{definition}

\end{document}