\documentclass[letterpaper,12pt]{article}
\newcommand{\myname}{Cameron Geisler}
\newcommand{\mynumber}{90856741}
\usepackage{amsmath, amsfonts, amssymb, amsthm}
\usepackage[paper=letterpaper,left=25mm,right=25mm,top=3cm,bottom=25mm]{geometry}
\usepackage{fancyhdr}
\usepackage{float}
\usepackage{siunitx}
\usepackage{caption}
\usepackage{graphicx}
\pagestyle{fancy}
\usepackage{tkz-euclide} \usetkzobj{all} %% figures
\usepackage{hyperref} %% for links
\usepackage{exsheets} %% for tasks
\usepackage{systeme} %% for linear systems
\graphicspath{{../images/}} %% graphics in images folder

\lhead{Math 223} \chead{} \rhead{\myname}
\lfoot{} \cfoot{Page \thepage} \rfoot{}
\renewcommand{\headrulewidth}{0.4pt}
\renewcommand{\footrulewidth}{0.4pt}

\setlength{\parindent}{0pt}
\usepackage{enumerate}
\theoremstyle{definition}
\newtheorem*{definition}{Definition}
\newtheorem*{theorem}{Theorem}
\newtheorem*{example}{Example}
\newtheorem*{corollary}{Corollary}
\newtheorem*{lemma}{Lemma}
\newtheorem*{result}{Result}

%% Math
\newcommand{\abs}[1]{\left\lvert #1 \right\rvert}
\newcommand{\set}[1]{\left\{ #1 \right\}}
\renewcommand{\neg}{\sim}
\newcommand{\brac}[1]{\left( #1 \right)}
\newcommand{\eval}[1]{\left. #1 \right|}
\renewcommand{\vec}[1]{\mathbf{#1}}
\newenvironment{amatrix}[1]{\left[\begin{array}{@{}*{#1}{c}|c@{}}}{\end{array}\right]} %% for augmented matrix

\newcommand{\vecii}[2]{\left< #1, #2 \right>}
\newcommand{\veciii}[3]{\left< #1, #2, #3 \right>}

%% Linear algebra
\DeclareMathOperator{\Ker}{Ker}
\DeclareMathOperator{\nullity}{nullity}
\DeclareMathOperator{\Image}{Im}
\newcommand{\Span}[1]{\text{Span}\left(#1 \right)}
\DeclareMathOperator{\rank}{rank}
\DeclareMathOperator{\colrk}{colrk}
\DeclareMathOperator{\rowrk}{rowrk}
\DeclareMathOperator{\Row}{Row}
\DeclareMathOperator{\Col}{Col}
\DeclareMathOperator{\Null}{N}
\newcommand{\tr}[1]{tr\left( #1 \right)}
\DeclareMathOperator{\matref}{ref}
\DeclareMathOperator{\matrref}{rref}
\DeclareMathOperator{\sol}{Sol}
\newcommand{\inp}[2]{\left< #1, #2 \right>}
\newcommand{\norm}[1]{\left\lVert #1 \right\rVert}

%% Statistics
\newcommand{\prob}[1]{P\left( #1 \right)}
\newcommand{\overbar}[1]{\mkern 1.5mu \overline {\mkern-1.5mu#1 \mkern-1.5mu} \mkern 1.5mu}


\renewcommand{\frame}[1]{\tilde{\underline{\vec{#1}}}}

\chead{Vector Subspaces}

\begin{document}

\section*{Vector Subspaces}
\begin{definition}
Let $V$ be a vector space over $\mathbb{F}$, $U \subseteq V$ be a subset. Then, $U$ is a \textbf{vector subspace} (or simply \textbf{subspace}) of $V$ if $U$ is a vector space over $\mathbb{F}$ under the operations $+$ and $\cdot$ inherited from $V$.
\end{definition}

\begin{theorem}
Let $V$ be a vector space, $U \subseteq V$. Then, $U$ is a subspace of $V$ if and only if,
\begin{enumerate}[(a)]
    \item \textbf{$U$ is non-empty, or $U$ contains the zero vector in $V$}: $0 \in U$. Since all vector spaces contain the zero vector, $U$ is non-empty if and only if $U$ contains the zero vector.
    \item \textbf{Closed under addition}. If $x, y \in U$, then $x + y \in U$.
    \item \textbf{Closed under scalar multiplication}. If $x \in U, \lambda \in \mathbb{F}$, then $\lambda x \in U$.
\end{enumerate}
\end{theorem}

\begin{proof}
EXERCISE.
\end{proof}

\begin{example}
For every vector space $V$, $V$ has two \textbf{trivial subspaces}. First, $V$ itself is a subspace of itself, and $V$ contains the \textbf{zero subspace} $U = \set{0}$.
\end{example}

\begin{theorem}
Let $x \in U$, $\lambda \in \mathbb{F}$. Then,
\begin{itemize}
    \item $-x \in U$
\end{itemize}
\end{theorem}

\begin{example}
The set of all $n \times n$ symmetric matrices is a subspace of the set of all $n \times n$ matrices, $M(n \times n, \mathbb{F})$.
\end{example}

\begin{example}
Let $n \in \mathbb{N} \cup \set{0}$. The set of all polynomials with degree less than or equal to $n$, $P_{n}(\mathbb{F})$, is a subspace of the set of all polynomials, $P(\mathbb{F})$.
\end{example}

\begin{example}
\textbf{Calculus}. Let
\begin{align*}
    V & = \text{set of all functions on $[a,b]$} \\
    V_1 & = \text{set of all integrable functions on $[a,b]$} \\
    V_2 & = \text{set of all continuous functions on $[a,b]$} \\
    V_3 & = \text{set of all differentiable functions on $[a,b]$} \\
    V_4 & = \text{set of all polynomial functions on $[a,b]$}
\end{align*}
Recall that all polynomials are differentiable, differentiable functions are continuous, continuous functions are integrable. Also, recall that $V$ is a vector space, and $V_1, V_2, V_3, V_4$ are vector subspaces of $V$. This forms a chain of subspaces,
\begin{equation*}
    V_4 \subseteq V_3 \subseteq V_2 \subseteq V_1 \subseteq V
\end{equation*}
\end{example}


\section*{Intersection of Subspaces is a Subspace}
\begin{theorem}
Let $V$ be a vector space over $\mathbb{F}$, $U_1$, $U_2 \subseteq V$ be subspaces. Then, $U_1 \cap U_2$ is a subspace of $V$.
\end{theorem}
\begin{proof}
First, since $U_1$ and $U_2$ are subspaces, $0 \in U_1$ and $0 \in U_2$, and so $0 \in U_1 \cap U_2$. Let $x$, $y \in U_1 \cap U_2$. Then, $x \in U_1$, $x \in U_2$, $y \in U_1$, and $y \in U_2$. Then, since $U_1$ and $U_2$ are subspaces, $x + y \in U_1$ and $x + y \in U_2$, and so $x + y \in U_1 \cap U_2$. Then, let $\lambda \in \mathbb{F}$. Then, $x \in U_1$ and $x \in U_2$. Since $U_1$ and $U_2$ are subspaces, $\lambda x \in U_1$ and $\lambda x \in U_2$, so $\lambda x \in U_1 \cap U_2$.
\end{proof}

\section*{Subspaces of $\mathbb{R}^n$}
\begin{example}
\textbf{Subspaces of $\mathbb{R}^2$}. Subspaces $U \subseteq \mathbb{R}^2$ are of one of the following forms,
\begin{enumerate}[(a)]
    \item \textbf{Trivial}, either $U = \set{(0,0)}$ is the single point (the origin) or $U = \mathbb{R}^2$ the entire plane.
    \item \textbf{Line through the origin}, $U$ consists of the set of all points on a line which passes through the origin.
\end{enumerate}
\end{example}

\begin{example}
\textbf{Subspaces of $\mathbb{R}^3$}. Subspaces $U \subseteq \mathbb{R}^3$ are of one of the following forms,
\begin{enumerate}[(a)]
    \item \textbf{Trivial}, either $U = \set{(0,0,0)}$ or $U = \mathbb{R}^3$.
    \item \textbf{Line through the origin}, $U$ consists of the set of all points on a line which passes through the origin.
    \item \textbf{Plane through the origin}, $U$ consists of the set of all points on a plane which passes through the origin.
\end{enumerate}
\end{example}

\section*{Sum of Subspaces is a Subspace}
The union of two subspace is, in general, not a subspace. Instead, we define the sum of subspaces.
\begin{definition}
Let $V$ be a vector space, $U_1$, $U_2 \subseteq V$ be subspaces. Then, the \textbf{sum} of $U_1$ and $U_2$ is the set
\begin{equation*}
    U_1 + U_2 = \set{x + y \mid x \in U_1, y \in U_2}
\end{equation*}
In other words, it is the set of all possible sums of elements, where one is from $U_1$ and the other from $U_2$.
\end{definition}

\begin{theorem}
Let $V$ be a vector space over $\mathbb{F}$, $U_1$, $U_2 \subseteq V$ be subspaces. Then, $U_2 + U_2$ is a subspace of $V$.
\end{theorem}
\begin{proof}
Let $U_1$, $U_2$ be subspaces of $V$. First, $0 \in U_1$ and $0 \in U_2$, so $0 + 0 = 0 \in U_1 + U_2$.
\\ \\ Let $x$, $y \in U_1 + U_2$. Then, for some $x_1, y_1 \in U_1$, $x_2, y_2 \in U_2$, we have $x = x_1 + x_2$ and $y = y_1 + y_2$. Thus,
\begin{equation*}
    x + y = (x_1 + y_1) + (x_2 + y_2)
\end{equation*}
Since $x_1 + y_1 \in U_1$ and $x_2 + y_2 \in U_2$, their sum $(x_1 + y_1) + (x_2 + y_2) \in U_1 + U_2$.
\\ \\ Let $\lambda \in \mathbb{F}$. Then,
\begin{equation*}
    \lambda x = \lambda (x_1 + y_1) = \lambda x_1 + \lambda y_1
\end{equation*}
Since $\lambda x_1 \in U_1$ and $\lambda y_2 \in U_2$, $\lambda x_1 + \lambda y_1 \in U_1 + U_2$.
\end{proof}

\section*{Union of Strict Subspaces is a Strict Subspace}
\begin{theorem}
Let $V$ be a vector space over $\mathbb{F}$, $U_1$, $U_2 \subseteq V$ be subspaces of $V$. Then, if $U_1 \cup U_2 = V$, then $U_1 = V$, $U_2 = V$, or both.
\begin{itemize}
    \item For example, for $V = \mathbb{R}^2$, the union of two lines through the origin cannot be $\mathbb{R}^2$. In order for the union of two subspaces of $\mathbb{R}^2$ to be equal to $\mathbb{R}^2$, at least one of them must be the entire plane $\mathbb{R}^2$.
\end{itemize}
\end{theorem}
\begin{proof}
By contradiction, suppose $U_1 \cup U_2 = V$ but $U_1 \neq V$ and $U_2 \neq V$. Then, there exists $x \in V \setminus U_1$, $y \in V \setminus U_2$. Since $U_1 \cup U_2 = V$, $x \in U_2$ and $y \in U_1$. Also, $x + y \in V$, so either $x + y \in U_1$ or $x + y \in U_2$.
\begin{itemize}
    \item If $x + y \in U_1$, then $x = (x + y) - y \in U_1$, which contradicts our assumption that $x \in V \setminus U_1$.
    \item If $x + y \in U_2$, then $y = (x + y) - x \in U_2$, which contradicts our assumption that $y \in V \setminus U_2$.
\end{itemize}
In either case, we get a contradiction.
\end{proof}

\section*{Direct Sum and Complementary Subspaces}
\begin{definition}
Let $V$ be a vector space, $U_1$, $U_2 \subseteq V$ be subspaces. $V$ is the \textbf{direct sum} of $U_1$ and $U_2$, $V = U_1 \oplus U_2$, if $U_1 + U_2 = V$ and $U_1 \cap U_2 = \set{0}$.
\begin{itemize}
    \item $U_1$ and $U_2$ are \textbf{complementary subspaces}.
\end{itemize}
\end{definition}

\end{document}