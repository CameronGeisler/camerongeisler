\documentclass[letterpaper,12pt]{article}
\newcommand{\myname}{Cameron Geisler}
\newcommand{\mynumber}{90856741}
\usepackage{amsmath, amsfonts, amssymb, amsthm}
\usepackage[paper=letterpaper,left=25mm,right=25mm,top=3cm,bottom=25mm]{geometry}
\usepackage{fancyhdr}
\usepackage{float}
\usepackage{siunitx}
\usepackage{caption}
\usepackage{graphicx}
\pagestyle{fancy}
\usepackage{tkz-euclide} \usetkzobj{all} %% figures
\usepackage{hyperref} %% for links
\usepackage{exsheets} %% for tasks
\usepackage{systeme} %% for linear systems
\graphicspath{{../images/}} %% graphics in images folder

\lhead{Math 223} \chead{} \rhead{\myname}
\lfoot{} \cfoot{Page \thepage} \rfoot{}
\renewcommand{\headrulewidth}{0.4pt}
\renewcommand{\footrulewidth}{0.4pt}

\setlength{\parindent}{0pt}
\usepackage{enumerate}
\theoremstyle{definition}
\newtheorem*{definition}{Definition}
\newtheorem*{theorem}{Theorem}
\newtheorem*{example}{Example}
\newtheorem*{corollary}{Corollary}
\newtheorem*{lemma}{Lemma}
\newtheorem*{result}{Result}

%% Math
\newcommand{\abs}[1]{\left\lvert #1 \right\rvert}
\newcommand{\set}[1]{\left\{ #1 \right\}}
\renewcommand{\neg}{\sim}
\newcommand{\brac}[1]{\left( #1 \right)}
\newcommand{\eval}[1]{\left. #1 \right|}
\renewcommand{\vec}[1]{\mathbf{#1}}
\newenvironment{amatrix}[1]{\left[\begin{array}{@{}*{#1}{c}|c@{}}}{\end{array}\right]} %% for augmented matrix

\newcommand{\vecii}[2]{\left< #1, #2 \right>}
\newcommand{\veciii}[3]{\left< #1, #2, #3 \right>}

%% Linear algebra
\DeclareMathOperator{\Ker}{Ker}
\DeclareMathOperator{\nullity}{nullity}
\DeclareMathOperator{\Image}{Im}
\newcommand{\Span}[1]{\text{Span}\left(#1 \right)}
\DeclareMathOperator{\rank}{rank}
\DeclareMathOperator{\colrk}{colrk}
\DeclareMathOperator{\rowrk}{rowrk}
\DeclareMathOperator{\Row}{Row}
\DeclareMathOperator{\Col}{Col}
\DeclareMathOperator{\Null}{N}
\newcommand{\tr}[1]{tr\left( #1 \right)}
\DeclareMathOperator{\matref}{ref}
\DeclareMathOperator{\matrref}{rref}
\DeclareMathOperator{\sol}{Sol}
\newcommand{\inp}[2]{\left< #1, #2 \right>}
\newcommand{\norm}[1]{\left\lVert #1 \right\rVert}

%% Statistics
\newcommand{\prob}[1]{P\left( #1 \right)}
\newcommand{\overbar}[1]{\mkern 1.5mu \overline {\mkern-1.5mu#1 \mkern-1.5mu} \mkern 1.5mu}


\renewcommand{\frame}[1]{\tilde{\underline{\vec{#1}}}}

\chead{Basis and Dimension}

%% Statistics
\newcommand{\prob}[1]{P\left( #1 \right)}
\newcommand{\overbar}[1]{\mkern 1.5mu \overline {\mkern-1.5mu#1 \mkern-1.5mu} \mkern 1.5mu}

\begin{document}

A basis is intuitively analogous to units of measurement.

\section*{Basis of a Vector Space}
\begin{definition}
Let $V$ be a vector space over $\mathbb{F}$, $B = \set{v_1, \dots, v_n} \subseteq V$. Then, $B$ is a \textbf{basis} of $V$ if
\begin{enumerate}[(a)]
    \item $B$ is linearly independent.
    \item $B$ spans $V$.
\end{enumerate}
\begin{itemize}
    \item The plural of basis is \textbf{bases}.
\end{itemize}
\end{definition}
A basis is a \textbf{maximal, linearly independent set}, in that adding any other vector in $V$ to the basis would make it linearly dependent, i.e. it is the largest possible linearly independent set, in that any other linearly independent set has a size less than or equal to the size of the basis. A basis is also a \textbf{minimally spanning set}, in that while a basis spans $V$, any strict subset of a basis does not span $V$, i.e. it is the smallest possible spanning set.
\\ \\ Intuitively, a basis has enough vectors to span $V$, but not so many that one vector can be written as a linear combination of the others.
\\ \\ In fact, every vector space has a basis. However, some vector spaces have bases which are infinite, i.e. they are made up of infinitely many elements.
\begin{itemize}
    \item If a basis $B$ of $V$ has a finite number of vectors, it is a \textbf{finite} basis.
\end{itemize}

\begin{definition}
In $\mathbb{F}^n$, the \textbf{standard basis} is the set $\set{e_1, \dots, e_n}$, where $e_1 = (1, 0, \dots, 0), e_2 = (0, 1, 0, \dots, 0), \dots, e_n = (0, 0, \dots, 0, 1)$.
\end{definition}

\begin{definition}
In $P_{n}(\mathbb{F})$, the \textbf{standard basis} is the set $\set{1, x, x^2, \dots, x^n}$.
\end{definition}

\section*{Vector Uniquely Expressed in Terms of Basis}
\begin{theorem}
Let $V$ be a vector space over $\mathbb{F}$, $B = \set{v_1, \dots, v_n} \subseteq V$. Then, $B$ is a basis of $V$ if and only if for all $v \in V$, there exists unique $a_1, \dots, a_n \in \mathbb{F}$ such that
\begin{equation*}
    v = a_1 v_1 + \dots + a_n v_n
\end{equation*}
\end{theorem}
\begin{proof}
Let $B$ be a basis of $V$. To prove existence, since $B$ is a basis, by definition, there exists $a_1, \dots, a_n \in \mathbb{F}$ such that $v = a_1 v_1 + \dots + a_n v_n$. To prove uniqueness, by contradiction, suppose there exists $a_1, \dots, a_n, b_1, \dots, b_n$ such that $v = a_1 v_1 + \dots + a_n v_n$ and $v = b_1 v_1 + \dots + b_n v_n$. Then,
\begin{align*}
    0 & = v - v \\
    & = (a_1 v_1 + \dots + a_n v_n) - (b_1 v_1 + \dots + b_n v_n) \\
    & = (a_1 - b_1) v_1 + \dots + (a_n - b_n) v_n
\end{align*}
Since $\set{v_1, \dots, v_n}$ is a basis of $V$, it is linearly independent, so this linear combination is a trivial combination, and so $a_1 - b_1 = \dots = a_n - b_n = 0$. Thus, $a_1 = b_1, \dots, a_n = b_n$.
\end{proof}

\section*{Spanning Set Can be Reduced to Basis}
\begin{theorem}
Let $V$ be a vector space, $S \subseteq V$ be finite. If $S$ spans $V$, then there exists a (finite) subset of $S$, $B \subseteq S$, such that $B$ is a basis of $V$.
\begin{itemize}
    \item Intuitively, a spanning set for $V$ can be reduced to a basis of $V$
\end{itemize}
\end{theorem}

\section*{Steinitz Exchange Lemma}
\begin{theorem}
Let $V$ be a vector space, $B = \set{v_1, \dots, v_n} \subseteq V$ be a spanning set for $V$, $S = \set{w_1, \dots, w_m} \subseteq V$ be linearly independent. Then, $m \leq n$, and there exists a subset $B' \subseteq B$ of $n - m$ vectors such that $S \cup B'$ spans $V$.
\begin{itemize}
    \item In other words, we can suitably add $n - m$ vectors from $B$ to $S$ to form a basis $\set{w_1, \dots, w_m, v_{m+1}, \dots, v_n}$ of $V$.
\end{itemize}
\end{theorem}

\section*{Replacement Theorem (Exchange Theorem)}
\begin{theorem}

\end{theorem}

\section*{All Bases Have Equal Length}
\begin{theorem}
Let $V$ be a vector space, $B_1 = \set{v_1, \dots, v_n}$, $B_2 = \set{w_1, \dots, w_m}$ be bases of $V$. Then, $n = m$.
\begin{itemize}
    \item In other words, every basis for $V$ has the same number of elements
\end{itemize}
\end{theorem}
\begin{proof}
By contradiction, suppose $m > n$. Then, let $S = \set{w_1, \dots, w_n, w_{n+1}} \subseteq B_2$. Since $B_2$ is a basis, it is linearly independent, and since $S \subseteq B_2$, $S$ is linearly independent. Then, since $B_1$ spans $V$ and $S$ is a linearly independent subset of $V$, by the (exchange lemma), $n + 1 \leq n$, which is a contradiction. Thus, $m \leq n$.
\\ \\ Then, by contradiction, suppose $n < m$. Then, let $S = \set{v_1, \dots, v_m, v_{m+1}} \subseteq B_1$. Since $B_1$ is a basis, it is linearly independent, and since $S \subseteq B_1$, $S$ is linearly independent. Then, since $B_2$ spans $V$ and $S$ is a linearly independent subset of $V$, by the (exchange lemma), $m + 1 \leq m$, which is a contradiction. Thus, $n \geq m$.
\\ \\ Thus, $m \leq n$ and $m \geq n$, so $m = n$.
\end{proof}

\section*{Dimension}
\begin{definition}
Let $V$ be a vector space. $V$ is \textbf{finite-dimensional} if and only if it has a finite basis.
\begin{itemize}
    \item Otherwise, $V$ is \textbf{infinite-dimensional}, $\dim{V} = \infty$
\end{itemize}
\end{definition}

Infinite-dimensional vector spaces are much more complicated, and are beyond the scope.

\begin{definition}
Let $V$ be a vector space with a finite basis. The \textbf{dimension} of $V$, $\dim{V}$, is the unique length of a basis of $V$.
\end{definition}


\begin{example}
The vector space $\set{0}$ has dimension $0$, $\dim{\set{0}} = 0$.
\end{example}

\begin{example}
The vector space $\mathbb{F}^n$ has dimension $n$, $\dim{\mathbb{F}^n} = n$, as the standard basis of $\mathbb{F}^n$ has $n$ elements.
\end{example}

\begin{example}
The vector space $P_{n}(\mathbb{F})$ has dimension $n+1$.
\end{example}

\begin{example}
$\dim{(M_{m \times n}(\mathbb{F}))} = mn$.
\end{example}

\begin{example}
The vector space of all polynomials $P(\mathbb{F})$ is an infinite-dimensional vector space.
\end{example}

\begin{theorem}
Let $V$ be a vector space with dimension $n$, $S \subseteq V$. Then,
\begin{enumerate}[(a)]
    \item If $S$ is linearly independent, then $S$ has at most $n$ elements. Equivalently, if $S$ has more than $n$ elements, then $S$ is linearly dependent.
    \item If $S$ spans $V$, then $S$ has at least $n$ elements.
    \item $S$ is a basis of $V$ if and only if $S$ is linearly independent and contains exactly $n$ elements.
\end{enumerate}
\end{theorem}
\begin{proof}
\begin{enumerate}
    \item 
    \item Let $S$ span $V$. Then, there exists a subset $B \subseteq S$ such that $B$ is a basis of $V$. Since $\dim{V} = n$, $B$ contains $n$ vectors. Thus, $S$ contains at least $n$ vectors.
\end{enumerate}
\end{proof}

\section*{Basis Extension Theorem}
\begin{theorem}
Let $V$ be a vector space with dimension $n$, $B = \set{w_1, \dots, w_n}$ be a basis of $V$, $S = \set{v_1, \dots, v_k} \subseteq V$ be linearly independent. Then, $S$ can be extended to a basis of $V$, by adding $n-k$ vectors to $S$.
\end{theorem}

\section*{Basis Test}
\begin{theorem}
Let $V$ be a vector space with dimension $n$. Then,
\begin{enumerate}
    \item If $S = \set{v_1, \dots, v_n}$ is linearly independent, then $S$ is a basis for $V$.
    \item If $S = \set{v_1, \dots, v_n}$ spans $V$, then $S$ is a basis for $V$.
\end{enumerate}
\begin{proof}
EXERCISE.
\end{proof}

\end{theorem}


\end{document}